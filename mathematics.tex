\chapter{Mathematics}
\section{Calculus}

\subsection{Continuity and differentiability}

\begin{shaded}
\textbf{Definition (continuous function) \cite{hubbard_hubbard}.} Let $X \subset \mathbb{R}^n$. Then a mapping $\mathbf{f} : X \to \mathbb{R}^m$ is continuous at $\mathbf{x}_0 \in X$ if
$$ \lim_{\mathbf{x} \to \mathbf{x}_0} \mathbf{f}(\mathbf{x}) = \mathbf{f}(\mathbf{x}_0); $$
$\mathbf{f}$ is continuous on $X$ if it is continuous at every point of $X$.

Equivalently, $\mathbf{f}: X \to \mathbf{R}^m$ is continuous at $\mathbf{x}_0$ if and only if for every $\epsilon > 0$, there exists $\delta > 0$ such that when $|\mathbf{x} - \mathbf{x}_0| < \delta$, then $|\mathbf{f}(\mathbf{x}) - \mathbf{f}(\mathbf{x}_0)| < \epsilon$.
\end{shaded}

\begin{shaded}
\textbf{Definition (continuity and discontinuity) \cite{math2111_notes}.} Suppose $[a, b] \subset \mathbb{R}$. Consider a function $f : [a, b] \to \mathbb{R}$ and a point $c \in \mathbb{R}$. Supposed the one-sided limits
$$ f(c^-) = \lim_{x \to c^-} f(x) \qquad \textnormal{and} \qquad f(c^+) = \lim_{x \to c^+} f(x) $$
exist.

(1) If
$$ f(c^-) = f(c^+) = f(c) , $$
then $f$ is \textbf{continuous} at $c$.

(2) If
$$ f(c^-) = f(c^+) \not= f(c) $$
--- regardless of whether or not $f(c)$ exists --- then $f$ has a \textbf{removable discontinuity} at $c$.

(3) If
$$ f(c^-) \not= f(c^+) , $$
then $f$ has a \textbf{jump discontinuity} at $c$.
\end{shaded}

\begin{shaded}
\textbf{Definition (piecewise continuous function) \cite{math2111_notes}.} A piecewise function is continuous on $[a, b] \subseteq \mathbb{R}$ if

(1) $f(x^-)$ exists for each $x \in (a, b]$;

(2) $f(x^+)$ exists for each $x \in [a, b)$;

(3) $f$ is continuous on $(a, b)$ except at (at most) a finite number of points where there exist jump discontinuities.

Moreover, $f$ is piecewise continuous on $\mathbb{R}$ if it is piecewise continuous on any finite interval $[a, b] \subseteq \mathbb{R}$.
\end{shaded}

\begin{shaded}
\textbf{Definition (differentiability) \cite{math2111_notes}.} Consider a function $ f : \mathbb{R} \to \mathbb{R} $ and a point $c \in \mathbb{R}$. Let
$$ f(c^-) = \lim_{x \to c^-} f(x) \qquad \textnormal{and} \qquad f(c^+) = \lim_{x \to c^+} f(x) , $$
and let
\begin{align*}
	D^+ f(c) &= \lim_{h \to 0^+} \frac{f(c + h) - f(c^+)}{h} \qquad \text{and} \\
	D^- f(c) &= \lim_{h \to 0^-} \frac{f(c + h) - f(c^-)}{h} .
\end{align*}
%Then $f$ is differentiable at $c$ if and only if $f(c^+) = f(c^-)$ and $D^+ f(c) = D^- f(c)$.

Warning: $D^{+/-} f(c)$ is not necessarily the same as $f'(c^{+/-}) = \lim_{x \to c^{+/-}} f'(x)$.
\end{shaded}

\begin{shaded}
\textbf{Definition (piecewise differentiable function) \cite{math2111_notes}.} A function $f$ is piecewise differentiable on $[a, b] \subseteq \mathbb{R}$ if

(1) $D^- f(x)$ exists for each $x \in (a, b]$;

(2) $D^+ f(x)$ exists for each $x \in [a, b)$;

(3) $f$ is differentiable on $(a, b)$ except at (at most) a finite number of points.

Moreover, $f$ is piecewise differentiable on $\mathbb{R}$ if it is piecewise differentiable on any finite interval $[a, b] \subseteq \mathbb{R}$.
\end{shaded}

\subsection{Fourier series and convergence}

\begin{shaded}
\textbf{Fourier series \cite{math2111_notes}.} Consider a function $f : \mathbb{R} \to \mathbb{R}$ which is $2 L$-periodic and is square integrable (i.e., $\int_{-\pi}^\pi f(x)^2 \ dx < \infty $). Its Fourier series is given by
$$ S_f(x) = \frac{a_0}{2} + \sum_{k = 1}^{\infty} \left( a_k \cos \left( \frac{k \pi}{L}x \right) + b_k \sin \left( \frac{k \pi}{L}x \right) \right) . $$
We can view the Fourier series as expression a function $f$ as a linear combination of the orthogonal functions $\frac{1}{2}, \cos(kx), ..., \sin(kx), k \geq 1$. Then, via vector decomposition, we can find the following formulas to yield the Fourier coefficients $a_k$ and $b_k$:
$$ a _k = \frac{1}{L} \int_{-L}^L f(x) \cos \left( \frac{k \pi}{L}x \right) \ dx, \quad k = 0, 1, 2, ... $$
and
$$ b _k = \frac{1}{L} \int_{-L}^L f(x) \sin \left( \frac{k \pi}{L}x \right) \ dx, \quad k = 1, 2, 3, ... $$
\end{shaded}

\begin{shaded}
\textbf{Definition (pointwise convergence) \cite{math2111_notes}.} Let $f_k : \mathbb{R} \to \mathbb{R}$. We say $f_k$ converges to $f$ on $[a, b]$ pointwisely if $f_k(x) \to f(x)$ for every $x \in [a, b]$ as $k \to \infty$.
\end{shaded}

\begin{shaded}
\textbf{Theorem (pointwise convergence of Fourier series) \cite{math2111_notes}.} Let $c \in \mathbb{R}$, and suppose that a function $f : \mathbb{R} \to \mathbb{R}$ has the following properties:
\begin{enumerate}
	\item $f$ is $2 \pi$-periodic;
	\item $f$ is piecewise continuous on $[-\pi, \pi]$;
	\item $D^+ f(c)$ and $D^- f(c)$ exist.
\end{enumerate}
If $f$ is continuous at $c$, then the Fourier series of $f$ agrees with $f$ at $c$, i.e.,
$$ S f(c) = f(c) . $$
On the other hand, if $f$ has a jump discontinuity at $c$, then
$$ S f(c) = \frac{1}{2} \left( f(c^+) + f(c^-) \right) . $$
\end{shaded}

\begin{shaded}
\textbf{Definition (uniform convergence) \cite{math2111_notes}.} Let $f_k : \mathbb{R} \to \mathbb{R}$. We say $f_k$ converges to $f$ on $[a, b]$ uniformly if for every $\epsilon > 0$, there exists a $K$ (depends on $\epsilon$ only), such that
$$ \sup_{x \in [a, b]} | f_k(x) - f(x) | \leq \epsilon \quad \text{for all} \quad k \geq K . $$
Moreover, we say $\sum_{k = 1}^{\infty} f_k$ converges uniformly to $f$ if the partial sum $\tilde{f}_n = \sum_{k = 1}^{\infty} f_k$ converges uniformly to $f$ as $n \to \infty$

\textbf{Theorem.} If $f_k : \mathbb{R} \to \mathbb{R}$ is continuous on $[a, b]$ for all $n$ and if $f_k$ converge to $f$ uniformly on $[a, b]$, then $f$ is continuous on $[a, b]$.
\end{shaded}

\begin{shaded}
\textbf{Theorem (Weierstrass test) \cite{math2111_notes}.} Let $f_k : \mathbb{R} \to \mathbb{R}$ be a sequence of functions defined on $[a, b]$. Suppose that there exists a sequence of numbers $c_k$ such that
$$ |f_k(x)| \leq c_k \quad \text{for all} \quad x \in [a, b] $$
and $\sum_{k = 1}^\infty c_k$ exists. Then, $\sum_{k = 1}^\infty f_k(x)$ converges uniformly to a function $f$ on $[a, b]$.
\end{shaded}

\begin{shaded}
\textbf{Definition (norm convergence) \cite{math2111_notes}.} Consider the maximum norm $\lVert f \rVert_\infty = \sup_{x \in [a, b]} |f(x)|$. The definition of uniform convergence can be equivalently rewritten as: for every $\epsilon > 0$, there exists $K > 0$ such that
$$ \lVert f_k - f \rVert \leq \epsilon \quad \text{for all} \quad k \geq K , $$
or simply
$$ \lim_{k \to \infty} \lVert f_k - f \rVert = 0 . $$
\end{shaded}

\begin{shaded}
\textbf{Theorem (Parseval's theorem).} Let $f$ be $2 \pi$-periodic, bounded and $\int_{-\pi}^\pi f(x)^2 \ dx < + \infty$. Then, the Fourier series of $f$ converges to $f$ in the mean square sense, i.e.
$$ \int_{- \pi}^\pi (S_n f(x) - f(x))^2 \ dx \to 0 \quad \text{as} \quad k \to \infty . $$
That is to say, $\lVert S_n f - f \rVert_2 \to 0$ as $n \to \infty$. Moreover, the following \textbf{Parseval's identity} holds:
$$ \int_{-\pi}^\pi f(x)^2 \ dx = \lVert f \rVert_2^2 = \frac{\pi}{2} a_0^2 + \pi \sum_{k = 1}^{\infty} (a_k^2 + b_k^2) . $$
\end{shaded}

\subsection{Grad, div, curl}

\begin{shaded}
\textbf{Definition (gradient) \cite{thomas_calculus}.} The \textbf{gradient vector} (or \textbf{gradient}) of $f(x, y, z)$ is the vector
$$ \nabla f = \diffp{f}{x} \mathbf{i} + \diffp{f}{y} \mathbf{j} + \diffp{f}{z} \mathbf{k} . $$
The value of this gradient vector obtained by evaluating the partial derivatives at a point $P_0(x_0, y_0, z_0)$ is written
$$ \nabla f |_{P_0} \qquad \text{or} \qquad \nabla f(x_0, y_0, z_0) . $$
\end{shaded}

\begin{shaded}
\textbf{Definition (divergence) \cite{math2111_notes}.} If $\mathbf{F} = F_1 \mathbf{i} + F_2 \mathbf{j} + F_3 \mathbf{k}$, the \textbf{divergence} of $\mathbf{F}$ is the scalar field
\begin{align*}
\text{div} \ \mathbf{F} &= \nabla \cdot \mathbf{F} \\
&= \begin{pmatrix}[2] \diffp{}{x} & \diffp{}{y} & \diffp{}{z} \end{pmatrix}^{T}
\cdot \begin{pmatrix} F_1 \\ F_2 \\ F_3 \end{pmatrix} \\
&= \diffp{F_1}{x} + \diffp{F_2}{y} + \diffp{F_3}{z} .
\end{align*}
\end{shaded}

An \textbf{incompressible} liquid is one that has zero divergence.

\begin{shaded}
\textbf{Definition (curl) \cite{math2111_notes}.} If $\mathbf{F} = F_1 \mathbf{i} + F_2 \mathbf{j} + F_3 \mathbf{k}$, the \textbf{curl} of $\mathbf{F}$ is the vector field
\begin{align*}
\text{curl} \ \mathbf{F} &= \nabla \times \mathbf{F} \\
&= \begin{vmatrix}[1.5]
	\mathbf{i} & \mathbf{j} & \mathbf{k} \\
	\diffp{}{x} & \diffp{}{y} & \diffp{}{z} \\
	F_1 & F_2 & F_3
\end{vmatrix} \\
&= \left( \diffp{F_3}{y} - \diffp{F_2}{z} \right) \mathbf{i} + \left( \diffp{F_1}{z} - \diffp{F_3}{x} \right) \mathbf{j} + \left( \diffp{F_2}{x} - \diffp{F_1}{y} \right) \mathbf{k} .
\end{align*}
\end{shaded}

\subsection{Line integrals}

\textsc{Notation \cite{marsden_vector_calculus} \cite{thomas_calculus}.}
\begin{itemize}
	\item Velocity vector of a particle on a path
	$$\mathbf{c}(t) = x(t) \mathbf{i} + y(t) \mathbf{j} + z(t) \mathbf{k} \qquad \text{or} \qquad \mathbf{s} = x \mathbf{i} + y \mathbf{j} + z \mathbf{k} $$
is
$$ \mathbf{c}'(t) = \mathbf{v} = \diff{\mathbf{s}}{t} = \diff{x}{t} \mathbf{i} + \diff{y}{t} \mathbf{j} + \diff{z}{t} \mathbf{k} . $$
The speed of the particle is its magnitude: $\lVert \mathbf{c}'(t) \rVert = \lVert \mathbf{v} \rVert$.
	\item An \textbf{infinitesimal displacement} of a particle following the path $\mathbf{c}$ is
$$ d \mathbf{s} = dx \mathbf{i} + dy \mathbf{j} + dz \mathbf{k} = \left( \diff{x}{t} \mathbf{i} + \diff{y}{t} \mathbf{j} + \diff{z}{t} \mathbf{k} \right) \ dt , $$
and its length
$$ ds = \sqrt{dx^2 + dy^2 + dz^2} = \sqrt{{\diff{x}{t}}^2 + {\diff{y}{t}}^2 + {\diff{z}{t}}^2} \ dt $$
is the \textbf{differential of arc length}.
	\item The unit tangent vector is the velocity vector divided by its magnitude, i.e.
$$ \mathbf{T} = \frac{\mathbf{v}}{\lVert \mathbf{v} \rVert} . $$
	\item If $\mathbf{c}(t)$ is a smooth curve, then the principal unit normal is
$$ \mathbf{N} = \frac{d\mathbf{T} / dt}{\lVert d\mathbf{T} / dt \rVert} , $$
where $\mathbf{T} = \mathbf{v} / \lVert \mathbf{v} \rVert$ is the unit tangent vector. (More information about the derivation of this formula can be found in reference \cite{thomas_calculus} of the bibliography.)
	\item $dA$ is shorthand for $dx dy$
\end{itemize}

\begin{shaded}
\textbf{Definition (line integral of a scalar field) \cite{marsden_vector_calculus}.} The \textbf{line integral of a scalar field}, or \textbf{path integral}, or the \textbf{integral of $f(x, y, z)$ along the path $\mathbf{c}$}, is defined when $\mathbf{c}: I = [a, b] \to \mathbb{R}^3$ is of class $C^1$ and when the composite function $t \mapsto f(x(t), y(t), z(t))$ is continuous on $I$. We define this integral by the equation
$$ \int_{\mathbf{c}} f \ ds = \int_a^b f(x(t), y(t), z(t)) \ \lVert \mathbf{c}'(t) \rVert \ dt . $$
Sometimes $\int_\mathbf{c} f \ ds$ is denoted
$$ \int_\mathbf{c} f(x, y, z) \ ds $$
or
$$ \int_\mathbf{c} f(\mathbf{c}(t)) \ \lVert \mathbf{c}'(t) \rVert \ dt . $$

If $\mathbf{c}(t)$ is only piecewise $C^1$ or $f(\mathbf{c}(t))$ is piecewise continuous, we define $\int_\mathbf{c} f \ ds$ by breaking $[a, b]$ into pieces over which $f(\mathbf{c}(t)) \ \lVert \mathbf{c}'(t) \rVert$ is continuous, and summing the integrals over the pieces.
\end{shaded}

\begin{shaded}
\textbf{Definition (line integral of a vector field) \cite{marsden_vector_calculus} \cite{thomas_calculus}.} Let $\mathbf{F}$ be a vector field on $\mathbb{R}^3$ that is continuous on the $C^1$ path $\mathbf{c} : [a, b] \to \mathbb{R}^3$, and let $\mathbf{•}$ be the unit tangent vector to $\mathbf{c}$. We define $\int_\mathbf{c} \mathbf{F} \cdot d\mathbf{s}$, the \textbf{line integral} of $\mathbf{F}$ along $\mathbf{c}$, by the formula
\begin{align*}
\int_{\mathbf{c}} F_1 \ dx + F_2 \ dy + F_3 \ dz &= \int_a^b \left( F_1 \diff{x}{t} + F_2 \diff{y}{t} + F_3 \diff{z}{t} \right) \ dt \\
&= \int_\mathbf{c} \mathbf{F} \cdot d \mathbf{s} \\
&= \int_\mathbf{c} \left( \mathbf{F} \cdot \diff{\mathbf{s}}{s} \right) \ ds \\
&= \int_\mathbf{c} \mathbf{F} \cdot \mathbf{T} \ ds \\
&= \int_a^b \mathbf{F}(\mathbf{c}(t)) \cdot \mathbf{c}'(t) \ dt ;
\end{align*}
that is, we integrate the dot product of $\mathbf{F}$ with $\mathbf{c}'$ over the interval $[a, b]$.

As is the case with scalar functions, we can also define $\int_\mathbf{c} \mathbf{F} \cdot d\mathbf{s}$ if $\mathbf{F}(\mathbf{c}(t)) \cdot \mathbf{c}'(t)$ is only piecewise continuous.
\end{shaded}

\subsection{Definitions and theorems about line integrals}

\begin{shaded}
\textbf{Definition (arc length in $\mathbb{R}^n$) \cite{marsden_vector_calculus}.} Let $\mathbf{c}: [t_0, t_1] \to \mathbb{R}^n$ be a piecewise $C^1$ path. Its \textbf{length} is defined to be
$$ L(\mathbf{c} = \int_{t_0}^{t_1} \lVert \mathbf{c}'(t) \rVert \ dt . $$
The integrand is the square root of the sume of the squares of the coordinate functions of $\mathbf{c}'(t)$: If
$$ \mathbf{c}(t) = (x_1(t), x_2(t), ..., x_n(t)) , $$
then
$$ L(\mathbf{c}) = \sqrt{[x'_1(t)]^2 + [x'_2(t)]^2 + ... + [x'_n(t)]^2} \ dt . $$
\end{shaded}

\begin{shaded}
\textbf{Theorem (line integrals of gradient vector fields) \cite{marsden_vector_calculus}.} Suppose $f : \mathbb{R}^3 \to \mathbb{R}$ is of class $C^1$ and that $\mathbf{c} : [a, b] \to \mathbb{R}^3$ is a piecewise $C^1$ path. Then
$$ \int_{\mathbf{c}} \nabla f \cdot d\mathbf{s} = f(\mathbf{c}(b)) - f(\mathbf{c}(a)) . $$
\end{shaded}

\begin{shaded}
\textbf{Theorem (cross partials of gradient vector fields) \cite{math2111_notes}.} Let
$$ \mathbf{F} = (F_1, F_2, F_3) $$
be a gradient vector field whose components ahve continuous partial derivatives. Then the cross partials are equal:
$$ \diffp{F_1}{y} = \diffp{F_2}{x}, \quad \diffp{F_2}{z} = \diffp{F_3}{y}, \quad \diffp{F_3}{x} = \diffp{F_1}{z} . $$

Similarly, if the vector field in the plane
$$ \mathbf{F} = (F_1, F_2) $$
is a gradient vector field, then
$$ \diffp{F_1}{y} = \diffp{F_2}{x} . $$
\end{shaded}

\begin{shaded}
\textbf{Green's theorem \cite{thomas_calculus}.} Let $C$ be a piecewise smooth, simple closed curve enclosing a region $R$ in the
plane. Let $\mathbf{F} = P \mathbf{i} + Q \mathbf{j}$ be a vector field with $P$ and $Q$ having continuous first partial derivatives in an open region containing $R$. Then:
\begin{itemize}
	\item \textsc{Circulation-curl or tangential form} --- the anticlockwise circulation of $\mathbf{F}$ around $C$ equals the double integral of $(\text{curl} \ \mathbf{F}) \cdot \mathbf{k}$ over $R$.
\begin{align*}
\oint_C \mathbf{F} \cdot \mathbf{T} &= \oint_C P \ dx + Q \ dy \\
&= \iint_R (\text{curl} \ \mathbf{F}) \cdot \mathbf{k} \ dA \\
&= \iint_R \left( \diffp{Q}{x} - \diffp{P}{y} \right) \ dx dy
\end{align*}
	\item \textsc{Flux-divergence or normal form} --- the outward flux of $\mathbf{F}$ across $C$ equals the double integral of $\text{div} \ \mathbf{F}$ over the region $R$ enclosed by $C$.
\begin{align*}
\oint_C \mathbf{F} \cdot \mathbf{N} &= \oint_C P \ dy - Q \ dx \\
&= \iint_R \text{div} \ \mathbf{F} \ dA \\
&= \iint_R \left( \diffp{P}{x} + \diffp{Q}{y} \right) \ dx dy
\end{align*}
\end{itemize}
\end{shaded}

\subsection{Miscellaneous}

\begin{shaded}
\textbf{Even and odd functions.}
\begin{itemize}
	\item The product of two even functions is an even function.
	\item The product of two odd functions is an even function.
	\item The product of and even function and an odd function is an odd function.
\end{itemize}

This can be helpful when solving integrals of the form $\int f(x)g(x) \ dx$, where $f$ and $g$ are each even or odd.
\end{shaded}