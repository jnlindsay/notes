\chapter{Mathematics}
\section{Calculus}

\begin{shaded}
\textbf{Definition (Continuous function) \cite{hubbard_hubbard}.} Let $X \subset \mathbb{R}^n$. Then a mapping $\mathbf{f} : X \to \mathbb{R}^m$ is continuous at $\mathbf{x}_0 \in X$ if
$$ \lim_{\mathbf{x} \to \mathbf{x}_0} \mathbf{f}(\mathbf{x}) = \mathbf{f}(\mathbf{x}_0); $$
$\mathbf{f}$ is continuous on $X$ if it is continuous at every point of $X$.

Equivalently, $\mathbf{f}: X \to \mathbf{R}^m$ is continuous at $\mathbf{x}_0$ if and only if for every $\epsilon > 0$, there exists $\delta > 0$ such that when $|\mathbf{x} - \mathbf{x}_0| < \delta$, then $|\mathbf{f}(\mathbf{x}) - \mathbf{f}(\mathbf{x}_0)| < \epsilon$.
\end{shaded}

\begin{shaded}
\textbf{Definition (Continuity and discontinuity).} Suppose $[a, b] \subset \mathbb{R}$. Consider a function $f : [a, b] \to \mathbb{R}$ and a point $c \in \mathbb{R}$. Supposed the one-sided limits
$$ \lim_{x \to c^-} f(x) \qquad \textnormal{and} \qquad \lim_{x \to c^+} f(x) $$
exist.

(1) If
$$ \lim_{x \to c^-} f(x) = \lim_{x \to c^+} f(x) = f(c) , $$
then $f$ is \textbf{continuous} at $c$.

(2) If
$$ \lim_{x \to c^-} f(x) = \lim_{x \to c^+} f(x) \not= f(c) $$
--- regardless of whether or not $f(x)$ exists --- then $f$ has a \textbf{removable discontinuity} at $c$.

(3) If
$$ \lim_{x \to c^-} f(x) \not= \lim_{x \to c^+} f(x) , $$
then $f$ has a \textbf{jump discontinuity} at $c$.
\end{shaded}

\begin{shaded}
\textbf{Definition (Piecewise continuous function) \cite{wikipedia_piecewise}.} A piecewise function is continuous on a given interval if the following conditions are met:
\begin{itemize}
	\item it is defined throughout that interval,
	\item its constituent functions are continuous on the corresponding intervals (subdomains), and
	\item there is no discontinuity at each endpoint of the subdomains within that interval.
\end{itemize}
\end{shaded}