\documentclass[oneside]{book}

% PACKAGES
\usepackage[utf8]{inputenc}
\usepackage[
	backend=biber,
	urldate=long
]{biblatex}				            % bibliography
%\usepackage{xeCJK}					% Chinese
\usepackage{listings}               % code
\usepackage{csquotes}				% quotes
\usepackage{parskip}                % paragraph indentation
\usepackage{fancyhdr}				% fancy headers
\usepackage[yyyymmdd]{datetime}		% proper date format
\usepackage{tgschola}				% Tex Gyre Schola (based on New Century Schoolbook)
\usepackage{amsmath}
\usepackage{amsfonts}
\usepackage{amssymb}			    % symbols
\usepackage{amsthm}
\usepackage{mathtools}              % various
\usepackage{tcolorbox}				% boxes for theorems, definitions, etc.
\usepackage{xcolor}
\usepackage{bm}                     % bold symbols in maths mode
\usepackage{framed}    				% paragraph shading
\usepackage[hidelinks]{hyperref}    % clickable table of contents
\usepackage[italic]{esdiff}			% differentiation notation
\usepackage[shortlabels]{enumitem}

% SETTINGS
\colorlet{shadecolor}{lightgray!20}
\setlength{\parindent}{0em}
\addbibresource{bibliography.bib}
\tcbuselibrary{most}

% code
\newcommand{\code}{\lstinline} %} (fix for TexMaker parsing because it's retarded)
\lstset{
    backgroundcolor=\color[gray]{0.9},
    columns=fixed,
    basicstyle=\ttfamily\small,
    basewidth=0.5em,
    numbers=left,
    numbersep=10pt,
    xleftmargin=5pt,
    frame=lrtb,
    framesep=5pt,
    framerule=0pt,
    breaklines=true,
    language=Python,
    upquote=true
}


% 'THEOREM' ENVIRONMENTS
\newtcbtheorem[number within=section]{definition}{Definition}%
    {colback=blue!5, colframe=blue!35!black, sharp corners,
    	 fonttitle=\bfseries, parbox=false,
    	 separator sign={\ ---}}{def}

\newtcbtheorem[number within=section]{theorem}{Theorem}%
    {colback=red!5, colframe=red!35!black, sharp corners,
     fonttitle=\bfseries, parbox=false,
     separator sign={\ ---}}{th}
     
\newtcbtheorem[number within=section,
               use counter from=theorem]{lemma}{Lemma}%
    {colback=red!5, colframe=red!35!black, sharp corners,
     fonttitle=\bfseries, parbox=false,
     separator sign={\ ---}}{lem}
     
\newtcbtheorem[number within=section,
               use counter from=theorem]{corollary}{Corollary}%
    {colback=red!5, colframe=red!35!black, sharp corners,
     fonttitle=\bfseries, parbox=false,
     separator sign={\ ---}}{cor}
     
\newtcbtheorem[number within=section,
               use counter from=theorem]{law}{Law}%
    {colback=red!5, colframe=red!35!black, sharp corners,
     fonttitle=\bfseries, parbox=false,
     separator sign={\ ---}}{law}

\newtcbtheorem[number within=section]{formula}{Formula}%
    {colback=green!5, colframe=green!35!black, sharp corners,
     fonttitle=\bfseries, parbox=false,
     separator sign={\ ---}}{form}
       
\tcolorboxenvironment{proof}{% 'proof' from 'amsthm'
	blanker,breakable,left=3mm,
	before skip=10pt,after skip=10pt, parbox=false,
	borderline west={0.5mm}{0pt}{black}}

% REDEFINITIONS

% Credit to @StefanKottwitz on Tex Stack Exchange
% This redefinition allows matrices to be stretched on the fly like so:
% \begin{pmatrix}[1.5] ... \end{pmatrix}
\makeatletter
\renewcommand*\env@matrix[1][\arraystretch]{
  \edef\arraystretch{#1}%
  \hskip -\arraycolsep
  \let\@ifnextchar\new@ifnextchar
  \array{*\c@MaxMatrixCols c}}
\makeatother

% #################################################################

\title{%
Notes\\
\large Git hash: N/A\\
}
\author{Jeremy Lindsay}
\date{}

\begin{document}

\maketitle

\newpage

\tableofcontents

\chapter{Mathematics}

\section{Calculus}

\subsection{Continuity and differentiability}

\begin{theorem}{Limit of function of several variables using vector sequences \cite{math2111_notes}}{limit_function_sequences}
Given some function $f : \mathbb{R}^n \to \mathbb{R}$, the following identity holds:
$$
    \lim_{\mathbf{x} \to \mathbf{a}} f(\mathbf{x}) = b \quad
    \iff \quad \lim_{n \to \infty} f(\mathbf{x}_n) = b \qquad
    \forall \mathbf{x}_n \in \mathbb{R}^n : \lim_{n \to \infty} = \mathbf{a}
$$
\end{theorem}

\begin{definition}{Continuous function \cite{hubbard_hubbard}}{continuous_function}
Let $X \subset \mathbb{R}^n$. Then a mapping $\mathbf{f} : X \to \mathbb{R}^m$ is continuous at $\mathbf{x}_0 \in X$ if
$$ \lim_{\mathbf{x} \to \mathbf{x}_0} \mathbf{f}(\mathbf{x}) = \mathbf{f}(\mathbf{x}_0); $$
$\mathbf{f}$ is continuous on $X$ if it is continuous at every point of $X$.

Equivalently, $\mathbf{f}: X \to \mathbf{R}^m$ is continuous at $\mathbf{x}_0$ if and only if for every $\epsilon > 0$, there exists $\delta > 0$ such that when $|\mathbf{x} - \mathbf{x}_0| < \delta$, then $|\mathbf{f}(\mathbf{x}) - \mathbf{f}(\mathbf{x}_0)| < \epsilon$.
\end{definition}

\begin{definition}{Continuity and discontinuity \cite{math2111_notes}}{continuity_discontinuity}
Suppose $[a, b] \subset \mathbb{R}$. Consider a function $f : [a, b] \to \mathbb{R}$ and a point $c \in \mathbb{R}$. Supposed the one-sided limits
$$ f(c^-) = \lim_{x \to c^-} f(x) \qquad \textnormal{and} \qquad f(c^+) = \lim_{x \to c^+} f(x) $$
exist.

(1) If
$$ f(c^-) = f(c^+) = f(c) , $$
then $f$ is \textbf{continuous} at $c$.

(2) If
$$ f(c^-) = f(c^+) \not= f(c) $$
--- regardless of whether or not $f(c)$ exists --- then $f$ has a \textbf{removable discontinuity} at $c$.

(3) If
$$ f(c^-) \not= f(c^+) , $$
then $f$ has a \textbf{jump discontinuity} at $c$.
\end{definition}

\begin{definition}{Piecewise continuous function \cite{math2111_notes}}{piecewise_continuous_function}
A piecewise function is continuous on $[a, b] \subseteq \mathbb{R}$ if

(1) $f(x^-)$ exists for each $x \in (a, b]$;

(2) $f(x^+)$ exists for each $x \in [a, b)$;

(3) $f$ is continuous on $(a, b)$ except at (at most) a finite number of points where there exist jump discontinuities.

Moreover, $f$ is piecewise continuous on $\mathbb{R}$ if it is piecewise continuous on any finite interval $[a, b] \subseteq \mathbb{R}$.
\end{definition}

\begin{definition}{Differentiability) \cite{math2111_notes}}{differentiability}
Consider a function $ f : \mathbb{R} \to \mathbb{R} $ and a point $c \in \mathbb{R}$. Let
$$ f(c^-) = \lim_{x \to c^-} f(x) \qquad \textnormal{and} \qquad f(c^+) = \lim_{x \to c^+} f(x) , $$
and let
\begin{align*}
	D^+ f(c) &= \lim_{h \to 0^+} \frac{f(c + h) - f(c^+)}{h} \qquad \text{and} \\
	D^- f(c) &= \lim_{h \to 0^-} \frac{f(c + h) - f(c^-)}{h} .
\end{align*}
%Then $f$ is differentiable at $c$ if and only if $f(c^+) = f(c^-)$ and $D^+ f(c) = D^- f(c)$.

Warning: $D^{+/-} f(c)$ is not necessarily the same as
$$ f'(c^{+/-}) = \lim_{x \to c^{+/-}} f'(x) . $$
\end{definition}

\begin{definition}{Piecewise differentiable function \cite{math2111_notes}}{piecewise_differentiable_function}
A function $f$ is piecewise differentiable on $[a, b] \subseteq \mathbb{R}$ if

(1) $D^- f(x)$ exists for each $x \in (a, b]$;

(2) $D^+ f(x)$ exists for each $x \in [a, b)$;

(3) $f$ is differentiable on $(a, b)$ except at (at most) a finite number of points.

Moreover, $f$ is piecewise differentiable on $\mathbb{R}$ if it is piecewise differentiable on any finite interval $[a, b] \subseteq \mathbb{R}$.
\end{definition}

\subsection{Fourier series and convergence}

\begin{definition}{Fourier series \cite{math2111_notes}}{fourier_series}
Consider a function $f : \mathbb{R} \to \mathbb{R}$ which is $2 L$-periodic and is square integrable (i.e., $\int_{-\pi}^\pi f(x)^2 \ dx < \infty $). Its Fourier series is given by
$$ S_f(x) = \frac{a_0}{2} + \sum_{k = 1}^{\infty} \left( a_k \cos \left( \frac{k \pi}{L}x \right) + b_k \sin \left( \frac{k \pi}{L}x \right) \right) . $$
We can view the Fourier series as expression a function $f$ as a linear combination of the orthogonal functions $\frac{1}{2}, \cos(kx), ..., \sin(kx), k \geq 1$. Then, via vector decomposition, we can find the following formulas to yield the Fourier coefficients $a_k$ and $b_k$:
$$ a_k = \frac{1}{L} \int_{-L}^L f(x) \cos \left( \frac{k \pi}{L}x \right) \ dx, \quad k = 0, 1, 2, ... $$
and
$$ b_k = \frac{1}{L} \int_{-L}^L f(x) \sin \left( \frac{k \pi}{L}x \right) \ dx, \quad k = 1, 2, 3, ... $$
\end{definition}

\begin{definition}{Pointwise convergence \cite{math2111_notes}}{pointwise_convergence}
Let $f_k : \mathbb{R} \to \mathbb{R}$. We say $f_k$ converges to $f$ on $[a, b]$ pointwisely if $f_k(x) \to f(x)$ for every $x \in [a, b]$ as $k \to \infty$.
\end{definition}

\begin{theorem}{Pointwise convergence of Fourier series \cite{math2111_notes}}{pointwise_convergence_fourier_series}
Let $c \in \mathbb{R}$, and suppose that a function $f : \mathbb{R} \to \mathbb{R}$ has the following properties:
\begin{enumerate}
	\item $f$ is $2 \pi$-periodic;
	\item $f$ is piecewise continuous on $[-\pi, \pi]$;
	\item $D^+ f(c)$ and $D^- f(c)$ exist.
\end{enumerate}
If $f$ is continuous at $c$, then the Fourier series of $f$ agrees with $f$ at $c$, i.e.,
$$ S f(c) = f(c) . $$
On the other hand, if $f$ has a jump discontinuity at $c$, then
$$ S f(c) = \frac{1}{2} \left( f(c^+) + f(c^-) \right) . $$
\end{theorem}

\begin{definition}{Uniform convergence \cite{math2111_notes}}{uniform_convergence}
Let $f_k : \mathbb{R} \to \mathbb{R}$. We say $f_k$ converges to $f$ on $[a, b]$ uniformly if for every $\epsilon > 0$, there exists a $K$ (depends on $\epsilon$ only), such that
$$ \sup_{x \in [a, b]} | f_k(x) - f(x) | \leq \epsilon \quad \text{for all} \quad k \geq K . $$
Moreover, we say $\sum_{k = 1}^{\infty} f_k$ converges uniformly to $f$ if the partial sum $\tilde{f}_n = \sum_{k = 1}^{\infty} f_k$ converges uniformly to $f$ as $n \to \infty$

\textbf{Theorem.} If $f_k : \mathbb{R} \to \mathbb{R}$ is continuous on $[a, b]$ for all $n$ and if $f_k$ converge to $f$ uniformly on $[a, b]$, then $f$ is continuous on $[a, b]$.
\end{definition}

\begin{theorem}{Weierstrass test \cite{math2111_notes}}{weierstrass_test}
Let $f_k : \mathbb{R} \to \mathbb{R}$ be a sequence of functions defined on $[a, b]$. Suppose that there exists a sequence of numbers $c_k$ such that
$$ |f_k(x)| \leq c_k \quad \text{for all} \quad x \in [a, b] $$
and $\sum_{k = 1}^\infty c_k$ exists. Then, $\sum_{k = 1}^\infty f_k(x)$ converges uniformly to a function $f$ on $[a, b]$.
\end{theorem}

\begin{definition}{Norm convergence \cite{math2111_notes}}{norm_convergence}
Consider the maximum norm $\lVert f \rVert_\infty = \sup_{x \in [a, b]} |f(x)|$. The definition of uniform convergence can be equivalently rewritten as: for every $\epsilon > 0$, there exists $K > 0$ such that
$$ \lVert f_k - f \rVert \leq \epsilon \quad \text{for all} \quad k \geq K , $$
or simply
$$ \lim_{k \to \infty} \lVert f_k - f \rVert = 0 . $$
\end{definition}

\begin{theorem}{Parseval's theorem}{parsevals_theorem}
Let $f$ be $2 \pi$-periodic, bounded and $\int_{-\pi}^\pi f(x)^2 \ dx < + \infty$. Then, the Fourier series of $f$ converges to $f$ in the mean square sense, i.e.
$$ \int_{- \pi}^\pi (S_n f(x) - f(x))^2 \ dx \to 0 \quad \text{as} \quad k \to \infty . $$
That is to say, $\lVert S_n f - f \rVert_2 \to 0$ as $n \to \infty$. Moreover, the following \textbf{Parseval's identity} holds:
$$ \int_{-\pi}^\pi f(x)^2 \ dx = \lVert f \rVert_2^2 = \frac{\pi}{2} a_0^2 + \pi \sum_{k = 1}^{\infty} (a_k^2 + b_k^2) . $$
\end{theorem}

\subsection{Grad, div, curl}

\begin{definition}{Gradient \cite{thomas_calculus}}{gradient}
The \textbf{gradient vector} (or \textbf{gradient}) of $f(x, y, z)$ is the vector
$$ \nabla f = \diffp{f}{x} \mathbf{i} + \diffp{f}{y} \mathbf{j} + \diffp{f}{z} \mathbf{k} . $$
The value of this gradient vector obtained by evaluating the partial derivatives at a point $P_0(x_0, y_0, z_0)$ is written
$$ \nabla f |_{P_0} \qquad \text{or} \qquad \nabla f(x_0, y_0, z_0) . $$
\end{definition}

\begin{definition}{Divergence \cite{math2111_notes}}{divergence} If $\mathbf{F} = F_1 \mathbf{i} + F_2 \mathbf{j} + F_3 \mathbf{k}$, the \textbf{divergence} of $\mathbf{F}$ is the scalar field
\begin{align*}
\text{div} \ \mathbf{F} &= \nabla \cdot \mathbf{F} \\
&= \begin{pmatrix}[2] \diffp{}{x} & \diffp{}{y} & \diffp{}{z} \end{pmatrix}^{T}
\cdot \begin{pmatrix} F_1 \\ F_2 \\ F_3 \end{pmatrix} \\
&= \diffp{F_1}{x} + \diffp{F_2}{y} + \diffp{F_3}{z} .
\end{align*}
\end{definition}

An \textbf{incompressible} liquid is one that has zero divergence.

\begin{definition}{Curl \cite{math2111_notes}}{curl}
If $\mathbf{F} = F_1 \mathbf{i} + F_2 \mathbf{j} + F_3 \mathbf{k}$, the \textbf{curl} of $\mathbf{F}$ is the vector field
\begin{align*}
\text{curl} \ \mathbf{F} &= \nabla \times \mathbf{F} \\
&= \begin{vmatrix}[1.5]
	\mathbf{i} & \mathbf{j} & \mathbf{k} \\
	\diffp{}{x} & \diffp{}{y} & \diffp{}{z} \\
	F_1 & F_2 & F_3
\end{vmatrix} \\
&= \left( \diffp{F_3}{y} - \diffp{F_2}{z} \right) \mathbf{i} + \left( \diffp{F_1}{z} - \diffp{F_3}{x} \right) \mathbf{j} + \left( \diffp{F_2}{x} - \diffp{F_1}{y} \right) \mathbf{k} .
\end{align*}
\end{definition}

\subsection{Line integrals}

\textsc{Notation \cite{marsden_vector_calculus} \cite{thomas_calculus}.}
\begin{itemize}
	\item \textbf{Velocity vector} of a particle on a path
	$$\mathbf{c}(t) = x(t) \mathbf{i} + y(t) \mathbf{j} + z(t) \mathbf{k} \qquad \text{or} \qquad \mathbf{s} = x \mathbf{i} + y \mathbf{j} + z \mathbf{k} $$
is
$$ \mathbf{c}'(t) = \mathbf{v} = \diff{\mathbf{s}}{t} = \diff{x}{t} \mathbf{i} + \diff{y}{t} \mathbf{j} + \diff{z}{t} \mathbf{k} . $$
The \textbf{speed} of the particle is its magnitude: $\lVert \mathbf{c}'(t) \rVert = \lVert \mathbf{v} \rVert$.
	\item An \textbf{infinitesimal displacement} of a particle following the path $\mathbf{c}$ is
\begin{align*}
d \mathbf{s} &= dx \mathbf{i} + dy \mathbf{j} + dz \mathbf{k} \\
&= \left( \diff{x}{t} \mathbf{i} + \diff{y}{t} \mathbf{j} + \diff{z}{t} \mathbf{k} \right) \ dt \\
&= \mathbf{c}'(t) \ dt
\end{align*}
and its length
\begin{align*}
ds &= \sqrt{dx^2 + dy^2 + dz^2} \\
&= \sqrt{\left(\diff{x}{t}\right)^2 + \left(\diff{y}{t}\right)^2 + \left(\diff{z}{t}\right)^2} \ dt \\
&= \lVert \mathbf{c}'(t) \rVert \ dt
\end{align*}
is the \textbf{differential of arc length}.
	\item The \textbf{unit tangent vector} is the velocity vector divided by its magnitude, i.e.
$$ \mathbf{T} = \frac{\mathbf{v}}{\lVert \mathbf{v} \rVert} . $$
	\item If $\mathbf{c}(t)$ is a smooth curve, then the \textbf{principal unit normal} is
$$ \mathbf{N} = \frac{d\mathbf{T} / dt}{\lVert d\mathbf{T} / dt \rVert} , $$
where $\mathbf{T} = \mathbf{v} / \lVert \mathbf{v} \rVert$ is the unit tangent vector. (More information about the derivation of this formula can be found in reference \cite{thomas_calculus} of the bibliography.)
	\item $dA$ is shorthand for $dx dy$
\end{itemize}

\begin{definition}{Line integral of a scalar field \cite{marsden_vector_calculus}}{line_integral_scalar_field}
The \textbf{line integral of a scalar field}, or \textbf{path integral}, or the \textbf{integral of $f(x, y, z)$ along the path $\mathbf{c}$}, is defined when $\mathbf{c}: I = [a, b] \to \mathbb{R}^3$ is of class $C^1$ and when the composite function $t \mapsto f(x(t), y(t), z(t))$ is continuous on $I$. We define this integral by the equation
$$ \int_{\mathbf{c}} f \ ds = \int_a^b f(x(t), y(t), z(t)) \ \lVert \mathbf{c}'(t) \rVert \ dt . $$
Sometimes $\int_\mathbf{c} f \ ds$ is denoted
$$ \int_\mathbf{c} f(x, y, z) \ ds $$
or
$$ \int_\mathbf{c} f(\mathbf{c}(t)) \ \lVert \mathbf{c}'(t) \rVert \ dt . $$

If $\mathbf{c}(t)$ is only piecewise $C^1$ or $f(\mathbf{c}(t))$ is piecewise continuous, we define $\int_\mathbf{c} f \ ds$ by breaking $[a, b]$ into pieces over which $f(\mathbf{c}(t)) \ \lVert \mathbf{c}'(t) \rVert$ is continuous, and summing the integrals over the pieces.
\end{definition}

\begin{definition}{Line integral of a vector field \cite{thomas_calculus}}{line_integral_vector_field}
Let $\mathbf{F}$ be a vector field with continuous components defined along a smooth curve $C$ parametrised by $\mathbf{c}(t), a \leq t \leq b$. Then the \textbf{line integral of} $\mathbf{F}$ along $\mathbf{C}$ is
\begin{align*}
\int_C F_1 \ dx + F_2 \ dy + F_3 \ dz &= \int_a^b \left( F_1 \diff{x}{t} + F_2 \diff{y}{t} + F_3 \diff{z}{t} \right) \ dt \\
&= \int_C \mathbf{F} \cdot d \mathbf{s} \\
&= \int_C \left( \mathbf{F} \cdot \diff{\mathbf{s}}{s} \right) \ ds \\
&= \int_C \mathbf{F} \cdot \mathbf{T} \ ds \\
&= \int_a^b \mathbf{F}(\mathbf{c}(t)) \cdot \mathbf{c}'(t) \ dt
\end{align*}
\end{definition}

\begin{definition}{Flow integral, circulation \cite{thomas_calculus}}{flow_integral_circulation}
If $\mathbf{c}(t)$ parametrises a smooth curve $C$ in the domain of a continuous velocity field $\mathbf{F}$, the \textbf{flow} along the curve from $A = \mathbf{c}(a) to B = \mathbf{r}(b)$ is
$$ \text{Flow} = \int_C \mathbf{F} \cdot \mathbf{T} \ ds . $$
The integral is called a \textbf{flow integral}. If the curve starts and ends at the same point, so that $A = B$, the flow is called the \textbf{circulation} around the curve.
\end{definition}

\begin{definition}{Flux in the plane \cite{thomas_calculus}}{flux_plane}
If $C$ is a smooth  simple closed curve in the domain of a continuous vector field $F = P(x, y) \mathbf{i} + Q(x, y) \mathbf{j}$ in the plane, and if $\mathbf{n}$ is the outward-pointing unit normal vector on $C$, the \textbf{flux} of $\mathbf{F}$ across $C$ is
\begin{align*}
\text{Flux of } \mathbf{F} \text{ across } C &= \int_C \mathbf{F} \cdot \mathbf{n} \ ds \\
&= \oint_C P \ dy - Q \ dx .
\end{align*}
The integral can be evaluated from any smooth parametrisation $x = g(t), y = h(t), a \leq t \leq b$, that traces $C$ anticlockwise exactly once.
\end{definition}

\subsection{Definitions and theorems about line integrals}

\begin{definition}{Arc length in $\mathbb{R}^n$ \cite{marsden_vector_calculus}}{arc_length_Rn}
Let $\mathbf{c}: [t_0, t_1] \to \mathbb{R}^n$ be a piecewise $C^1$ path. Its \textbf{length} is defined to be
$$ L(\mathbf{c} = \int_{t_0}^{t_1} \lVert \mathbf{c}'(t) \rVert \ dt . $$
The integrand is the square root of the sume of the squares of the coordinate functions of $\mathbf{c}'(t)$: If
$$ \mathbf{c}(t) = (x_1(t), x_2(t), ..., x_n(t)) , $$
then
$$ L(\mathbf{c}) = \sqrt{[x'_1(t)]^2 + [x'_2(t)]^2 + ... + [x'_n(t)]^2} \ dt . $$
\end{definition}

\begin{theorem}{Fundamental theorem of line integrals \cite{thomas_calculus}}{fundamental_line_integrals}
Let $C$ be a smooth curve joining the point $A$ to the point $B$ in the plane or in space and parametrised by $\mathbf{c}(t)$. Let $f$ be a differentiable function with a continuous gradient vector $\mathbf{F} = \nabla f$ on a domain $D$ containing $C$. Then
$$ \int_C \mathbf{F} \cdot d \mathbf{r} = f(B) - f(A) . $$
\end{theorem}

\begin{theorem}{Cross partials of gradient vector fields \cite{math2111_notes}}{cross_partials_gradient_vector_fields}
Let
$$ \mathbf{F} = (F_1, F_2, F_3) $$
be a gradient vector field whose components have continuous partial derivatives. Then the cross partials are equal:
$$ \diffp{F_1}{y} = \diffp{F_2}{x}, \quad \diffp{F_2}{z} = \diffp{F_3}{y}, \quad \diffp{F_3}{x} = \diffp{F_1}{z} . $$

Similarly, if the vector field in the plane
$$ \mathbf{F} = (F_1, F_2) $$
is a gradient vector field, then
$$ \diffp{F_1}{y} = \diffp{F_2}{x} . $$
\end{theorem}

\begin{theorem}{Green's theorem \cite{thomas_calculus}}{greens_theorem}
Let $C$ be a piecewise smooth, simple closed curve enclosing a region $R$ in the
plane. Let $\mathbf{F} = P \mathbf{i} + Q \mathbf{j}$ be a vector field with $P$ and $Q$ having continuous first partial derivatives in an open region containing $R$. Then:
\begin{itemize}
	\item \textsc{Circulation-curl or tangential form} --- the anticlockwise circulation of $\mathbf{F}$ around $C$ equals the double integral of $(\text{curl} \ \mathbf{F}) \cdot \mathbf{k}$ over $R$.
\begin{align*}
\oint_C \mathbf{F} \cdot \mathbf{T} &= \oint_C P \ dx + Q \ dy \\
&= \iint_R (\text{curl} \ \mathbf{F}) \cdot \mathbf{k} \ dA \\
&= \iint_R \left( \diffp{Q}{x} - \diffp{P}{y} \right) \ dx dy
\end{align*}
	\item \textsc{Flux-divergence or normal form} --- the outward flux of $\mathbf{F}$ across $C$ equals the double integral of $\text{div} \ \mathbf{F}$ over the region $R$ enclosed by $C$.
\begin{align*}
\oint_C \mathbf{F} \cdot \mathbf{n} &= \oint_C P \ dy - Q \ dx \\
&= \iint_R \text{div} \ \mathbf{F} \ dA \\
&= \iint_R \left( \diffp{P}{x} + \diffp{Q}{y} \right) \ dx dy
\end{align*}
\end{itemize}
\end{theorem}

\subsection{Miscellaneous}
\textbf{Even and odd functions.}
\begin{itemize}
	\item The product of two even functions is an even function.
	\item The product of two odd functions is an even function.
	\item The product of and even function and an odd function is an odd function.
\end{itemize}

This can be helpful when solving integrals of the form $\int f(x)g(x) \ dx$, where $f$ and $g$ are each even or odd.
\section{Linear algebra}

\subsection{Matrices}

\subsubsection{Inverse matrices}

\begin{definition}{Inverse matrix \cite{math1141_notes}}{inverse_matrix}
A matrix $X$ is said to be the \textbf{inverse} of a matrix $A$ if both
$$ AX = I \qquad \text{and} \qquad XA = I , $$
where $I$ is the identity matrix of appropriate size.
\end{definition}

If a matrix $A$ has an inverse, then $A$ is said to be an \textbf{invertible matrix}. If a matrix $A$ is not an invertible matrix, then it is called a \textbf{singular} matrix.

\begin{lemma}{\cite{math1141_notes}}{}
All invertible matrices are square.
\end{lemma}

\begin{lemma}{}{singular_matrix}
A (square) matrix is singular if its determinant is $0$.
\end{lemma}

\begin{definition}{Right/left inverse \cite{math1141_notes}}{right_left_inverse}
\begin{itemize}
	\item An $n \times m$ matrix $X$ is said to be a \textbf{right inverse} of the $m \times n$ matrix $A$ if
$$ AX = I_m ; $$

	\item An $n \times m$ matrix $Y$ is said to be a \textbf{left inverse} of the $m \times n$ matrix $A$ if
$$ YA = I_n . $$
\end{itemize}
\end{definition}

\textbf{Finding the inverse of a matrix}. A matrix $A$ is invertible if and only if it can be reduced by elementary row operations to an identity matrix $I$ and if $(A|I)$ can be reduced to $(I|B)$, in which case, $B = A^{-1}$.

\begin{formula}{Inverse of a $2 \times 2$ matrix}{inverse_2b2_matrix}
$$
\begin{pmatrix}
	a & b \\
	c & d
\end{pmatrix}^{-1}
= \dfrac{1}{ad - bc}
\begin{pmatrix}
	d & -b \\
	-c & a
\end{pmatrix} , \quad
\text{provided } ad - bc \not = 0.
$$
\end{formula}

\subsubsection{Determinants}

\begin{definition}{Determinant of a $2 \times 2$ matrix}{determinant_2b2}
Given the matrix
$$ A = \begin{pmatrix}
a_{11} & a_{12} \\
a_{21} & a_{22}
\end{pmatrix} , $$
its \textbf{determinant} is
$$ \det (A) = a_{11} a_{22} - a_{12} a_{21} . $$
\end{definition}

\begin{definition}{Determinant}{determinant}
First, define the row $i$, column $j$ \textbf{minor} of a matrix $X$ to be the resulting matrix obtained by deleting row $i$ and column $j$ from $X$, denoted by $ \lvert X_{ij} \rvert$.

Then the \textbf{determinant} of an $n \times n$ matrix $A$ is
\begin{align*}
\lvert A \rvert &=
a_{11} \lvert A_{11} \rvert 
- a_{12} \lvert A_{12} \rvert
+ a_{13} \lvert A_{13} \rvert
- \dots +
(-1)^{1+n} a_{1n} \lvert A_{1n} \rvert \\
&= \sum_{k = 1}^n (-1)^{1+k} a_{1k} \lvert A_{1k} \rvert .
\end{align*}
\end{definition}

\subsection{Groups and fields}

\begin{definition}{Group \cite{math2601_notes}}{group}
A \textbf{group} is a a set, $G$, together with an operation $*$ (called the \textbf{group law} of $G$) that combines any two elements $a$ and $b$ to form another element, denoted $a * b$ or $ab$. To qualify as a group, the set and operation, $(G, *)$, must satisfy four requirements known as the \textbf{group axioms}:
\begin{itemize}
 	\item \textbf{Closure.} For all $a, b$ in $G$, the result of the operation, $a * b$, is also in $G$.
 	\item \textbf{Associativity.} For all $a, b$ and $c$ in $G, (a * b) * c = a * (b * c)$.
 	\item \textbf{Identity element.} There exists an element $e$ in $G$ such that, for every element $a$ in $G$, the equation $e * a = a * e = a$ holds. Such an element is unique, and thus one speaks of \textit{the} identity element.
 	\item \textbf{Inverse element.} For each $a$ in $G$, there exists an element $b$ in $G$, commonly denoted $a^{-1}$ (or $-a$, if the operation is denoted ``$+$''), such that $a * b = b * a = e$, where $e$ is the identity element.
\end{itemize}
\end{definition}

A group $G$ is \textbf{abelian} if the operation satisfies the \textbf{commutative law}
$$ a * b = b * a \qquad \text{for all} \ a, b \in G . $$

We often write the \textbf{cyclic group of order $m$} as
$$ C_m = \langle a : a^m = e \rangle , $$
and say it is \textit{generated} by $a$.


\begin{definition}{Subgroup \cite{math2601_notes}}{subgroup}
Let $(G, *)$ be a group and $H$ a non-empty subset of $G$. If $H$ is a group under the restriction of $*$ to $H$, we call it a \textbf{subgroup} of $G$. We write this as $H \leq G$ and say $H$ \textit{inherits} the group structure from $G$.
\end{definition}

\begin{lemma}{\cite{math2601_notes}}{subgroup}
Let $(G, *)$ be a group and $H$ a non-empty subset of $G$. Then $H$ is a subgroup of $G$ if and only if
\begin{itemize}
	\item for all $a, b \in H, \ a * b \in H$;
	\item for all $a \in H, \ a^{-1} \in H$,
\end{itemize}
i.e. $H$ is closed under $*$ and $^{-1}$.
\end{lemma}

\begin{definition}{Field \cite{math2601_notes}}{field}
A \textbf{field}, $(\mathbb{F}, +, \times)$ is a set $\mathbb{F}$ with two binary operations on it --- addition ($+$) and multiplication ($\times$) --- where
\begin{enumerate}
	\item $(\mathbb{F}, +)$ is an abelian group,
	\item $\mathbb{F}^{*} = \mathbb{F} \setminus \{0\}$ (where 0 is the additive identity) is an abelian group under multiplication,
	\item the distributive laws $a \times (b + c) = a \times b + a \times c$ and $(a + b) \times c = a \times c + b \times c$ hold.
\end{enumerate}
\end{definition}

\begin{definition}{Subfield \cite{math2601_notes}}{subfield}
If $(\mathbb{F}, +, \times)$ is a field and $\mathbb{E} \subset \mathbb{F}$ is also a field under the same operations (restricted to $\mathbb{E}$), then $(\mathbb{E}, +, \times)$ is a \textbf{subfield} of $(\mathbb{F}, +, \times)$, usually written $\mathbb{E} \leq \mathbb{F}$.
\end{definition}

\begin{lemma}{\cite{math2601_notes}}{subfield}
Let $\mathbb{E} \not = \{0\}$ be a non-empty subset of field $\mathbb{F}$. Then $\mathbb{E}$ is a subfield of $\mathbb{F}$ if and only if for all $a, b \in \mathbb{E}$:
$$ a + b \in \mathbb{E}, \quad -b \in \mathbb{E}, \quad a \times b \in \mathbb{E}, \quad b^{-1} \in \mathbb{E} \text{ if } b \not = 0 . $$

\begin{proof}
The distributive laws are inherited from $\mathbb{F}$ to $\mathbb{E}$. The rest of the proof follows from applying the subgroup lemma \ref{lem:subgroup} to both $(\mathbb{E}, +)$ and $(\mathbb{E}^*, \times)$.
\end{proof}
\end{lemma}

\begin{definition}{General linear group \cite{math2601_notes}}{general_linear_group}
Let $n \geq 1$ be any integer. The \textbf{general linear group}, $\operatorname{GL}(n, \mathbb{F})$ is the set of invertible $n \times n$ matrices over field $\mathbb{F}$ under matrix multiplication, and is non-abelian if $n > 1$.
\end{definition}

Subgroups of $\operatorname{GL}(n, \mathbb{F})$ include
\begin{itemize}
	\item the \textbf{special linear groups} $\operatorname{SL}(n, \mathbb{R})$ and $\operatorname{SL}(n, \mathbb{C})$ of matrices with determinant 1;
	\item $\operatorname{O}(n) \leq \operatorname{GL}(n, \mathbb{R})$, the group of \textbf{orthogonal matrices};
	\item $\operatorname{SO}(n) = \operatorname{O}(n) \cap \operatorname{SL}(n, \mathbb{R})$, the group of \textbf{special orthogonal matrices}.
\end{itemize}

\begin{definition}{Group homomorphism \cite{math2601_notes}}{group_homomorphism}
Let $(G, *)$ and $(H, \circ)$ be two groups. A \textbf{(group) homomorphism} from $G$ to $H$ is a map $\phi : G \to H$ that respects the two operations, that is where
$$ \phi (a * b) = \phi (a) \circ \phi (b) \qquad \text{for all } a, b \in G . $$
A bijective homomorphism $\phi : G \to H$ is called an \textbf{isomorphism}: the groups are then said to be \textbf{isomorphic}.
\end{definition}

\begin{lemma}{\cite{math2601_notes}}{homomorphism}
Let $(G, *)$ and $(H, \circ)$ be two groups and $\phi$ a homomorphism between them. Then
\begin{itemize}
	\item $\phi$ maps the identity of $G$ to the identity of $H$;
	\item $\phi$	 maps inverse to inverses, i.e. $\phi (a^{-1}) = (\phi(a))^{-1}$ for all $a \in G$;
	\item if $\phi$ is an isomorphism from $G$ to $H$ then $\phi^{-1}$ is an isomorphism from $H$ to $G$.
\end{itemize}
\end{lemma}

\begin{definition}{Kernel and image \cite{math2601_notes}}{kernel_and_image}
Let $\phi : G \to H$ be a group homomorphism, with $e'$ the identity of $H$.

The \textbf{kernel} of $\phi$ is the set
$$ \ker (\phi) = \{ g \in G : \phi (g) = e' \} . $$

The \textbf{image} of $\phi$ is the set
$$ \operatorname{im} (\phi) = \{ h \in H : h = \phi (g), \text{ some } g \in G \} . $$
\end{definition}

\begin{lemma}{\cite{math2601_notes}}{kernel_image_subgroups}
For $\phi : G \to H$, a group homomorphism, $\ker \phi \leq G$ and $\operatorname{im} \phi \leq H$.

\begin{proof}
Let $e$ be the identity of $G$. From lemma \ref{lem:homomorphism}, $e \in \ker \phi$, so the kernel is non-empty.

If $a, b \in \ker \phi$ then
$$ \phi (a * b) = \phi (a) \circ \phi (b) = e' \circ e' = e' , $$
so $a * b \in \ker \phi$.

If $a \in \ker \phi$ then from lemma \ref{lem:homomorphism},
$$ \phi (a^{-1}) = (\phi (a))^{-1} = (e')^{-1} = e' , $$
and so $a^{-1} \in \ker \phi$.

Thus by the subgroup lemma \ref{lem:homomorphism}, $\ker \phi \leq G$.

\textit{The proof that $\operatorname{im} \phi \leq H$ is omitted.}
\end{proof}
\end{lemma}

\begin{lemma}{\cite{math2601_notes}}{}
A group homomorphism $\phi : G \to H$ is one-to-one if and only if $\ker \phi = \{e\}$, with $e$ the identity of $G$; if $\phi$ is one-to-one then $\operatorname{im} \phi$ is isomorphic to $G$.

\begin{proof}
From lemma \ref{lem:kernel_image_subgroups}, $e \in \ker \phi$, and if $\phi$ is one-to-one, $e$ is the only element that maps to $e'$, the identity of $H$.

Conversely, suppose $\ker \phi = \{e\}$ and $\phi (a) = \phi (b)$, where $a, b \in G$. Then
$$ \phi (a * b^{-1}) = \phi (a) \circ (\phi (b))^{-1} = e' $$
and so $a * b^{-1} = e$ and $a = b$.

If $\phi$ is one-to-one, it is a bijection from $G$ to $\operatorname{im} \phi$, and hence an isomorphism.
\end{proof}
\end{lemma}

A common use of group homomorphisms is to look for a homomorphism $\phi : G \to \operatorname{GL}(n, \mathbb{F})$ for some $n$ and some field $F$. The group $\operatorname{im} (\phi)$ is called a \textbf{(linear) representation of $G$ on $\mathbb{F}^n$}.

If $\phi$ is one-to-one (so every element maps to a distinct matrix), we call the representation \textbf{faithful}.

\subsection{Vector spaces}

\begin{definition}{Vector space \cite{math2601_notes}}{vector_space}
Let $\mathbb{F}$ be a field. A \textbf{vector space over the field} $\mathbb{F}$ consists of an abelian group $(V, +)$ plus a function from $\mathbb{F} \times V$ to $V$ called \textbf{scalar multiplication} and written $\alpha \mathbf{v}$, where
\begin{itemize}
	\item $\alpha (\beta \mathbf{v}) = (\alpha \beta) \mathbf{v}$ for all $\alpha, \beta \in \mathbb{F}$ for all $\mathbf{v} \in V$.
	\item $1 \mathbf{v} = \mathbf{v}$ for all $\mathbf{v} \in V$.
	\item $\alpha (\mathbf{u} + \mathbf{v}) = \alpha \mathbf{u} + \alpha \mathbf{v}$ for all $\alpha \in \mathbb{F}$ and for all $\mathbf{u}, \mathbf{v} \in V$.
	\item $(\alpha + \beta) \mathbf{u} = \alpha \mathbf{u} + \beta \mathbf{u}$ for all $\alpha, \beta \in \mathbb{F}$ for all $\mathbf{u} \in V$.
\end{itemize}
\end{definition}

\begin{lemma}{\cite{math2601_notes}}{}
Let $V$ be a vector space over field $\mathbb{F}$. For all $\mathbf{v}, \mathbf{w} \in V$ and $\lambda \in \mathbb{F}$:
\begin{itemize}
	\item $0 \mathbf{v} = \mathbf{0}$ and $\lambda \mathbf{0} = \mathbf{0}$;
	\item $(-1) \mathbf{v} = - \mathbf{v}$;\
	\item $\lambda \mathbf{v} = \mathbf{0}$ implies either $\lambda = 0$ or $\mathbf{v} = \mathbf{0}$;
	\item if $\lambda \mathbf{v} = \lambda \mathbf{w}$ and $\lambda \not = 0$ then $\mathbf{v} = \mathbf{w}$.
\end{itemize}
\end{lemma}
\section{Statistics}

\subsection{Probability}

\subsubsection{Probability spaces}

\begin{definition}{Probability space \cite{math2901_notes}}{probability_space}
In probability theory, a \textbf{probability space} or a \textbf{probability triple} $(\Omega, \mathcal{F}, P)$ is a mathematical construct that provides a formal model of a random process or ``experiment''.

A probability space consists of three elements:
\begin{enumerate}
	\item A \textbf{sample space} $\Omega$, which is the set of all possible outcomes.
	\item An \textbf{event space}, which is a set of events $\mathcal{F}$, an event being a set of outcomes in the sample space.
	\item A \textbf{probability function}, which assigns each event in the event space a probability, which is a number between 0 and 1.
\end{enumerate}
\end{definition}

The minimal assumption that we impose on $\mathcal{F}$ is that it should be an object called a $\sigma$-algebra.

\begin{definition}{$\sigma$-algebra \cite{wikipedia_sigma_algebra}}{sigma-algebra}
Let $X$ be some set, and let $\mathcal{P}(X)$ represent its power set. Then a subset $\Sigma \subseteq \mathcal{P}(X)$ is called a $\bm{\sigma}$\textbf{-algebra} if it satisfies the following three properties:

\begin{enumerate}
	\item $X$ is in $\Sigma$, and $X$ is considered to be the universal set in the following context;
	\item $\Sigma$ is closed under complementation: if $A$ is in $\Sigma$, then so is its complement $X \setminus A$;
	\item $\Sigma$ is closed under countable unions: if $A_1, A_2, A_3, \ldots$ are in $\Sigma$, then so is $A = A_1 \cup A_2 \cup A_3 \cup \ldots$
\end{enumerate}
\end{definition}

Given a sample space $(\Sigma, \mathcal{F})$, a probability function $\mathbb{P}$ can be defined in the following way: to every event $A \in \mathcal{F}$ we assign a number $\mathbb{P} (A)$, the \textbf{probability that $A$ occurs}. The function $\mathbb{P}$ must satisfy the axioms

\begin{enumerate}
	\item $\mathbb{P} (A) \geq 0$ for each $A \subset \Omega$;
	\item $\mathbb{P} (\Omega) = 1$;
	\item if $A_1, A_2, \ldots$ are mutually exclusive (disjoint), i.e.
	$$ A_i \cap A_j = \emptyset \text{ for all } i, j \text{ with } i \not = j  ,$$
	then
	$$ \mathbb{P} \left( \bigcup_{i = 1}^{\infty} A_i \right) = \sum_{i = 1}^{\infty} \mathbb{P} (A_i) . $$
\end{enumerate}

\begin{lemma}{\cite{math2901_notes}}{}
\begin{enumerate}
	\item If $A_1, A_2, \ldots, A_k$ are mutually exclusive, then
	$$ \mathbb{P} \left( \bigcup_{i = 1}^{k} A_i \right) = \sum_{i = 1}^{k} \mathbb{P} (A_i) ; $$
	\item $\mathbb{P} (\emptyset) = 0$;
	\item for any $A \subseteq \Omega$, it holds that $0 \leq \mathbb{P} (A) \leq 1$ and $\mathbb{P} (\bar A) = 1 - \mathbb{P} (A)$;
	\item if $B \subset A$, then $\mathbb{P} (B) \leq \mathbb{P} (A)$.
\end{enumerate}
\end{lemma}

\subsubsection{Monotonic sequences of events}

\begin{theorem}{Continuity properties \cite{math2901_notes}}{continuity_properties}
If $A_1, A_2, \ldots$ is an increasing sequence of events, i.e., $A_1 \subset A_2 \subset \dots$, then
$$ \lim_{n \to \infty} \mathbb{P} (A_n) = \mathbb{P} \left( \bigcup_{n = 1}^{\infty} \right) . $$
We say that $\mathbb{P}$ is \textbf{continuous from below}.

If $A_1, A_2, \ldots$ is a decreasing sequence of events, i.e., $A_1 \supseteq A_2 \supseteq \dots$, then
$$ \lim_{n \to \infty} \mathbb{P} (A_n) = \mathbb{P} \left( \bigcap_{n = 1}^{\infty} \right) . $$
We say that $\mathbb{P}$ is \textbf{continuous from above}.
\end{theorem}

\subsubsection{Conditional probability}

\begin{definition}{Conditional probability \cite{math2901_notes}}{conditional_probability}
The \textbf{conditional probability} that an event $A$ occurs, given that an event $B$ has occurred, is
$$ \mathbb{P} (A | B) = \dfrac{\mathbb{P} (A \cap B)}{\mathbb{P} (B)} \text{ if } \mathbb{P} (B) \not = 0 . $$
\end{definition}

\begin{lemma}{\cite{math2901_notes}}{conditional_probability}
$$ \mathbb{P} (A|B) = \mathbb{P} (A) \iff \mathbb{P} (B|A) = \mathbb{P} (B) $$
\end{lemma}

\subsubsection{Independent events}

\begin{definition}{Independent events \cite{math2901_notes}}{independent_events}
For a countable sequence of events $\{A_i\}$, the events are  
\begin{itemize}
	\item \textbf{pairwise independent} if
$$ \mathbb{P} (A_i \cap A_j) = \mathbb{P} (A_i) \mathbb{P} (A_j) \text{ for all } i \not = j ; $$

	\item \textbf{(mutually) independent} if for any sub-collection $A_{i_1}, \ldots, A_{i_n}$ we have
$$ \mathbb{P} \left( \bigcap_{j = 1}^n A_{i_j} \right) = \prod_{j = 1}^n \mathbb{P} (A_{i_j}) . $$
\end{itemize}
\end{definition}

Note that for any two events $A$ and $B$, $\mathbb{P} (A \cap B) = \mathbb{P} (A|B) \mathbb{P} (B)$, so $A$ and $B$ are independent if and only if the equalities in lemma \ref{lem:conditional_probability} are true.

Also note that mutual independence implies pairwise independence (but not vice versa).

\subsubsection{Probability laws}

\begin{theorem}{Multiplicative law}{multiplicative}
For events $A_1, A_2$,
$$ \mathbb{P} (A_1 \cap A_2) = \mathbb{P} (A_2 \cap A_1) = \mathbb{P} (A_2 \vert A_1) \mathbb{P} (A_1) . $$

For events $A_1, A_2, A_3$,
\begin{align*}
\mathbb{P} (A_1 \cap A_2 \cap A_3)
&= \mathbb{P} (A_3 \cap A_2 \cap A_1) \\
&= \mathbb{P} (A_3 \vert A_2 \cap A_1) \mathbb{P} (A_2 \cap A_1) \\
&= \mathbb{P} (A_3 \vert A_1 \cap A_2) \mathbb{P} (A_2 \vert A_1) \mathbb{P} (A_1) .
\end{align*}

(The same pattern applies to higher numbers of events.)
\end{theorem}

\begin{theorem}{Additive Law}{additive}
For events $A$ and $B$,
$$ \mathbb{P} (A \cup B) = \mathbb{P}(A) + \mathbb{P}(B) - \mathbb{P}(A \cap B) . $$
\end{theorem}

\begin{theorem}{Total probability}{total_probability}
Suppose $A_1, A_2, \ldots, A_k$ are mutually exclusive ---
$$ A_i \cap A_j = \emptyset \quad \text{for all } i \not = j $$
--- and \textbf{exhaustive} ---
$$ \bigcup_{i = 1}^k A_i = \Omega = \text{sample space} $$
--- that is, $A_1, \ldots, A_k$ form a \textbf{partition} of $\Omega$. Then, for any event $B$,
$$ \mathbb{P}(B) = \sum_{i=1}^k \mathbb{P}(B \vert A_i) \mathbb{P}(A_i) . $$
\end{theorem}

\begin{theorem}{Bayes' theorem}{bayes}
For a partition $A_1, A_2, \ldots, A_k$ and an event $B$,
$$
\mathbb{P}(A_j \vert B) 
= \dfrac{\mathbb{P}(B \vert A_j) \mathbb{P}(A_j)}{\mathbb{P}(B)} \\
= \dfrac{\mathbb{P}(B \vert A_j) \mathbb{P}(A_j)}{\sum_{i=1}^k \mathbb{P}(B|A_i) \mathbb{P}(A_i)}
$$
\end{theorem}

\subsection{Statistics}

\begin{definition}{Random variable \cite{wikipedia_random_variable}}{random_variable}
	A \textbf{random variable} is a measurable function $X : \Omega \to E$ from a set of possible outcomes $\Omega$ to a measurable space $E$. The technical axiomatic definition requires $\Omega$ to be a sample space of a probability triple $(\Omega, \mathcal{F}, P)$.

	The probability that $X$ takes on a vaue in a measurable set $S \subseteq E$ is written as
	$$ P(X \in S) = P(\{ \omega \in \Omega \vert X(\omega) \in S \}) . $$
\end{definition}

When the image (or range) of $X$ is countable, the random variable is called a \textbf{discrete random variable} and its distribution is a discrete probability distribution, i.e. can be described by a probability mass function that assigns a probability to each value in the image of $X$. If the image is uncountably infinite then $X$ is called a \textbf{continuous random variable}. In the special case that it is absolutely continuous, its distribution can be described by a probability density function, which assigns probabilities to intervals; in particular, each individual point must necessarily have probability zero for an absolutely continuous random variable. Not all continuous random variables are absolutely continuous, for example a mixture distribution. Such random variables cannot be described by a probability density or a probability mass function.

\begin{definition}{Cumulative distribution function \cite{wikipedia_cdf}}{cdf}
	The \textbf{cumulative distribution function} of a real-valued random variable $X$ is the function given by
	$$ F_X(x) = P(X \leq x) , $$
	where the right-hand side represents the probability that the random variable $X$ takes on a value less than or equal to $x$.
	
	The probability that $X$ lies in the semi-closed interval $(a, b]$, where $a < b$, is therefore
	$$ P(a < X \leq b) = F_X(b) - F_X(a) . $$
\end{definition}

Every cumulative distribution function $F_X$ is non-decreasing and right-continuous, which makes it a càdlàg function. Furthermore,
$$ \lim_{x \to -\infty} F_X(x) = 0, \quad \lim_{x \to +\infty} F_X(x) = 1 . $$

Also note that
$$ P(X = x) = F_X(x) - F_X(x-) , $$
where $F_X(x-)$ is shorthand notation for $\lim_{n \to \infty} F_X(x - 1/n)$. That is, the probability of $X = x$ is the size of the jump/change of the cumulative distribution function at the point $x$.

\begin{definition}{Probability mass function \cite{wikipedia_probability_mass_function}}{probability_mass_function}
	Probability mass function is the probability distribution of a discrete random variable, and provides the possible values and their associated probabilities. It is the function $p : \mathbb{R} \to [0, 1]$ defined by
	$$ p_X(x_i) = P(X = x_i) $$
	for $-\infty < x < \infty$, where $P$ is a probability measure; $p_X(x)$ can also be simplified as $p(x)$.

	The probabilities associated with each possible values must be positive and sum up to 1. For all other values, the probabilities need to be 0:
	\begin{align*}
		\sum p_X(x_i) &= 1 ; \\
		p(x_i) &> 0 ; \\
		p(x) &= 0 \quad \text{for all other } x .
	\end{align*}
\end{definition}

Thinking of probability as mass helps to avoid mistakes, since the physical mass is conserved as is the total probability for all hypothetical outcomes $x$.

\begin{definition}{Probability density function \cite{wikipedia_probability_density_function}}{probability_density_function}
	A random variable $X$ with values in a measurable space $(\mathcal{X}, \mathcal{A})$ (usually $\mathbb{R}^n$ with the Borel sets as a measurable subsets) has as probability distribution the measure $X_* P$ on $(\mathcal{X}, \mathcal{A})$: the \textbf{density} of $X$ with respect to a reference measure $\mu$ on $(\mathcal{X}, \mathcal{A})$ is the Radon-Nikodym derivative
	$$ f = \dfrac{dX_* P}{d \mu} . $$
	That is, $f$ is any measurable function with the property that
	$$ P(X \in A) = \int_{X^{-1}A} dP = \int_A f d\mu $$
	for any measurable set $A \in \mathcal{A}$.
\end{definition}

For a function $f : \mathbb{R} \to \mathbb{R}$ to be a valid density function, the function $f$ must satisfy the following properties \cite{math2901_notes}:
\begin{enumerate}
	\item for all $x \in \mathbb{R}$, $f(x) \geq 0$;
	\item $\int_{-\infty}^{+\infty} f(x) dx = 1$.
\end{enumerate}

\begin{definition}{Expected value \cite{wikipedia_expected_value}}{expected_value}
	\begin{itemize}

		\item \textsc{Discrete case.} Let $X$ be a random variable with a finite number of finite outcomes $x_1, x_2, \ldots, x_k$ occurring with probabilities $p_1, p_2, \ldots, p_k$, repsectively. The \textbf{expectation} of $X$ is defined as
		$$ E(X) = \sum_{i=1}^k x_i p_i. $$
		Since the sum of all probabilites $p_i$ is 1, the expected value is the weighted average of the $x_i$ values, with the $p_i$ values being the weights.
	
		If all outcomes $x_{i}$ are equiprobable (that is, $p_{1} = p_{2} = \cdots = p_{k}$), then the weighted average turns into the simple average. If the outcomes $x_{i}$ are not equiprobable, then the simple average must be replaced with the weighted average, which takes into account the fact that some outcomes are more likely than the others.
		
		\item \textsc{Absolutely continuous case.} If $X$ is a random variable with a probability density function of $f$, then the expected value is deinfed as the Lebesgue integral
		$$ E(X) = \int_{\mathbb{R}} x f(x) dx , $$
		where the values on both sides are well defined or not well defined simultaneously.

	\end{itemize}
\end{definition}

In classical mechanics, the center of mass is an analogous concept to expectation. For example, suppose $X$ is a discrete random variable with values $x_i$ and corresponding probabilities $p_i$. Now consider a weightless rod on which are placed weights, at locations $x_i$ along the rod and having masses $p_i$ (whose sum is 1). The point at which the rod balances is $E(X)$ \cite{wikipedia_expected_value}.

\begin{lemma}{Expectation of transformed variable \cite{math2901_notes}}{expectation_transformed_variable}
	Suppose $g : \mathbb{R} \to \mathbb{R}$. Then the expectation of the transformed random variable $g(X)$ is
	$$
	E(g(X)) = 
		\begin{cases}
			\int_{\mathbb{R}} g(x) f_X(x) dx \quad & \text{\textsc{Continuous case.}} \\
			\sum_x g(x) P(X = x) \quad & \text{\textsc{Discrete case.}}
		\end{cases}
	$$
\end{lemma}

Usually, one is interested in computing $E(X^r)$ for $r \in \mathbb{N}$, which is called the $r$-th moment of $X$. \cite{math2901_notes}

In mathematics, a \textbf{moment} is a specific quantitative measure of the shape of a function.

The concept is used in both mechanics and statistics. If the function represents physical density, then the zeroth moment is the total mass, the first moment divided by the total mass is the center of mass, and the second moment is the rotational inertia. If the function is a probability distribution, then the zeroth moment is the total probability (i.e. one), the first moment is the expected value, the second central moment is the variance, the third standardized moment is the skewness, and the fourth standardized moment is the kurtosis. The mathematical concept is closely related to the concept of moment in physics. \cite{wikipedia_moment_mathematics}

\begin{lemma}{Linearity of expectation \cite{wikipedia_expected_value}}{lineary_expectation}
	The expected value operator (or expectation operator) $E(\cdot)$ is linear in the sense that, for any random variables $X$ and $Y$, and a constant $a$,
	\begin{align*}
		E(X + Y) &= E(X) + E(Y); \\
		E(aX) &= a E(X),
	\end{align*}
	whenever the right-hand side is well-defined.
\end{lemma}

\begin{definition}{Variance \cite{wikipedia_variance}}{variance}
	The variance of a random variable $X$ is the expected value of the squared deviation from the mean of $X$, $\mu = E(X)$:
	$$ \sigma^2 = \operatorname{Var}(X) = E((X - \mu)^2) . $$
\end{definition}

Intuitively, the variance measures on average how much the random variable deviates from its expectation/mean. \cite{math2901_notes}

\begin{lemma}{Properties of variance \cite{math2901_notes}}{properties_variance}
	Given a random variable $X$, for any constants $a, b \in \mathbb{R}$,
	\begin{itemize}
		\item $ \operatorname{Var}(X) = E(X^2) - (E(X))^2 $;
		\item $ \operatorname{Var}(aX) = a^2 \operatorname{Var}(X) $;
		\item $ \operatorname{Var}(X + b) = \operatorname{Var}(X) $;
		\item $ \operatorname{Var}(b) = 0 $.
	\end{itemize}
\end{lemma}

\begin{definition}{Standard deviation \cite{wikipedia_standard_deviation}}{standard_deviation}
	Let $X$ be a random variable with mean value $\mu$:
	$$ E(X) = \mu . $$
	Here the operator $E$ denotes the average or expected value of $X$. Then the standard deviation of $X$ is the quantity
	\begin{align*}
		\sigma &= \sqrt{\operatorname{Var}(X)} \\
		&= \sqrt{E((X - \mu)^2)} .
	\end{align*}
\end{definition}

\begin{definition}{Moment-generating function \cite{wikipedia_moment_generating_function}}{moment_generating_function}
	The moment-generating function of a random variable $X$ is
	$$ M_X(t) = E(e^{tX}) , $$
	wherever this expecation exists. In other words, the moment-generating function is the expectation of the random variable $e^{tX}$.
\end{definition}

The moment-generating function of $X$ exists if there exists $h > 0$ such that the $M_X(t)$ is finite for $x \in [-h, h]$.

\begin{lemma}{Calculations of moments \cite{wikipedia_moment_generating_function}}{calculation_moments}
	The moment-generating function is so called because if it exists on an open interval around $t = 0$, then it is the exponential generating function of the moments of the probability distribution:
	\begin{align*}
		m_{n} &= E\left(X^{n}\right) \\
		&= M_{X}^{(n)}(0) \\
		&= \left.{\frac {d^{n}M_{X}}{dt^{n}}}\right|_{t=0} .
	\end{align*}
	That is, with $n$ being a nonnegative integer, the $n$th moment about 0 is the $n$th derivative of the moment generating function, evaluated at $t = 0$.	
\end{lemma}
%\chapter{Computing}

\section{Algorithms}
\subsection{Dynamic programming}

\begin{shaded}
\textbf{Definition \cite{clrs_algorithms}.} We say that a problem exhibits \textbf{optimal substructure} if optimal solutions to related sub-problems (which may be solved independently) are incorporated into optimal solutions of the problem itself.
\end{shaded}

\begin{shaded}
\textbf{Definition (memoisation (top-down method)) \cite{clrs_algorithms}.} In this approach, we write the procedure recursively in a natural manner, but modified to save the result of each sub-problem (usually in an array or hash table). The procedure now first checks to see whether it has previously solved this sub-problem. If so, it returns the saved value, saving further computation at this level; if not, the procedure computes the value in the usual manner. We say that the recursive procedure has been \textbf{memoised}; it ``remembers'' what results it has computed previously.
\end{shaded}

\begin{shaded}
\textbf{Definition (bottom-up method) \cite{clrs_algorithms}.} This approach typically depends on some natural notion of the ``size'' of a sub-problem, such that solving any particular sub-problem depends only on solving ``smaller'' sub-problems. We sort the sub-problems by size and solve them in size order, smallest first. When solving a particular sub-problem, we have already solved all the smaller sub-problems its solution depends upon, and we have saved their solutions. We  solve each sub-problem only once, and when we first see it, we have already solved all of its prerequisite sub-problems.
\end{shaded}

\subsection{Linear programming}
\begin{shaded}
\textbf{Definition (standard form) \cite{clrs_algorithms}.} We are given a vector $\mathbf{c}$ of coefficients $c_1, c_2, \ldots, c_n$, which is associated with a vector $\mathbf{x}$ of variables $x_1, x_2, \ldots, x_n$. The \textbf{objective function} is formed by taking the dot product of the two vectors:
\begin{align*}
\text{objective function} &= \mathbf{c} \cdot \mathbf{x} \\
&= \sum_{j=1}^{n} c_j x_j \\
&= c_1 x_1 + c_2 x_2 + \ldots + c_n x_n .
\end{align*}

We want to find values for each $x_j$ that maximises the objective function, subject to the constraints
$$
\begin{pmatrix}
a_{11} & a_{12} & \ldots & a_{1n} \\
a_{21} & a_{22} & \ldots & a_{2n} \\
\vdots & \vdots & \ddots & a_{1n} \\
a_{m1} & a_{m2} & \ldots & a_{mn} \\
\end{pmatrix}
\begin{pmatrix} x_1 \\ x_2 \\ \vdots \\ x_n \end{pmatrix}
\begin{matrix} \leq \\ \leq \\ \vdots \\ \leq \end{matrix}
\begin{pmatrix} b_2 \\ b_2 \\ \vdots \\ b_m \end{pmatrix}
$$
and
$$ x_j \geq 0 , $$
for $i = 1, 2, \ldots, m$ and $j = 1, 2, \ldots, n$, where $a_{ij}, b_i \in \mathbb{R}$.

\end{shaded}

\section{Programming languages}

\subsection{General}

\subsubsection{Operator associativity}

\begin{quote}
Consider the expression \code{a ~ b ~ c}. If the operator \code{~} has left associativity, this expression would be interpreted as \code{(a ~ b) ~ c}. If the operator has right associativity, the expression would be interpreted as a \code{~ (b ~ c)}. If the operator is non-associative, the expression might be a syntax error, or it might have some special meaning. Some mathematical operators have inherent associativity. For example, subtraction and division, as used in conventional math notation, are inherently left-associative. Addition and multiplication, by contrast, are both left and right associative. (e.g. \code{(a * b) * c = a * (b * c)}). \cite{wikipedia_operator_associativity}
\end{quote}
%\chapter{Finance}

\section{Investing}
\begin{shaded}
\textbf{Definition (bonds) \cite{investopedia_bond}.} A bond is a fixed income instrument that represents a loan made by an investor to a borrower (typically corporate or governmental).
\begin{itemize}
	\item Bonds are units of corporate debt issued by companies and securitized as tradeable assets.
	\item A bond is referred to as a fixed income instrument since bonds traditionally paid a fixed interest rate (coupon) to debtholders. Variable or floating interest rates are also now quite common.
	\item Bond prices are inversely correlated with interest rates: when rates go up, bond prices fall and vice-versa.
	\item Bonds have maturity dates at which point the principal amount must be paid back in full or risk default.
\end{itemize}
\end{shaded}

\section{Personal finance}

\subsection{Miscellaneous terminology}
\begin{shaded}
\textbf{Definition (term deposit) \cite{investopedia_term_deposit}.} A term deposit is a fixed-term investment that includes the deposit of money into an account at a financial institution. Term deposit investments usually carry short-term maturities ranging from one month to a few years and will have varying levels of required minimum deposits.
\end{shaded}

\subsection{Banking}

\subsubsection{Terminology}
\begin{itemize}
	\item Retail banking --- ``Retail banking, also known as consumer banking, is the typical mass-market banking in which individual customers use local branches of larger commercial banks. Services offered include savings and checking accounts, mortgages, personal loans, debit/credit cards and certificates of deposit (CDs). In retail banking, the focus is on the individual consumer.'' --- Investopedia \cite{investopedia_retail_banking}
	\item Direct banks (such as ING) don't have branch networks and operate remotely. This means they can significantly reduce costs.
	\item Transaction (cheque) vs savings account:
	\begin{itemize}
		\item Transaction: short term, modest interest rates --- used for everyday transactions and paying bills
		\item Savings: long term, higher interest rates --- used for growing savings
	\end{itemize}
\end{itemize}

Avoid Big Four banks and their multitudinous fees.

\subsubsection{ING Orange Everyday}
Deposit \$1000+ every month and make 5+ (settled, not pending) card purchases, and you get
\begin{itemize}
	\item \$0 ING international transaction fees on online or overseas transactions,
	\item free ATMs around Australia and around the world, and
	\item (for Saving Maximiser) up to 1.95\% p.a. variable rate (limited to balances up to \$100,000).
\end{itemize}

In addition, ING Orange Everday charges no monthly fees.

\subsection{Superannuation}

\textit{Note: the following information mostly comes from the Moneysmart \cite{moneysmart_super} and ATO \cite{ato_super} websites.}

\subsubsection{Basics}
\begin{itemize}
	\item employers make compulsory payments to employees' superannuation funds, on top of wages and salary
	\item tax benefits apply
\end{itemize}

\subsubsection{Eligibility}
Must be paid over \$450 per month (before tax) to be eligible.

\subsubsection{Types of super funds}
\begin{itemize}
	\item Accumulation fund --- it...accumulates
	\item Defined benefit fund --- determined by a formula, mostly corporate or public sector funds
\end{itemize}

\subsubsection{Super fund categories}
\begin{itemize}
	\item Retail fund
	\begin{itemize}
		\item often have a wide range of options
		\item may be recommended by financial advisers who get paid a commission
		\item usually range from medium to high cost (may have low cost MySuper alternative)
		\item fund company makes profit
	\end{itemize}
	\item Industry fund
	\begin{itemize}
		\item mostly accumulation funds
		\item usually range from low to medium cost and offer MySuper option
		\item generally not-for-profit
	\end{itemize}
	\item Public sector fund --- for government employees
 	\item Corporate fund --- arranged by employer for employees
	\item SMSF (self-managed super fund)
\end{itemize}

\subsubsection{Super guarantee (SG) contributions}
Employers are required to pay at least 9.5\% of an employee's \textit{ordinary time earnings} into his super account, at least once every three months. Ordinary time earnings include:
\begin{itemize}
	\item over-award payments
	\item commissions
	\item allowances
	\item bonuses
	\item paid leave
\end{itemize}

\subsubsection{Salary sacrifices}
Salary sacrifices are considered employer contributions rather than employee contributions, and are taxed at a maximum rate of 15\% (generally less than marginal tax rate). There is no limit to how much can be contributed, however if contributions exceed a certain threshold, the concessional tax rate will not apply. (This threshold has been \$25,000 since 2017-2018.)

\subsubsection{Personal contributions}
Personal contributions come from after-tax income. There is a non-concessional contributions cap of \$100,000.

\subsubsection{Withdrawal}
Super can be withdrawn:
\begin{itemize}
	\item at age 65
	\item when preservation age is reached (60 for me)
	\item under transition to retirement rules
	\item if experiencing extraordinarily severe conditions financially, medically etc.
\end{itemize}
Some withdrawn money is taxable and some isn't:
\begin{itemize}
	\item Non-concessional (after-tax) contributions --- not taxable upon withdrawal
	\item Concessional (before-tax) contributions --- taxable. (Contributions include employer contributions, salary sacrificed, and tax-deducted personal contributions.) The amount of tax payable depends on and whether tax was paid for it before contribution --- taxable super is separated into taxed and untaxed elements.
\end{itemize}
Super can be withdrawn a number of ways, including as an income stream or as a lump sum. After an income stream starts, no more contributions can be made.

\subsubsection{Miscellaneous}
Having more than one super fund means there are avoidable fees. Better to have all superannuation money wrapped up in a single fund.

\subsubsection{Student Super}
Zero switching fees for \textbf{any} balance.
\begin{itemize}
	\item Balance under \$1,000 --- zero fees
	\item Balance between \$1,000 and \$4,999 --- flat \$39 fee per year and administration fee of 0.99\% p.a.
\end{itemize}

\section{Insurance}
\begin{itemize}
	\item Health (for income over \$90,000)
	\item Income insurance
\end{itemize}

\section{Transportation}

\textbf{Opal card (concession)} --- must carry proof of entitlement (student card). Benefits:

\begin{itemize}
	\item Daily travel cap of \$8
	\item Weekly travel cap of \$25
	\item Sunday travel cap of \$2.8
	\item Weekly travel award --- After 8 journeys, all remaining fares for the week are half price; note that a tap-on must be 60 minutes after the last tap-off to be considered a new journey (Manly ferries exception of 2hrs 10 min).
	\item A 30\% discount on metro/train fares outside of peak hours.
\end{itemize}

\printbibliography

\end{document}
