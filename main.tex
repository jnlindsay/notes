\documentclass[oneside]{book}

% PACKAGES
\usepackage[utf8]{inputenc}
\usepackage{biblatex}				% bibliography
%\usepackage{xeCJK}					% Chinese
\usepackage{csquotes}				% quotes
\usepackage{parskip}				% paragraph indentation
\usepackage{fancyhdr}				% fancy headers
\usepackage[yyyymmdd]{datetime}		% proper date format
\usepackage{tgschola}				% Tex Gyre Schola (based on New Century Schoolbook)
\usepackage{amsmath}
\usepackage{amsfonts}
\usepackage{xcolor}
\usepackage{framed}					% paragraph shading
\usepackage[hidelinks]{hyperref}	% clickable table of contents
\usepackage[italic]{esdiff}			% differentiation notation

% SETTINGS
\colorlet{shadecolor}{lightgray!20}
\setlength{\parindent}{0em}
\addbibresource{bibliography.bib}

% REDEFINITIONS

% Credit to @StefanKottwitz on Tex Stack Exchange
% This redefinition allows matrices to be stretched on the fly like so:
% \begin{pmatrix}[1.5] ... \end{pmatrix}
\makeatletter
\renewcommand*\env@matrix[1][\arraystretch]{
  \edef\arraystretch{#1}%
  \hskip -\arraycolsep
  \let\@ifnextchar\new@ifnextchar
  \array{*\c@MaxMatrixCols c}}
\makeatother

\begin{document}

\tableofcontents

\chapter{Mathematics}
\section{Calculus}

\subsection{Continuity and differentiability}

\begin{shaded}
\textbf{Definition (Continuous function) \cite{hubbard_hubbard}.} Let $X \subset \mathbb{R}^n$. Then a mapping $\mathbf{f} : X \to \mathbb{R}^m$ is continuous at $\mathbf{x}_0 \in X$ if
$$ \lim_{\mathbf{x} \to \mathbf{x}_0} \mathbf{f}(\mathbf{x}) = \mathbf{f}(\mathbf{x}_0); $$
$\mathbf{f}$ is continuous on $X$ if it is continuous at every point of $X$.

Equivalently, $\mathbf{f}: X \to \mathbf{R}^m$ is continuous at $\mathbf{x}_0$ if and only if for every $\epsilon > 0$, there exists $\delta > 0$ such that when $|\mathbf{x} - \mathbf{x}_0| < \delta$, then $|\mathbf{f}(\mathbf{x}) - \mathbf{f}(\mathbf{x}_0)| < \epsilon$.
\end{shaded}

\begin{shaded}
\textbf{Definition (Continuity and discontinuity) \cite{math2111_notes}.} Suppose $[a, b] \subset \mathbb{R}$. Consider a function $f : [a, b] \to \mathbb{R}$ and a point $c \in \mathbb{R}$. Supposed the one-sided limits
$$ f(c^-) = \lim_{x \to c^-} f(x) \qquad \textnormal{and} \qquad f(c^+) = \lim_{x \to c^+} f(x) $$
exist.

(1) If
$$ f(c^-) = f(c^+) = f(c) , $$
then $f$ is \textbf{continuous} at $c$.

(2) If
$$ f(c^-) = f(c^+) \not= f(c) $$
--- regardless of whether or not $f(c)$ exists --- then $f$ has a \textbf{removable discontinuity} at $c$.

(3) If
$$ f(c^-) \not= f(c^+) , $$
then $f$ has a \textbf{jump discontinuity} at $c$.
\end{shaded}

\begin{shaded}
\textbf{Definition (Piecewise continuous function) \cite{math2111_notes}.} A piecewise function is continuous on $[a, b] \subseteq \mathbb{R}$ if

(1) $f(x^-)$ exists for each $x \in (a, b]$;

(2) $f(x^+)$ exists for each $x \in [a, b)$;

(3) $f$ is continuous on $(a, b)$ except at (at most) a finite number of points where there exist jump discontinuities.

Moreover, $f$ is piecewise continuous on $\mathbb{R}$ if it is piecewise continuous on any finite interval $[a, b] \subseteq \mathbb{R}$.
\end{shaded}

\begin{shaded}
\textbf{Definition (Differentiability) \cite{math2111_notes}.} Consider a function $ f : \mathbb{R} \to \mathbb{R} $ and a point $c \in \mathbb{R}$. Let
$$ f(c^-) = \lim_{x \to c^-} f(x) \qquad \textnormal{and} \qquad f(c^+) = \lim_{x \to c^+} f(x) , $$
and let
\begin{align*}
	D^+ f(c) &= \lim_{h \to 0^+} \frac{f(c + h) - f(c^+)}{h} \qquad \text{and} \\
	D^- f(c) &= \lim_{h \to 0^-} \frac{f(c + h) - f(c^-)}{h} .
\end{align*}
%Then $f$ is differentiable at $c$ if and only if $f(c^+) = f(c^-)$ and $D^+ f(c) = D^- f(c)$.

Warning: $D^{+/-} f(c)$ is not necessarily the same as $f'(c^{+/-}) = \lim_{x \to c^{+/-}} f'(x)$.
\end{shaded}

\begin{shaded}
\textbf{Definition (Piecewise differentiable function) \cite{math2111_notes}.} A function $f$ is piecewise differentiable on $[a, b] \subseteq \mathbb{R}$ if

(1) $D^- f(x)$ exists for each $x \in (a, b]$;

(2) $D^+ f(x)$ exists for each $x \in [a, b)$;

(3) $f$ is differentiable on $(a, b)$ except at (at most) a finite number of points.

Moreover, $f$ is piecewise differentiable on $\mathbb{R}$ if it is piecewise differentiable on any finite interval $[a, b] \subseteq \mathbb{R}$.
\end{shaded}

\subsection{Fourier series and convergence}

\begin{shaded}
\textbf{Fourier series \cite{math2111_notes}.} Consider a function $f : \mathbb{R} \to \mathbb{R}$ which is $2 L$-periodic and is square integrable (i.e., $\int_{-\pi}^\pi f(x)^2 \ dx < \infty $). Its Fourier series is given by
$$ S_f(x) = \frac{a_0}{2} + \sum_{k = 1}^{\infty} \left( a_k \cos \left( \frac{k \pi}{L}x \right) + b_k \sin \left( \frac{k \pi}{L}x \right) \right) . $$
We can view the Fourier series as expression a function $f$ as a linear combination of the orthogonal functions $\frac{1}{2}, \cos(kx), ..., \sin(kx), k \geq 1$. Then, via vector decomposition, we can find the following formulas to yield the Fourier coefficients $a_k$ and $b_k$:
$$ a _k = \frac{1}{L} \int_{-L}^L f(x) \cos \left( \frac{k \pi}{L}x \right) \ dx, \quad k = 0, 1, 2, ... $$
and
$$ b _k = \frac{1}{L} \int_{-L}^L f(x) \sin \left( \frac{k \pi}{L}x \right) \ dx, \quad k = 1, 2, 3, ... $$
\end{shaded}

\begin{shaded}
\textbf{Definition (pointwise convergence) \cite{math2111_notes}.} Let $f_k : \mathbb{R} \to \mathbb{R}$. We say $f_k$ converges to $f$ on $[a, b]$ pointwisely if $f_k(x) \to f(x)$ for every $x \in [a, b]$ as $k \to \infty$.
\end{shaded}

\begin{shaded}
\textbf{Theorem (pointwise convergence of Fourier series) \cite{math2111_notes}.} Let $c \in \mathbb{R}$, and suppose that a function $f : \mathbb{R} \to \mathbb{R}$ has the following properties:
\begin{enumerate}
	\item $f$ is $2 \pi$-periodic;
	\item $f$ is piecewise continuous on $[-\pi, \pi]$;
	\item $D^+ f(c)$ and $D^- f(c)$ exist.
\end{enumerate}
If $f$ is continuous at $c$, then the Fourier series of $f$ agrees with $f$ at $c$, i.e.,
$$ S f(c) = f(c) . $$
On the other hand, if $f$ has a jump discontinuity at $c$, then
$$ S f(c) = \frac{1}{2} \left( f(c^+) + f(c^-) \right) . $$
\end{shaded}

\begin{shaded}
\textbf{Definition (uniform convergence) \cite{math2111_notes}.} Let $f_k : \mathbb{R} \to \mathbb{R}$. We say $f_k$ converges to $f$ on $[a, b]$ uniformly if for every $\epsilon > 0$, there exists a $K$ (depends on $\epsilon$ only), such that
$$ \sup_{x \in [a, b]} | f_k(x) - f(x) | \leq \epsilon \quad \text{for all} \quad k \geq K . $$
Moreover, we say $\sum_{k = 1}^{\infty} f_k$ converges uniformly to $f$ if the partial sum $\tilde{f}_n = \sum_{k = 1}^{\infty} f_k$ converges uniformly to $f$ as $n \to \infty$

\textbf{Theorem.} If $f_k : \mathbb{R} \to \mathbb{R}$ is continuous on $[a, b]$ for all $n$ and if $f_k$ converge to $f$ uniformly on $[a, b]$, then $f$ is continuous on $[a, b]$.
\end{shaded}

\begin{shaded}
\textbf{Theorem (Weierstrass test) \cite{math2111_notes}.} Let $f_k : \mathbb{R} \to \mathbb{R}$ be a sequence of functions defined on $[a, b]$. Suppose that there exists a sequence of numbers $c_k$ such that
$$ |f_k(x)| \leq c_k \quad \text{for all} \quad x \in [a, b] $$
and $\sum_{k = 1}^\infty c_k$ exists. Then, $\sum_{k = 1}^\infty f_k(x)$ converges uniformly to a function $f$ on $[a, b]$.
\end{shaded}

\begin{shaded}
\textbf{Definition (norm convergence) \cite{math2111_notes}.} Consider the maximum norm $\lVert f \rVert_\infty = \sup_{x \in [a, b]} |f(x)|$. The definition of uniform convergence can be equivalently rewritten as: for every $\epsilon > 0$, there exists $K > 0$ such that
$$ \lVert f_k - f \rVert \leq \epsilon \quad \text{for all} \quad k \geq K , $$
or simply
$$ \lim_{k \to \infty} \lVert f_k - f \rVert = 0 . $$
\end{shaded}

\begin{shaded}
\textbf{Theorem (Parseval's theorem).} Let $f$ be $2 \pi$-periodic, bounded and $\int_{-\pi}^\pi f(x)^2 \ dx < + \infty$. Then, the Fourier series of $f$ converges to $f$ in the mean square sense, i.e.
$$ \int_{- \pi}^\pi (S_n f(x) - f(x))^2 \ dx \to 0 \quad \text{as} \quad k \to \infty . $$
That is to say, $\lVert S_n f - f \rVert_2 \to 0$ as $n \to \infty$. Moreover, the following \textbf{Parseval's identity} holds:
$$ \int_{-\pi}^\pi f(x)^2 \ dx = \lVert f \rVert_2^2 = \frac{\pi}{2} a_0^2 + \pi \sum_{k = 1}^{\infty} (a_k^2 + b_k^2) . $$
\end{shaded}

\subsection{Div, grad, curl}

\begin{shaded}
\textbf{Definition (divergence) \cite{math2111_notes}.} If $\mathbf{F} = F_1 \mathbf{i} + F_2 \mathbf{j} + F_3 \mathbf{k}$, the \textbf{divergence} of $\mathbf{F}$ is the scalar field
\begin{align*}
\text{div} \ \mathbf{F} &= \nabla \cdot \mathbf{F} \\
&= \begin{pmatrix}[2] \diffp{}{x} & \diffp{}{y} & \diffp{}{z} \end{pmatrix}^{T}
\cdot \begin{pmatrix} F_1 \\ F_2 \\ F_3 \end{pmatrix} \\
&= \diffp{F_1}{x} + \diffp{F_2}{y} + \diffp{F_3}{z} .
\end{align*}
\end{shaded}

\begin{shaded}
\textbf{Definition (curl) \cite{math2111_notes}.} If $\mathbf{F} = F_1 \mathbf{i} + F_2 \mathbf{j} + F_3 \mathbf{k}$, the \textbf{curl} of $\mathbf{F}$ is the vector field
\begin{align*}
\text{curl} \ \mathbf{F} &= \nabla \times \mathbf{F} \\
&= \begin{vmatrix}[1.5]
	\mathbf{i} & \mathbf{j} & \mathbf{k} \\
	\diffp{}{x} & \diffp{}{y} & \diffp{}{z} \\
	F_1 & F_2 & F_3
\end{vmatrix} \\
&= \left( \diffp{F_3}{y} - \diffp{F_2}{z} \right) \mathbf{i} + \left( \diffp{F_1}{z} - \diffp{F_3}{x} \right) \mathbf{j} + \left( \diffp{F_2}{x} - \diffp{F_1}{y} \right) \mathbf{k} .
\end{align*}
\end{shaded}

\subsection{Line integrals}

\begin{shaded}
\textbf{Definition (line integral of a scalar field) \cite{marsden_vector_calculus}.} The \textbf{line integral of a scalar field}, or \textbf{path integral}, or the \textbf{integral of $f(x, y, z)$ along the path $\mathbf{c}$}, is defined when $\mathbf{c}: I = [a, b] \to \mathbb{R}^3$ is of class $C^1$ and when the composite function $t \mapsto f(x(t), y(t), z(t))$ is continuous on $I$. We define this integral by the equation
$$ \int_{\mathbf{c}} f \ ds = \int_a^b f(x(t), y(t), z(t)) \ \lVert \mathbf{c}'(t) \rVert \ dt . $$
Sometimes $\int_\mathbf{c} f \ ds$ is denoted
$$ \int_\mathbf{c} f(x, y, z) \ ds $$
or
$$ \int_\mathbf{c} f(\mathbf{c}(t)) \ \lVert \mathbf{c}'(t) \rVert \ dt . $$

If $\mathbf{c}(t)$ is only piecewise $C^1$ or $f(\mathbf{c}(t))$ is piecewise continuous, we define $\int_\mathbf{c} f \ ds$ by breaking $[a, b]$ into pieces over which $f(\mathbf{c}(t)) \ \lVert \mathbf{c}'(t) \rVert$ is continuous, and summing the integrals over the pieces.
\end{shaded}

\begin{shaded}
\textbf{Definition (line integral of a vector field) \cite{marsden_vector_calculus}.} Let $\mathbf{F}$ be a vector field on $\mathbb{R}^3$ that is continuous on the $C^1$ path $\mathbf{c} : [a, b] \to \mathbb{R}^3$. We define $\int_\mathbf{c} \mathbf{F} \cdot d\mathbf{s}$, the \textbf{line integral} of $\mathbf{F}$ along $\mathbf{c}$, by the formula
$$ \int_\mathbf{c} \mathbf{F} \cdot d\mathbf{s} = \int_a^b \mathbf{F}(\mathbf{c}(t)) \cdot \mathbf{c}'(t) \ dt ; $$
that is, we integrate the dot product of $\mathbf{F}$ with $\mathbf{c}'$ over the interval $[a, b]$.

As is the case with scalar functions, we can also define $\int_\mathbf{c} \mathbf{F} \cdot d\mathbf{s}$ if $\mathbf{F}(\mathbf{c}(t)) \cdot \mathbf{c}'(t)$ is only piecewise continuous.
\end{shaded}

\textsc{Note \cite{marsden_vector_calculus}}. Another common way of writing line integrals is
\begin{align*}
\int_{\mathbf{c}} F_1 \ dx + F_2 \ dy + F_3 \ dz &= \int_a^b \left( F_1 \diff{x}{t} + F_2 \diff{y}{t} + F_3 \diff{z}{t} \right) \ dt \\
&= \int_{\mathbf{c}} \mathbf{F} \cdot d \mathbf{s} .
\end{align*}
Note that we may think of $d \mathbf{s}$ as the differential form $d \mathbf{s} = dx \mathbf{i} + dy \mathbf{j} + dz \mathbf{k}$. Thus, the differential form $F_1 \ dx + F_2 \ dy + F_3 \ dz$ may be written as the dot product $\mathbf{F} \cdot d \mathbf{s}$.

\subsection{Definitions and theorems about line integrals}

\begin{shaded}
\textbf{Definition (arc length in $\mathbb{R}^n$) \cite{marsden_vector_calculus}.} Let $\mathbf{c}: [t_0, t_1] \to \mathbb{R}^n$ be a piecewise $C^1$ path. Its \textbf{length} is defined to be
$$ L(\mathbf{c} = \int_{t_0}^{t_1} \lVert \mathbf{c}'(t) \rVert \ dt . $$
The integrand is the square root of the sume of the squares of the coordinate functions of $\mathbf{c}'(t)$: If
$$ \mathbf{c}(t) = (x_1(t), x_2(t), ..., x_n(t)) , $$
then
$$ L(\mathbf{c}) = \sqrt{[x'_1(t)]^2 + [x'_2(t)]^2 + ... + [x'_n(t)]^2} \ dt . $$
\end{shaded}

\begin{shaded}
\textbf{Definition (arc length differential) \cite{marsden_vector_calculus}.} An \textbf{infinitesimal displacement} of a particle following a path $\mathbf{c}(t) = x(t) \mathbf{i} + y(t) \mathbf{j} + z(t) \mathbf{k}$ is
$$ d \mathbf{s} = dx \mathbf{i} + dy \mathbf{j} + dz \mathbf{k} = \left( \diff{x}{t} \mathbf{i} + \diff{y}{t} \mathbf{j} + \diff{z}{t} \mathbf{k} \right) \ dt , $$
and its length
$$ ds = \sqrt{dx^2 + dy^2 + dz^2} = \sqrt{{\diff{x}{t}}^2 + {\diff{y}{t}}^2 + {\diff{z}{t}}^2} \ dt $$
is the \textbf{differential of arc length}.
\end{shaded}

\begin{shaded}
\textbf{Theorem (line integrals of gradient vector fields) \cite{marsden_vector_calculus}.} Suppose $f : \mathbb{R}^3 \to \mathbb{R}$ is of class $C^1$ and that $\mathbf{c} : [a, b] \to \mathbb{R}^3$ is a piecewise $C^1$ path. Then
$$ \int_{\mathbf{c}} \nabla f \cdot d\mathbf{s} = f(\mathbf{c}(b)) - f(\mathbf{c}(a)) . $$
\end{shaded}

\begin{shaded}
\textbf{Theorem (cross partials of gradient vector fields) \cite{math2111_notes}.} Let
$$ \mathbf{F} = (F_1, F_2, F_3) $$
be a gradient vector field whose components ahve continuous partial derivatives. Then the cross partials are equal:
$$ \diffp{F_1}{y} = \diffp{F_2}{x}, \quad \diffp{F_2}{z} = \diffp{F_3}{y}, \quad \diffp{F_3}{x} = \diffp{F_1}{z} . $$

Similarly, if the vector field in the plane
$$ \mathbf{F} = (F_1, F_2) $$
is a gradient vector field, then
$$ \diffp{F_1}{y} = \diffp{F_2}{x} . $$
\end{shaded}

\begin{shaded}
\textbf{Green's theorem \cite{marsden_vector_calculus}.} Let $D$ be a simple region and let $C$ be its boundary. Suppose $P: D \to \mathbb{R}$ and $Q: D \to \mathbb{R}$ are of class $C^1$. Then
$$ \int_{C^+} P \ dx + Q \ dy = \iint_D \left( \diffp{Q}{x} - \diffp{P}{y} \right) \ dx dy . $$

\textsc{Vector form.} Let $D \subset \mathbb{R}^2$ be a region to which Green's theorem applies, let $\partial D$ be its (positively oriented) boundary, and let $\mathbf{F} = P \mathbf{i} + Q \mathbf{j}$ be a $C^1$ vector field on $D$. Then
\begin{align*}
\int_{\partial D} \mathbf{F} \cdot d\mathbf{s} &= \iint_D (\text{curl} \ \mathbf{F}) \cdot \mathbf{k} \ dA \\
&= \iint_D (\nabla \times \mathbf{F}) \cdot \mathbf{k} \ dA
\end{align*}
\end{shaded}

\subsection{Miscellaneous}

\begin{shaded}
\textbf{Even and odd functions.}
\begin{itemize}
	\item The product of two even functions is an even function.
	\item The product of two odd functions is an even function.
	\item The product of and even function and an odd function is an odd function.
\end{itemize}

This can be helpful when solving integrals of the form $\int f(x)g(x) \ dx$, where $f$ and $g$ are each even or odd.
\end{shaded}
\chapter{Computing}

\section{Graphics}

\subsection{Basics}

\begin{itemize}
	\item Display rate --- number of distinct frames shown per second
	\item Refresh rate --- number of times frames are updated (might be a lot higher than display rate)
\end{itemize}

\section{Algorithms}
\subsection{Dynamic programming}

\begin{shaded}
\textbf{Definition \cite{clrs_algorithms}.} We say that a problem exhibits \textbf{optimal substructure} if optimal solutions to related sub-problems (which may be solved independently) are incorporated into optimal solutions of the problem itself.
\end{shaded}

\begin{shaded}
\textbf{Definition (memoisation (top-down method)) \cite{clrs_algorithms}.} In this approach, we write the procedure recursively in a natural manner, but modified to save the result of each sub-problem (usually in an array or hash table). The procedure now first checks to see whether it has previously solved this sub-problem. If so, it returns the saved value, saving further computation at this level; if not, the procedure computes the value in the usual manner. We say that the recursive procedure has been \textbf{memoised}; it ``remembers'' what results it has computed previously.
\end{shaded}

\begin{shaded}
\textbf{Definition (bottom-up method) \cite{clrs_algorithms}.} This approach typically depends on some natural notion of the ``size'' of a sub-problem, such that solving any particular sub-problem depends only on solving ``smaller'' sub-problems. We sort the sub-problems by size and solve them in size order, smallest first. When solving a particular sub-problem, we have already solved all the smaller sub-problems its solution depends upon, and we have saved their solutions. We  solve each sub-problem only once, and when we first see it, we have already solved all of its prerequisite sub-problems.
\end{shaded}

\section{Cloud computing}

\subsection{Basics}

\subsubsection{Service models}

From Wikipedia \cite{wikipedia_cloud_computing}:
\begin{itemize}
	\item \textbf{SaaS} (application level) --- CRM, email, virtual desktop, communication, games, ...
	\item \textbf{PaaS} (platform level) --- execution runtime, database, web server, development tools, ...
	\item \textbf{IaaS} (infrastructure level) --- virtual machines, servers, storage, load balancers, network, ...
\end{itemize}
\chapter{Finance}

\section{Investing}
\begin{shaded}
\textbf{Definition (bonds) \cite{investopedia_bond}.} A bond is a fixed income instrument that represents a loan made by an investor to a borrower (typically corporate or governmental).
\begin{itemize}
	\item Bonds are units of corporate debt issued by companies and securitized as tradeable assets.
	\item A bond is referred to as a fixed income instrument since bonds traditionally paid a fixed interest rate (coupon) to debtholders. Variable or floating interest rates are also now quite common.
	\item Bond prices are inversely correlated with interest rates: when rates go up, bond prices fall and vice-versa.
	\item Bonds have maturity dates at which point the principal amount must be paid back in full or risk default.
\end{itemize}
\end{shaded}

\section{Personal finance}

\subsection{Miscellaneous terminology}
\begin{shaded}
\textbf{Definition (term deposit) \cite{investopedia_term_deposit}.} A term deposit is a fixed-term investment that includes the deposit of money into an account at a financial institution. Term deposit investments usually carry short-term maturities ranging from one month to a few years and will have varying levels of required minimum deposits.
\end{shaded}

\subsection{Banking}

\subsubsection{Terminology}
\begin{itemize}
	\item Retail banking --- ``Retail banking, also known as consumer banking, is the typical mass-market banking in which individual customers use local branches of larger commercial banks. Services offered include savings and checking accounts, mortgages, personal loans, debit/credit cards and certificates of deposit (CDs). In retail banking, the focus is on the individual consumer.'' --- Investopedia \cite{investopedia_retail_banking}
	\item Direct banks (such as ING) don't have branch networks and operate remotely. This means they can significantly reduce costs.
	\item Transaction (cheque) vs savings account:
	\begin{itemize}
		\item Transaction: short term, modest interest rates --- used for everyday transactions and paying bills
		\item Savings: long term, higher interest rates --- used for growing savings
	\end{itemize}
\end{itemize}

Avoid Big Four banks and their multitudinous fees.

\subsubsection{ING Orange Everyday}
Deposit \$1000+ every month and make 5+ (settled, not pending) card purchases, and you get
\begin{itemize}
	\item \$0 ING international transaction fees on online or overseas transactions,
	\item free ATMs around Australia and around the world, and
	\item (for Saving Maximiser) up to 1.95\% p.a. variable rate (limited to balances up to \$100,000).
\end{itemize}

In addition, ING Orange Everday charges no monthly fees.

\subsection{Superannuation}

\textit{Note: the following information mostly comes from the Moneysmart \cite{moneysmart_super} and ATO \cite{ato_super} websites.}

\subsubsection{Basics}
\begin{itemize}
	\item employers make compulsory payments to employees' superannuation funds, on top of wages and salary
	\item tax benefits apply
\end{itemize}

\subsubsection{Eligibility}
Must be paid over \$450 per month (before tax) to be eligible.

\subsubsection{Types of super funds}
\begin{itemize}
	\item Accumulation fund --- it...accumulates
	\item Defined benefit fund --- determined by a formula, mostly corporate or public sector funds
\end{itemize}

\subsubsection{Super fund categories}
\begin{itemize}
	\item Retail fund
	\begin{itemize}
		\item often have a wide range of options
		\item may be recommended by financial advisers who get paid a commission
		\item usually range from medium to high cost (may have low cost MySuper alternative)
		\item fund company makes profit
	\end{itemize}
	\item Industry fund
	\begin{itemize}
		\item mostly accumulation funds
		\item usually range from low to medium cost and offer MySuper option
		\item generally not-for-profit
	\end{itemize}
	\item Public sector fund --- for government employees
 	\item Corporate fund --- arranged by employer for employees
	\item SMSF (self-managed super fund)
\end{itemize}

\subsubsection{Super guarantee (SG) contributions}
Employers are required to pay at least 9.5\% of an employee's \textit{ordinary time earnings} into his super account, at least once every three months. Ordinary time earnings include:
\begin{itemize}
	\item over-award payments
	\item commissions
	\item allowances
	\item bonuses
	\item paid leave
\end{itemize}

\subsubsection{Salary sacrifices}
Salary sacrifices are considered employer contributions rather than employee contributions, and are taxed at a maximum rate of 15\% (generally less than marginal tax rate). There is no limit to how much can be contributed, however if contributions exceed a certain threshold, the concessional tax rate will not apply. (This threshold has been \$25,000 since 2017-2018.)

\subsubsection{Personal contributions}
Personal contributions come from after-tax income. There is a non-concessional contributions cap of \$100,000.

\subsubsection{Withdrawal}
Super can be withdrawn:
\begin{itemize}
	\item at age 65
	\item when preservation age is reached (60 for me)
	\item under transition to retirement rules
	\item if experiencing extraordinarily severe conditions financially, medically etc.
\end{itemize}
Some withdrawn money is taxable and some isn't:
\begin{itemize}
	\item Non-concessional (after-tax) contributions --- not taxable upon withdrawal
	\item Concessional (before-tax) contributions --- taxable. (Contributions include employer contributions, salary sacrificed, and tax-deducted personal contributions.) The amount of tax payable depends on and whether tax was paid for it before contribution --- taxable super is separated into taxed and untaxed elements.
\end{itemize}
Super can be withdrawn a number of ways, including as an income stream or as a lump sum. After an income stream starts, no more contributions can be made.

\subsubsection{Miscellaneous}
Having more than one super fund means there are avoidable fees. Better to have all superannuation money wrapped up in a single fund.

\subsubsection{Student Super}
Zero switching fees for \textbf{any} balance.
\begin{itemize}
	\item Balance under \$1,000 --- zero fees
	\item Balance between \$1,000 and \$4,999 --- flat \$39 fee per year and administration fee of 0.99\% p.a.
\end{itemize}

\section{Insurance}
\begin{itemize}
	\item Health (for income over \$90,000)
	\item Income insurance
\end{itemize}

\section{Transportation}

\textbf{Opal card (concession)} --- must carry proof of entitlement (student card). Benefits:

\begin{itemize}
	\item Daily travel cap of \$8
	\item Weekly travel cap of \$25
	\item Sunday travel cap of \$2.8
	\item Weekly travel award --- After 8 journeys, all remaining fares for the week are half price; note that a tap-on must be 60 minutes after the last tap-off to be considered a new journey (Manly ferries exception of 2hrs 10 min).
	\item A 30\% discount on metro/train fares outside of peak hours.
\end{itemize}

\printbibliography

\end{document}