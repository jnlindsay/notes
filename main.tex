\documentclass[oneside]{book}

% PACKAGES
\usepackage[utf8]{inputenc}
\usepackage{biblatex}				% bibliography
%\usepackage{xeCJK}					% Chinese
\usepackage{csquotes}				% quotes
\usepackage{parskip}                 % paragraph indentation
\usepackage{fancyhdr}				% fancy headers
\usepackage[yyyymmdd]{datetime}		% proper date format
\usepackage{tgschola}				% Tex Gyre Schola (based on New Century Schoolbook)
\usepackage{amsmath}
\usepackage{amsfonts}
\usepackage{amssymb}					% symbols
\usepackage{amsthm}
\usepackage{tcolorbox}				% boxes for theorems, definitions, etc.
\usepackage{xcolor}
\usepackage{framed}    				% paragraph shading
\usepackage[hidelinks]{hyperref}	% clickable table of contents
\usepackage[italic]{esdiff}			% differentiation notation

% SETTINGS
\colorlet{shadecolor}{lightgray!20}
\setlength{\parindent}{0em}
\addbibresource{bibliography.bib}
\tcbuselibrary{most}

% 'THEOREM' ENVIRONMENTS
\newtcbtheorem[number within=section]{definition}{Definition}%
    {colback=blue!5, colframe=blue!35!black, sharp corners,
    	 fonttitle=\bfseries, parbox=false,
    	 separator sign={\ ---}}{def}

\newtcbtheorem[number within=section]{theorem}{Theorem}%
    {colback=red!5, colframe=red!35!black, sharp corners,
     fonttitle=\bfseries, parbox=false,
     separator sign={\ ---}}{th}
     
\newtcbtheorem[number within=section,
               use counter from=theorem]{lemma}{Lemma}%
    {colback=red!5, colframe=red!35!black, sharp corners,
     fonttitle=\bfseries, parbox=false,
     separator sign={\ ---}}{lem}
     
\tcolorboxenvironment{proof}{% 'proof' from 'amsthm'
	blanker,breakable,left=3mm,
	before skip=10pt,after skip=10pt, parbox=false,
	borderline west={0.5mm}{0pt}{black}}

% REDEFINITIONS

% Credit to @StefanKottwitz on Tex Stack Exchange
% This redefinition allows matrices to be stretched on the fly like so:
% \begin{pmatrix}[1.5] ... \end{pmatrix}
\makeatletter
\renewcommand*\env@matrix[1][\arraystretch]{
  \edef\arraystretch{#1}%
  \hskip -\arraycolsep
  \let\@ifnextchar\new@ifnextchar
  \array{*\c@MaxMatrixCols c}}
\makeatother

% RELATIVE PATHS
\newcommand*{\MathematicsPath}{include/mathematics}

% #################################################################

\begin{document}

\tableofcontents

% \chapter{Mathematics}

\newcommand*{\MathematicsPath}{include/mathematics}

\chapter{Analysis}

\section{Introduction}

\begin{problem}{p. 2}{}

    \marginnote{Newton's method.}

    We want to approximate the solution of the equation $x^2 - 2 = 0$. Newton's method uses the following calculation:
        $$ x_{n+1} = x_n - \frac{f(x_n)}{f'(x_n)} . $$
    So we have
        \begin{align*}
            f(x) &= x^2 - 2 \\
            f'(x) &= 2x \\
            x_0 &= 1 \qquad \text{(initial guess)} \\
            x_1 &= 3/2 \\
            x_2 &= \ldots
        \end{align*}
\end{problem}

\begin{itemize}
    \item $L^p$ metric on $\mathbb{R}^n$:
        $$ d_p(\mathbf{x}, \mathbf{y})
            = \biggl( \sum_{i = 1}^n 
            \lvert y_i - x_i \rvert^p \biggr)^\frac{1}{p} $$
    \item $L^\infty$ metric on $\mathbb{R}^n$:
        $$ d_\infty(\mathbf{x}, \mathbf{y}) = \max\{ \lvert x_1 - y_1 \rvert,
            \lvert x_2 - y_2 \rvert, \ldots, 
            \lvert x_n - y_n \rvert \} . $$
\end{itemize}

\begin{problem}{p. 6}{}

    \marginnote{$L^p$ metrics on $\mathbb{R}^n$.}

    What does the set
        $$ \{ (x, y) : d_p((x, y), (0, 0)) \leq 1 \} $$
    look like for $p = 1, 2, 3, \infty$?

    \tcblower

    \begin{align*}
        p < 1 \qquad &\text{`pushed in' diamond} \\
        p = 1 \qquad &\text{diamond} \\
        p = 2 \qquad &\text{circle} \\
        p = 3 \qquad &\text{`pushed out' circle} \\
        \vdots \\
        p = \infty \qquad &\text{square}
    \end{align*}

\end{problem}

\begin{itemize}
    \item $L^p$ metric on $C[0, 1]$:
        $$ d_p(f, g) = \biggl( \int_0^1 \lvert f(x) - g(x) \rvert^p dx \biggr)^\frac{1}{p} $$
    \item $L^\infty$ metric on $C[0, 1]$:
        $$ d_\infty(f, g) = \sup_{x \in [0, 1]} \lvert f(x) - g(x) \rvert $$
\end{itemize}

\underline{See lecture notes.} \marginnote{$L^p$ metrics on $C[0, 1]$.}

\section{Sets and cardinality}

\begin{itemize}
    \item \textbf{Cartesian product}. Given $\{A_i\}_{i \in I}$,
        $$ \prod_{i \in I} A_i = \{ (a_i)_{i \in I}
            : a_i \in A_i, \forall i \in I \} $$
        \underline{Note} that $\{A_i\}_{i \in I}$ is a \textit{set of sets}, whereas $(a_i)_{i \in I}$ is a \textit{sequence}.
    \item Formally, the \textbf{Cartesian product} is
        $$ \{ f : I \to \cup_{i \in I} A_i
            : f(i) \in A_i, \forall i \in I \} . $$
        (Try $\mathbb{R} \times \mathbb{N}$.)
\end{itemize}

\begin{problem}{N/A}{}

    \marginnote{Formal definition of Cartesian product (easy).}

    Describe the formal definition of $\mathbb{R} \times \mathbb{N}$.

    \tcblower

    The set of all $f : \{1, 2\} \to \mathbb{R}$ such that
        $$ f(1) \in \mathbb{R} \quad 
            \text{and} \quad f(2) \in \mathbb{N} . $$
\end{problem}

\begin{problem}{N/A}{}

    \marginnote{Formal definition of Cartesian product (hard).}

    Describe the Cartesian product $\prod_{i \in I} A_i$ where $I \in [0, 1]$ and
        $$ A_i = 
            \begin{cases}
                \mathbb{R} \quad \text{for} \quad i \in [0, \frac{1}{2}] \\
                \mathbb{N} \quad \text{for} \quad i \in (\frac{1}{2}, 1]
            \end{cases} $$

    \tcblower

    The set of functions $f : [0, 1] \to \mathbb{R}$ with the restriction that $f(x) \in \mathbb{N}$ when $x > \frac{1}{2}$.
\end{problem}

\begin{itemize}
    \item Set of \textbf{functions} $f : A \to B$:
        \begin{itemize}
            \item $\{ \forall x \in A, \exists ! y \in B : f(x) = y \}$
            \item Set theorist's interpretation: $\{ (x, y) \in A \times B :
                f(x) = y \subseteq A \times B \}$
        \end{itemize}
\end{itemize}

\begin{problem}{p. 17}{}

    \marginnote{Show $\mathbb{N} \sim \mathbb{N} \times \mathbb{N}$ (tedious method).}

    \begin{enumerate}[a)]
        \item Show $\mathbb{N} \sim \mathbb{N} \times \mathbb{N}$.
        \item Show $\mathbb{N} \sim \mathbb{N} \times \mathbb{N} \times \ldots \times \mathbb{N}$.
    \end{enumerate}

    \tcblower

    \begin{enumerate}[a)]
        \item 
    \end{enumerate}

\end{problem}

\begin{problem}{p. 18}{}
    
    \marginnote{Proof of Cantor's theorem. (Similar to Russell's paradox.)}

    Prove Cantor's theorem: $\lvert A \rvert \neq \lvert \mathcal{P}(A) \rvert$.

    \tcblower

    Suppose by contradiction that $\lvert A \rvert = \lvert \mathcal{P}(A) \rvert$. Then there is a bijection $f : A \to \mathcal{P}(A)$.

    Consider the set $S = \{ a \in A : a \not \in f(a) \}$. (\underline{Example}: suppose $f(a) = \{a, b, c\}$ and $f(b) = \{a, c\}$. Then $a \not \in S$ but $b \in S$.)

    Since $f$ is surjective, there exists a $y \in A$ such that $f(y) = S$, since $S \in \mathcal{P}(A)$. But
        \begin{itemize}
            \item $y \in S \implies y \in \mathcal{P}(A) \implies y \not \in S$, and
            \item $y \not \in S \implies y \in S$,
        \end{itemize}
    so we have a paradoxical contradiction
        $$ y \in S \iff y \not \in S . $$
    
\end{problem}

\begin{itemize}
    \item $\lvert A \rvert \leq \lvert B \rvert$ if there is an \textit{injection} $f : A \to B$.
    \item \textbf{Injection} notation: $f : A \xhookrightarrow{} B$
    \item \textbf{Surjection} notation: $f : A \twoheadrightarrow B$
    \item \textbf{Schröder-Bernstein theorem}: let $A$ and $B$ be sets. If there exists injections $f : A \to B$ and $g : B \to A$, then there exists a bijection $h : A \to B$.
\end{itemize}

\begin{problem}{24}{}

    \marginnote{Show that $[0, 1]$ and $[0, 1)$ have the same cardinality.}

    See margin.

    \tcblower

    Consider the two injections:
    \begin{itemize}
        \item $f : [0, 1] \to [0, 1)$ \quad with \quad $x \mapsto \frac{x}{2}$
        \item $g : [0, 1) \to [0, 1]$ \quad with \quad $x \mapsto x$
    \end{itemize}
    By the Schröder-Bernstein theorem, there exists a bijection between $[0, 1]$ and $[0, 1)$, so they have the same cardinality.

\end{problem}

\begin{problem}{N/A}{cardinality_reals_0_1}

    \marginnote{$\lvert [0, 1] \rvert = \lvert \mathbb{R} \rvert$.}

    Prove the margin.

    \tcblower

    See problem \ref{prob:cardinality_set_functions}.

\end{problem}

\begin{problem}{p. 25}{}

    \marginnote{Show that $\mathbb{N}$ and $\mathbb{N} \times \mathbb{N}$ have the same cardinality.}

    See margin.

    \tcblower

    Consider the following injections:
    \begin{itemize}
        \item $f : \mathbb{N} \to \mathbb{N} \times \mathbb{N}$ \quad with \quad $n \mapsto (0, n)$
        \item $g : \mathbb{N} \times \mathbb{N} \to \mathbb{N}$ \quad with \quad $(m, n) \mapsto 2^m 3^n$
    \end{itemize}

    By Schröder-Bernstein, $\lvert \mathbb{N} \rvert = \lvert \mathbb{N} \times \mathbb{N} \rvert$.

\end{problem}

\begin{itemize}
    \item A set $S$ is \textbf{Dedekind-infinite} if there is a bijection from $S$ to a \textit{proper} subset of itself. (Otherwise, $S$ is Dedekind-finite).
        \begin{itemize}
            \item \textit{Assuming AC},
                \begin{align*}
                    \text{Dedekind-infinite} &\iff \text{infinite} \\
                    \text{Dedekind-finite} &\iff \text{finite}
                \end{align*} 
        \end{itemize}
\end{itemize}

\begin{problem}{p. 33}{}

    \marginnote{Show $\mathbb{Q}$ is countable.}

    See margin.

    \tcblower

    Take any $q = \frac{a}{b}$ (simplified) and write it as $(a, b)$. Therefore
        $$ \mathbb{Q} \sim \mathbb{Z} \times \mathbb{Z}
            \sim \mathbb{N} \times \mathbb{N} \sim \mathbb{N} . $$
    Hence $\mathbb{Q}$ is countable.
    
\end{problem}

\begin{itemize}
    \item The set $\mathcal{P}(\mathbb{N})$ can be identified with the set of \textbf{infinite binary bitstrings}. These sets are uncountably infinite.
\end{itemize}

\begin{problem}{p. 36}{}
    
    \marginnote{Show that $[0, 1]$ is uncountable using bitstrings.}

    See margin.

    \tcblower

    We can inject infinite binary bitstrings into $[0, 1]$. But the set of infinite binary bitstrings has the same cardinality as $\mathcal{P}(\mathbb{N})$, which is uncountably infinite.
    
    Therefore $\lvert [0, 1] \rvert \leq \lvert \mathcal{P}(\mathbb{N}) \rvert$, so $[0, 1]$ is uncountably infinite.

\end{problem}

See lecture notes: Theorem 1.3.6. Nice proof.

\begin{problem}{p. 39}{}

    \marginnote{Countable union of countable sets is countable.}

    Show that the set $S$ of finite subsets of $\mathbb{N}$ is countable.

    \tcblower

    Let $S_n$ be the set of finite subsets of $S$ of size $n$. Then show $S = \cup_{n \in \mathbb{N}} S_n$.

    We want to show that $S_n$ is comparable to $\mathbb{N}^n$; i.e. it is countable.
    
    Using the theorem above, we conclude $S$ is countable.
    
\end{problem}

\begin{problem}{Problem sheet 1, Q2}{}

    \marginnote{Explicit bijection between $\mathbb{N}$ and $\mathbb{N} \times \mathbb{N}$.}

    See margin.

    \tcblower

    Use diagonals from upper right to lower left. Then note (where $T_n$ is the $n$th triangular number):

    $$ A_{m, 0} = T_{m+1} = \sum_{i=1}^{m+1} = \frac{(m+1)(m+2)}{2} $$

    Therefore:

    \begin{align*}
        A_{m, n} &= A_{m, n-1} + m + n \\
        &= A_{m, n-2} + (m + n - 1) + (m + n) \\
        &\ \ \vdots \\
        &= A_{m, 0} + (m + 1) + (m + 2) + \cdots + (m + n) \\
        &= A_{m, 0} + nm + \frac{n (n+1)}{2} \\
        &= \frac{(m+1)(m+2)}{2} + nm + \frac{n (n+1)}{2} ,
    \end{align*}

    i.e. our bijection is the function
        $$ (m, n) \mapsto \frac{(m+1)(m+2)}{2} + nm + \frac{n (n+1)}{2} $$
    
\end{problem}

\begin{problem}{Problem sheet 1, Q3}{}

    \marginnote{Transitivity property of cardinality.}

    Prove carefully from the definition that if $\lvert A \rvert \leq \lvert B \rvert$ and $\lvert B \rvert \leq \lvert C \rvert$, then $\lvert A \rvert \leq \lvert C \rvert$.

    \tcblower

    From the definition, $\lvert A \rvert \leq \lvert B \rvert$ implies that there is an injection from $A$ to $B$. Similarly, there is an injection from $B$ to $C$.

    Now the entirety of $A$ is mapped to a subset of $B$, and the entirety of $B$ (and hence $\operatorname{range}(A)$) is mapped to a subset of $C$. Therefore there exists an injection from $A$ to $C$; that is, $\lvert A \rvert \leq \lvert C \rvert$.
    
\end{problem}

\begin{problem}{Problem sheet 1, Q4}{}

    \marginnote{Surjectivity as definition of cardinality inequality.}

    Prove that if $A$ is a nonempty set, then $\lvert A \rvert \leq \lvert B \rvert$ if and only if there exists an onto map $g : B \to A$ (assuming Axiom of Choice).

    \tcblower

    \underline{$f \ \text{injective} \ \implies g \ \text{surjective}$}:
    
    Define $g$ as follows: for each element in $\operatorname{range}(f)$, let $g = f^{-1}$. For those leftover elements $C = B \setminus \operatorname{range}(f)$, (assuming AC) choose any element $x_0 \in A$ and define $g(y) = x_0$ for all $y \in C$. We have therefore constructed $g$ to be a surjective function.

    \underline{$g \ \text{surjective} \ \implies f \ \text{injective}$}:

    For each $x \in A$, there is some set $S_x \subseteq B$ such that $g(y) = x$ for each $y \in S_x$. Define $f$ so that for each $x \in A$, (assuming AC) choose one element $y_0$ from $S_x$ and set $f(x) = y_0$. Then $f$ is injective.
    
\end{problem}

\begin{problem}{Problem set 1, Q5}{ps1_q5}

    \marginnote{$\lvert A \rvert = \lvert A \cup B \rvert$ when $A$ is infinite and $B$ is countable.}
    
    Prove the margin.

    \tcblower

    Interleaving method: take $\{a_n\}_{n \in \mathbb{N}}$ in $A$ and $\{b_n\}_{n \in \mathbb{N}}$ in $B$ and `interleave' $\{b_n\}_{n \in \mathbb{N}}$ between $\{a_n\}_{n \in \mathbb{N}}$.

\end{problem}

\begin{problem}{Problem set 1, Q6}{}

    \marginnote{Countable union of \textit{uncountable} sets.}

    Suppose that $\lvert A_n \rvert = \lvert \mathbb{R} \rvert$, for $n = 1, 2, 3, \ldots$. Prove that
        $$ \biggl\vert \bigcup_{n=1}^\infty A_n \biggr\vert
            = \lvert \mathbb{R} \rvert . $$

    \tcblower

    ?

\end{problem}

\begin{problem}{Problem set 1, Q7}{}

    \marginnote{$\lvert \mathbb{R} \setminus \mathbb{Q} \rvert = \lvert \mathbb{R} \rvert$}

    Prove the margin.

    \tcblower

    Using problem \ref{prob:ps1_q5}, let $A = \mathbb{R} \setminus \mathbb{Q}$ and $B = \mathbb{Q}$. Then $\lvert A \rvert = \lvert A \cup B \rvert$, i.e.
        $$ \lvert \mathbb{R} \setminus \mathbb{Q} \rvert = \lvert \mathbb{R} \rvert , $$
    as required.

\end{problem}

\begin{problem}{Problem sheet 1, Q8}{}

    how's it goin' fellas

\end{problem}

\begin{problem}{Problem sheet 1, Q9 (assignment 1)}{}

    \marginnote{Number of chess games with finite moves is countable.}

    In the game of chess, two players take turns moving pieces on an $8 \times 8$ grid.  The game ends when a state called \textit{checkmate} is reached.  Is the number of different possible chess games countable or uncountable?  (By a ``chess game'', we mean a game which ends in checkmate; we do not consider games that go on forever). Prove your answer.
    
    \underline{Hint}: You do not need to know the specific rules for how chess pieces move or what is considered checkmate to answer the question.

    \tcblower

    [\underline{NOTE}: This answer is unnecessarily complicated and does not consider positions where pieces have been captured and removed from the board.]

    \textit{I claim that the number chess games with finite moves is countable. To prove this, I represent positions and games as sets and sequences, and then use the theorem on slide 38 of Chapter 1.}
        
    There are 64 squares on a chess board and 32 pieces. Assuming all pieces are distinguishable from each other (e.g. Mike and Pradeep are the two white bishops), there are a total of $\frac{64!}{32!}$ ways to put the pieces on the board. The number of \textit{legal} positions $P$ --- that is, positions reachable only by a sequence of legal moves --- is less than $\frac{64!}{32!}$ and is hence finite.

    Let $P_n$ be the set of legal board positions that require $n$ moves to reach, starting from the initial position $p_0 \in P_0$. (Note that there are positions that are in multiple distinct $P_n$; this is irrelevant.) Now $P_n \subset P$ for any number of moves $n$, so $P_n$ is always finite.

    What is a chess game, then? It is simply a sequence of legal transitions between legal positions. We can describe the set $C_m$ of all chess games of $m$ moves as a subset of the Cartesian product
        $$ P_0 \times P_1 \times P_2 \times \cdots \times P_m . $$
    (Note that this product contains games with illegal moves between legal positions.) Now the Cartesian product of finitely many finite sets is finite, so any $C_m$ is finite.

    For the last step, note that the set of all possible games $C$ (with potentially illegal moves) is 
        $$ C = \bigcup_{m \in \mathbb{N}} C_m , $$
    which is a countable union of finite sets. Hence $C$ is countable. Now the number of \textit{legal} chess games of finite moves is a subset of $C$, and is also therefore countable.
\end{problem}

\begin{problem}{Problem sheet 1, Q10 (assignment 1)}{cardinality_set_functions}
    
    \marginnote{Cardinality of a set of functions.}

    Which of the following two sets has a greater cardinality, or are the cardinalities equal?

    \begin{enumerate}
        \item The set $S$ of functions from the interval $[0, 1]$ to $\mathbb{N}$
        \item The set $\mathbb{R}$ of real numbers
    \end{enumerate}

    Prove your answer.

    \tcblower

    I claim that
        $$ \lvert S \rvert
            \geq \lvert \mathcal{P}([0, 1]) \rvert
            > \lvert [0, 1] \rvert
            = \lvert \mathbb{R} \rvert , $$
    and hence that $\lvert S \rvert > \lvert \mathbb{R} \rvert$.

    To prove the first inequality, we can show that there is an injection
        $$ g : \mathcal{P}([0, 1]) \to S 
            \qquad \text{given by} \qquad
            T \mapsto f$$
    such that
        $$ f(y) = 
            \begin{cases}
                1 \qquad \text{if} \ y \in T \\
                0 \qquad \text{otherwise}.
            \end{cases} $$

    The second inequality comes from (the inequality version of) Cantor's theorem.

    To prove the equality, note that the sigmoid function
        $$ f(x) = \frac{e^x}{1 + e^x} $$
    is an injection from $\mathbb{R}$ to $(0, 1) \subset [0, 1]$. On the other hand, the map $x \mapsto x$ injects $[0, 1]$ into $\mathbb{R}$, and so by the Schröder-Bernstein theorem, the two sets are in bijection and hence $\lvert [0, 1] \rvert = \lvert \mathbb{R} \rvert$.

\end{problem}

\section{Metric spaces}

\begin{itemize}
    \item \textbf{Metric} conditions:
        \begin{enumerate}
            \item $d(x, y) = 0 \iff x = y$
            \item $d(x, y) = d(y, x)$
            \item $d(x, y) + d(y, z) \geq d(x, z)$ \quad (triangle inequality)
        \end{enumerate}
\end{itemize}

\begin{problem}{Problem set 2, Q1}{}
    
    \marginnote{Metric spaces: proofs.}

    Which of the following are metric spaces? (If they are not, can you `fix' the problem?)

    \begin{enumerate}[i)]
        \item $X = \mathbb{R}$ with $d(x, y) = \sqrt{\lvert x - y \rvert}$.
        \item $X = c_{00}$, the set of real sequences $\mathbf{x} = (x_k)_{k = 1}^\infty$ which have only finitely many non-zero terms, with $d(x,y) = \sum_{k = 1}^\infty \lvert x_k - y_k \rvert$.
        \item $X = $ all airports served by Qantas, with $d(A, B) = $ minimum travel time on Qantas flights to get from $A$ to $B$.
        \item $X = $ all $100$ digit bit-strings, with $d(v, w) = $ the number of digits where $v$ and $w$ differ.
        \item $X = $ the vector space of all real polynomials, with $d(p, q) = \displaystyle\int_0^3 \lvert p(t) - q(t) \rvert \ dt$.
        \item $X = $ the vector space of all real polynomials, with $d(p, q) = \displaystyle\sum_{k = 0}^\infty \frac{\lvert p(k) - q(k) \rvert}{2^k}$.
    \end{enumerate}

    \tcblower

    \begin{enumerate}[i)]
        \item $(X, d)$ is a metric space. Criteria 1 and 2 are easy. Then, if $x' = x - y$ and $y' = y - z$, we have
            \begin{align*}
                d(x, y) + d(y, z) &= \sqrt{\lvert x - y \rvert}
                    + \sqrt{\lvert y - z \rvert} \\
                &\geq \sqrt{\lvert x - z \lvert} \\
                &= d(x, z) ,
            \end{align*}
        since $\lvert x' \rvert + \lvert y' \rvert \geq \lvert x' + y' \rvert$.
        \item $(X, d)$ is a metric space for similar reasons to part i).
        \item $(X, d)$ is not a metric space because the minimum flying time from $A$ to $B$ may not be the same as from $B$ to $A$. Also, it is not guaranteed that Qantas has flights from every airport to \textit{every other} airport.
        \item $(X, d)$ is a metric space. Criteria 1 and 2 are easy. By contradiction, let us suppose that this space fails to meet criterion 3. Then, there must be at least one $k$ such that $d(x_k, y_k) + d(y_k, z_k) < d(x_k, z_k)$, in which case $d(x_k, z_k) \neq 0$ and hence $d(x_k, z_k) = 1$. Then there are two cases: $x_k = y_k$, and $x_k \neq y_k$. In both cases, we have $d(x_k, y_k) + d(y_k, z_k) = 1$ and so by contradiction transitivity must hold.
        \item $(X, d)$ is a metric space. Proof is similar to part i) with some additional subtlety when dealing with criterion 1: first, note that polynomials are continuous so we cannot have a constant zero polynomial with discontinuous jumps. Second, $d(p, q) = 0$ requires $p$ and $q$ to be zero on the interval $[0, 3]$. However, the only way for $p$ and $q$ to have uncountably many zeros is if they are identically $0$.
        \item $(X, d)$ is a metric space.
    \end{enumerate}
    
\end{problem}

\begin{problem}{Problem set 2, Q4}{}

    \marginnote{$\operatorname{Int}(Y) = Y$ in finite metric spaces.}

    Let $(X, d)$ be a metric space with only finitely many elements. Prove that every subset of $X$ must be open (and closed!).

    \tcblower

    Since $X$ is finite, for any $y \in Y$, there is some $z_y \in X$ that is \textit{closest} to $y$. Setting $\epsilon = \frac{1}{2} \cdot d(y, z_y)$ then yields a ball that sits entirely within $Y$. Hence, $\operatorname{Int(Y) = Y}$ and so every subset of $X$ is open. This also means that every subset is closed.
    
\end{problem}

\begin{problem}{Problem set 2, Q5}{}

    \marginnote{Symmetric difference of sets as metric.}

    Let $X = \mathcal{P}(S)$ where $S$ is finite, and let $d(A, B) = \lvert A \Delta B$. Show that $(X, d)$ is a metric space.
    
    \tcblower

    For criterion 3, use a Venn diagram and label the sub sections $D$ through $F$ (we can ignore) $A \cap B \cap C$. Then the desired inequality is easy to prove.

\end{problem}

\subsection{Interior, closure, boundary}

\begin{itemize}
    \item \textbf{Interior} (greatest open subset):
        $$ \operatorname{Int}(Y) = \{ y \in Y : \exists \epsilon > 0 
            \ \text{such that} \ B(y, \epsilon) \subseteq Y \} . $$
    \item \textbf{Closure} (smallest closed superset): $\operatorname{Cl}(Y) = \operatorname{Int}(Y) \sqcup \operatorname{Bd}(Y)$
    \item A set $Y \subset (X, d)$ is \textbf{open} if $\operatorname{Int}(Y) = Y$, and \textbf{closed} if $Y^c$ is open.
    \item \textbf{Boundary}:
        $$ \operatorname{Bd}(Y) = X \setminus (\operatorname{Int}(Y)
            \cup \operatorname{Int}(Y^c)) . $$
\end{itemize}

\begin{problem}{Problem sheet 2, Q10}{}

    \marginnote{\textit{Limit points}, interior, closure, boundary of union of varied sets (interesting).}

    Let $X = \mathbb{R}$ with the usual metric and let
        $$ Y = \{\sin k\}_{k = 1}^\infty \cup (0, 2)
            \cup \{ 3, 3.1, 3.14, 3.141, 3.1415, \ldots \} . $$
    
    \begin{enumerate}[i)]
        \item What are the interior points of $Y$?
        \item What are the boundary points of $Y$?
        \item What are the limit points of $Y$?
        \item What is the closure of $Y$?
    \end{enumerate}

    \underline{Note}: \textit{limit points} are points in the closure of a set that aren't `isolated'.

    \tcblower

    \begin{enumerate}[i)]
        \item $(0, 2)$
        \item $[-1, 0] \cup \{2\} \cup \{ 3, 3.1, 3.14, \ldots \}
            \cup \{ \pi \}$
        \item $[-1, 2] \cup \{ \pi \}$
        \item $[-1, 2] \cup \{ 3, 3.1, 3.14, \ldots \}
            \cup \{ \pi \}$
    \end{enumerate}

\end{problem}

\begin{problem}{Problem sheet 2, Q11}{}

    \marginnote{Closure theorems.}

    Suppose that $Y$ is a subset of a metric space $(X, d)$.

    \begin{enumerate}[i)]
        \item Prove that $\operatorname{cl(Y)}$ is closed in $(X, d)$.
        \item Prove that $Y$ is closed if and only if $Y = \operatorname{cl}(Y)$.
        \item Let $\mathcal{S}$ denote the set of all closed subsets of $X$ which contain $Y$, that is
            $$ \mathcal{S} = \{ C \subseteq X : C \
            \text{is closed and} \ Y \subseteq C \} . $$
        \begin{enumerate}[(a)]
            \item Explain why $\mathcal{S}$ is always nonemtpy.
            \item Prove that if $C \in \mathcal{S}$ then $\operatorname{cl}(Y) \subseteq C$ (and hence that $\operatorname{cl}(Y)$ is the `smallest' closed set cotaining $Y$).
            \item Prove that $\operatorname{cl}(Y) = \cap_{C \in \mathcal{S}} C$.
        \end{enumerate}
    \end{enumerate}

    \tcblower

    \begin{enumerate}[i)]
        \item $\operatorname{cl}(Y)^c = \operatorname{Int}(Y^c)$ is open, so $\operatorname{cl}(Y)$ is closed.
        \item \begin{align*}
            Y \ \text{closed} &\iff Y^c \ \text{open} \\
                &\iff \operatorname{Int}(Y^c) = Y^c \\
                &\iff \operatorname{cl}(Y) = Y .
                    &&\text{(complement of both sides)}
        \end{align*}
        \item \begin{enumerate}[(a)]
            \item The entire metric space is closed and contains $Y$, so at the very least $X \subseteq \mathcal{S}$.
            \item Since $\operatorname{cl}(Y)$ is closed, Q\ref{prob:closed_alternate} says that the limit of all of its convergent sequences are contained within $\operatorname{cl}(Y)$. But since $Y \subseteq C$, and because $C$ is also closed, the limit of every convergent sequence in $\operatorname{cl}(C)$ is also contained in $C$. Hence $\operatorname{cl}(C) \subseteq C$.
            \item Consider $C_1 \subseteq C_2 \subseteq \cdots \subseteq X$. Then
                \begin{align*}
                    c \in C_1 &\iff c \in C_1
                        \ \text{and} \ c \in C_2
                        \ \text{and} \ \cdots 
                        \ \text{and} \ c \in X \\
                        &\iff c \in \bigcap_{C \in \mathcal{S}} C .
                \end{align*}
            But from part (b), $\operatorname{cl}(Y) = C_1$, so
                $$ \operatorname{cl}(Y)
                    = \bigcap_{C \in \mathcal{S}} C . $$
        \end{enumerate}
    \end{enumerate}
    
\end{problem}

\begin{problem}{Problem sheet 3, Q1 (assignment 2)}{}

    \marginnote{Interior, closure, boundary of subset of $(\ell^1, \lVert \cdot \rVert_1)$.}

    Let $S \subset (\ell^1, \lVert \cdot \rVert_1)$ be the set of sequences $\{ x_n \}_{n=1}^\infty \in \ell^1$ such that
        $$ x_{2n} = x_{2n - 1}, \quad \forall n . $$
    What are the interior, closure, and boundary of $S$? (Prove your answer.)

    \tcblower

    An $\epsilon$-ball around any sequence in $(\ell^1, \lVert \cdot \rVert_1)$ can be written as
        \begin{align*}
            B \bigl( \{ x_n \}_{n=1}^\infty, \epsilon \bigr)
                &= \biggl\{ \{ y_n \}_{n=1}^\infty \in S :
                \Bigl\Vert
                    \{ y_n \}_{n=1}^\infty
                    - \{ x_n \}_{n=1}^\infty 
                \biggr\Vert_1 < \epsilon \Bigr\} \\
            &= \biggl\{
                \{ y_n \}_{n=1}^\infty \in S :
                \sum_{n = 1}^\infty
                \bigl\vert x_n - y_n \bigr\vert < \epsilon
            \biggr\} .
        \end{align*}
    Given \textit{any} sequence $\{ x_n \}_{n=1}^\infty \in S$ and \textit{any} positive $\epsilon$, we can choose some positive $\delta$ less than $\frac{\epsilon}{2}$ and construct a new sequence
        $$ \{ x'_n \}_{n=1}^\infty = (x_1 - \delta, x_1 + \delta, x_3, x_3, x_5, x_5, \ldots) , $$
    which lies outside $S$ but within $\ell^1$, since
        $$ \sum_{n = 1}^\infty
            \bigl\vert x_n - x'_n \bigr\vert
            = 2 \delta < \epsilon < \infty . $$
    However, $\{ x'_n \}_{n=1}^\infty$ also lies in the $\epsilon$-ball of $\{ x_n \}_{n=1}^\infty$, so the interior of $S$ is empty.

    Next, let us consider sequences in $S^c$. For \textit{any} such sequence, choose
        $$ \epsilon < \frac{1}{2} \cdot \inf_{i, j \in \mathbb{Z}^+} \bigl\{ \lvert x_i - x_j \rvert \bigr\}, \quad x_i \neq x_j . $$
    Then the $\epsilon$-ball around $\{ x_n \}_{n=1}^\infty$ is wholly contained within $S^c$; this means that the interior of $S^c$ is itself.

    To conclude, we have
        \begin{align*}
            \operatorname{Bd}(S)
                &= \ell^1 \setminus (\operatorname{Int}(S) \cup \operatorname{Int}(S^c)) \\
            &= \ell^1 \setminus S^c \\
            &= S ,
        \end{align*}
    which also implies that the closure of $S$ is itself.
        
\end{problem}

\subsection{Topology}

\begin{itemize}
    \item $\mathcal{O}(X)$ is the \textbf{topology} of $X$, with properties:
        \begin{enumerate}
            \item $\emptyset, X \in \mathcal{O}(X)$
            \item Union of open sets is open
            \item \textit{Finite} intersection of open sets is open
        \end{enumerate}
\end{itemize}

\begin{problem}{Problem sheet 2, Q8}{}

    \marginnote{Unions of open sets.}

    
    
\end{problem}

\begin{itemize}
    \item \textbf{Minkowski's inequality}
\end{itemize}

\begin{problem}{5}{}
    .
\end{problem}

\begin{problem}{6}{}
    .    
\end{problem}

\begin{problem}{12}{}
    prove property 1 (iff 0)
\end{problem}

\subsection{Convergence}

\marginnote{\underline{IMPORTANT} \underline{NOTE}} `Open' and `closed' are NOT logical opposites, so one must use the $\epsilon$-ball definition \textit{directly} when proving via contradiction or via contrapositive.

\begin{itemize}
    \item \textbf{Cauchy sequences}: in general,
        $$ \text{convergent} \implies \text{Cauchy} , $$
    but not vice versa. The utility of Cauchy sequences is that they don't \textit{presuppose} the existence of a limit. In a complete metric space, though, all Cauchy sequences converge to a limit.
\end{itemize}

\begin{problem}{Problem sheet 2, Q7}{closed_alternate}

    \marginnote{Sequences in $Y$ converge iff $Y$ closed (important).}

    Let $(X, d)$ be a metric space, and let $Y \subseteq X$. Show that $Y$ is closed iff the limit of every convergent sequence $\{y_n\}_{n=1}^\infty \subseteq Y$ is in $Y$.

    \tcblower

    Note that
        \begin{align*}
            Y \ \text{closed} &\iff Y^c \ \text{open} \\
                &\iff \forall x \in Y^c, \ \exists \epsilon > 0
                \ \text{such that} \ B(x, \epsilon)
                \subseteq Y^c \tag{$A$} .
        \end{align*}
    We can also write the contrapositive of the second statement as
        \begin{equation}
            y \in Y \qquad
            \forall \{y_n\} \subseteq Y
                \ \text{where} \ y_n \to y .  \tag{$B$}
        \end{equation}
    Let us first prove $A \implies B$. If $A$ is true, then there is an $\epsilon$-ball around any $x \in Y^c$, so if there is a sequence in $Y$ that converges, its limit must be within $Y$.

    For $A \impliedby B$, let us prove the contrapositive $\neg A \implies \neg B$. We can write $\neg A$ as:
        \begin{equation}
            \exists x \in Y^c
                \ \text{such that} \ B(x, \epsilon) \not \subseteq Y^c
                \ \forall \epsilon > 0 . \tag{$\neg A$}
        \end{equation}
    Then $\neg B$ is:
        \begin{equation}
            \exists \{y_n\} \subseteq Y \ \text{where} \ y_n \to x
            \ \text{such that} \ x \in Y^c . \tag{$\neg B$}
        \end{equation}
    Now if $\neg A$ is true, then we can form an $\epsilon$-ball around $x$ that necessarily intersects with $Y$---therefore, we can construct a sequence wholly within $Y$ that converges to $x$, and so $\neg B$ is also true.
    
\end{problem}

\begin{problem}{Problem sheet 3, Q10 (assignment 2)}{}

    \marginnote{Cauchy sequence in $L^1$. Showing no limit in $L^1$.}

    Let $X = C[0,1]$ with the metric $d_1$. Consider the sequence of piecewise linear functions
    $$ f_n(x) =
        \begin{cases}
            0 & 0 \leq x \leq \frac{1}{2} - \frac{1}{n} \\
            \frac{1}{2} - \frac{n}{4} + \frac{n}{2} x
                & \frac{1}{2} - \frac{1}{n} < x
                < \frac{1}{2} + \frac{1}{n} \\
            1 & \frac{1}{2} + \frac{1}{n} \leq x \leq 1
        \end{cases} ,
        \qquad n \geq 2 .
    $$
    \begin{enumerate}
        \item Sketch the first few functions in the sequence.
        \item Show that this is a Cauchy sequence.
        \item Show that the sequence does not converge. (Do not refer to any larger spaces such as $L^p$ spaces---rather, show directly that the sequence does not converge in the $d_1$ metric to any function in $C[0,1])$.
    \end{enumerate}

    \tcblower

    \begin{enumerate}
        \item Use the following Mathematica code:
        \begin{lstlisting}
f[x_, n_] = Piecewise[{{0, 0 <= x <= 1/2 - 1/n}, {1/2 - n/4 + (n x)/2, 1/2 - 1/n <= x <= 1/2 + 1/n}, {1, 1/2 + 1/n <= x <= 1}}]
Plot[f[x, 4], {x, 0, 1}]
        \end{lstlisting}
        \item Let $m, n \in \mathbb{Z}^+$ and assume without loss of generality that $m < n$. Then
            $$ d_1 \bigl( f_m(x), f_n(x) \bigr)
                = \int_0^1 \bigl\vert f_m(x) - f_n(x) \bigr\vert \ dx . $$
        Instead of solving this integral directly, we can draw a picture of $f_m$ and $f_n$ like so:

        \begin{center}
            \begin{tikzpicture}[scale=0.5]
                \draw (0, 0) -- (10/6, 0)
                    -- (50/6, 10) -- (10, 10);
                \draw (0, 0) -- (3, 0)
                    -- (7, 10) -- (10, 10);
                \draw[densely dotted] (27/13, 40/64) -- (3, 0);
                \draw[dotted] (5, 10) -- (5, 0);
                \node at (10/6, 0) [below] {A};
                \node at (5, 5) [right] {B};
                \node at (3, 0) [below] {C};
                \node at (27/13, 40/64) [left] {M};
            \end{tikzpicture}
        \end{center}

        and find the areas of the two triangles. Let us begin by calculating the perpendicular distance between $\overline{AB}$ and $C$. The line $\overline{AB}$ has equation
            $$ 2mx - 4y + (2 - m) = 0 , $$
        so its perpendicular distance to $C$ is
            $$
                \lvert \overline{MC} \rvert = \frac{\Bigl\vert 
                    2m \cdot (\frac{1}{2} - \frac{1}{n})
                    + (2 -  m) \Bigr\vert}
                    {\sqrt{4m^2 + 16}}
                    = \frac{\Bigl\vert 
                    \frac{2}{n} (n - m) \Bigr\vert}
                    {\sqrt{4m^2 + 16}} 
                    = \frac{\frac{2}{n} (n - m)}
                    {\sqrt{4m^2 + 16}} .
            $$
        (The absolute value signs are redundant since $n > m$.) Next, we have
            $$
                \lvert \overline{AB} \rvert
                    = \sqrt{\biggl( \frac{1}{2} \biggr)^2
                    + \Biggl( \frac{1}{2} 
                    - \biggl( \frac{1}{2} - \frac{1}{m} \biggr)
                    \Biggr)^2 }
                    = \frac{1}{4m} \sqrt{4m^2 + 16} ,
            $$
        and so the two triangles have a total area of
            $$ 
                d_1 \bigl( f_m(x), f_n(x) \bigr)
                    = 2 \cdot \frac{1}{2}
                    \cdot \lvert \overline{AB} \rvert
                    \cdot \lvert \overline{MC} \rvert
                    = \frac{n - m}{2mn} .
            $$
        We can now show that the sequence is Cauchy. Set $K(\epsilon) = \frac{1}{2 \epsilon}$. Then whenever $m, n > K$,
            \begin{align*}
                \frac{n - m}{2mn}
                    &= \frac{1}{2m} - \frac{1}{2n} \\
                    &< \frac{1}{2m} \\
                    &< \frac{1}{2 K} \\
                    &= \frac{1}{2 \frac{1}{2 \epsilon}} \\
                    &= \epsilon .
            \end{align*}
        In other words, for any $\epsilon > 0$, there is a $K(\epsilon)$ such that
            $$ m, n > K \qquad \implies
                \qquad d_1 \bigl( f_m(x), f_n(x) \bigr)
                < \epsilon , $$
        and therefore $\{ f_n \}_{n = 2}^\infty$ is a Cauchy sequence.
        \item \textit{My idea for the following argument was inspired by the MathSoc Discord channel; I improved the idea by being more careful with how I split the integral.} Suppose there is some function $f(x)$ to which the sequence converges. Then we would have
            $$ \lim_{n \to \infty} d_1(f_n(x), f(x)) = 0 , $$
        i.e.
            $$ \lim_{n \to \infty} \int_0^1 \bigl\vert
                f_n(x) - f(x) \bigr\vert \ dx = 0 . $$
        We can split the integral like so:
            \begin{align*}
                \lim_{n \to \infty} \int_0^1 \bigl\vert
                f_n(x) - f(x) \bigr\vert \ dx
                &= \lim_{n \to \infty} \biggl(
                    \int_0^{\frac{1}{2} - \frac{1}{n}}
                    \bigl\vert f_n(x) - f(x) \bigr\vert \ dx \\
                &\qquad \qquad + \int_{\frac{1}{2} - \frac{1}{n}}^
                    {\frac{1}{2} + \frac{1}{n}}
                    \bigl\vert f_n(x) - f(x) \bigr\vert \ dx \\
                &\qquad \qquad + \int_{\frac{1}{2} + \frac{1}{n}}^1 
                    \bigl\vert f_n(x) - f(x) \bigr\vert \ dx \biggr) \\
                &= \int_0^\frac{1}{2}
                    \bigl\vert 0 - f(x) \bigr\vert \ dx
                    + \int_\frac{1}{2}^1
                    \bigl\vert 1 - f(x) \bigr\vert \ dx .
            \end{align*}
        We want this expression to equal zero, which, because of the absolute value signs, necessitates \textit{both} integrals equalling zero. Therefore, when $0 \leq x \leq \frac{1}{2}$, the function $f(x)$ must equal $0$ except for at finitely many points. Similarly, when $\frac{1}{2} \leq x \leq 1$, $f(x)$ must equal $1$---except at finitely many points. However, this means that $f(x)$ will necessarily be discontinuous at $x = \frac{1}{2}$, and so the sequence $\{ f_n \}_{n = 2}^\infty$ does not converge in $C[0, 1]$.
    \end{enumerate}

\end{problem}


\section{Sequences and series of functions}

\section{Topological spaces}

\section{Compactness}
% \chapter{Differential equations}

\section{Linear Ordinary Differential Equations}

\subsection{Forms cheat sheet}

\begin{itemize}
    \item \textbf{Separable}:
        $$ y'(t) = f(t) g(y) . $$
    Solve with
        $$ \int \frac{1}{g(y)} dy = \int f(t) dt . $$
    \item \textbf{Linear}:
        $$ y'(t) + a(t) y(t) = f(t) $$
    Solve with integrating factor (an ansatz)
        $$ \mu(t) = e^{\int a(t) dt} $$
    in
        $$ y(t) = \frac{1}{\mu(t)} \int \mu(t) f(t) dt
            + \frac{C}{\mu(t)} . $$
    \item An ODE of the form
        $$ F(x, y) + G(x, y) \frac{dy}{dx} $$
    is called \textbf{exact} if
        $$ \frac{\partial F}{\partial y}
            = \frac{\partial G}{\partial x} . $$
    \item \textbf{Reduction of order} method: given $u = u_1(x) \neq 0$, a solution to the ODE
        $$ u'' + p(x)u' + q(x)u = 0 , $$
    on some interval $I$, then a second solution is
        $$ u = u_1(x) \int \frac{1}{u_1^2
            \operatorname{exp}\bigl\{ \int p(x) dx \bigr\}} dx . $$
\end{itemize}

\newpage

\subsection{Basic methods}

\begin{problem}{Problem set 1, Q1}{}

    \marginnote{Separable, linear, exact, homogeneous (with solutions).}

    Classify and solve:
        \begin{enumerate}
            \item $4xy' = 5y$
            \item $(x + 1)^2 y' + 3(x + 1)y = 4$
            \item $(x^2 + y^2) y' + 2xy = 0$
            \item $2xy dx + x^2 dy = 0$
        \end{enumerate}

    \tcblower

    \begin{enumerate}
        \item Separable, linear, homogeneous. Rewrite as
            $$ y' = \frac{5y}{4x} . $$
        Then use the technique for separable ODEs:
            \begin{align*}
                \int \frac{1}{5y} dy
                    &= \int \frac{1}{4x} dx \\
                \ln \lvert y \rvert^4
                    &= \ln \lvert x \rvert^5 + c_1 \\
                y &= c_2 x^\frac{5}{4} .
            \end{align*}
        \item Linear, inhomogeneous. Rewrite as
            $$ y' + \frac{3}{x+1}y = \frac{4}{(x+1)^2} . $$
        Then, find an integrating factor:
            \begin{align*}
                \mu(x) &= \exp \biggl\{ 
                    \int \frac{3}{x+1} dx \biggr\} \\
                &= \exp \{ 3 \ln \lvert x + 1 \rvert + c_1 \}] \\
                &= c_2 (x+1)^3 .
            \end{align*}
        Therefore,
            \begin{align*}
                y(x) &= \frac{1}{c_2 (x+1)^3}
                    \int c_2 (x+1)^3 \frac{4}{(x+1)^2} dx
                    + \frac{c_3}{c_2(x+1)^3} \\
                &= \frac{4}{(x+1)^3} \int (x + 1) dx 
                    + \frac{c_4}{(x+1)^3} \\
                &= \frac{2x^2 + 4x}{(x+1)^3}
                    + \frac{c_5}{(x+1)^3} \\
                &= \frac{2(x^2 + 2x + 1) - 2}{(x+1)^3}
                    + \frac{c_5}{(x+1)^3} \\
                &= \frac{2}{x + 1} + \frac{c_6}{(x+1)^3} .
            \end{align*}
        \item Homogeneous. Using the substitution $y = v(x) \cdot x$ gives
            $$ v' = \frac{3v - 3v^3}{1 + v^2} x^{-1} . $$
        Then, using the method for separable equations, we have
            \begin{align*}
                \int \frac{1 + v^2}{-3v - v^3} dv
                    &= \int \frac{1}{x} dx \\
                -\frac{1}{3} \ln \lvert 3v + v^3 \rvert
                    &= \ln \lvert x \rvert + c_1 \\
                3v + v^3 &= \frac{c^3}{x^3} \\
                \frac{3y}{x} + \frac{y^3}{x^3} &= \frac{c^3}{x^3} \\
                y(y^2 + 3x^2) &= c_3 .
            \end{align*}
        \item Separable, linear, exact. Rewrite as
            $$ 2xy \ dx = -x^2 \ dy \implies y' = -\frac{2xy}{x^2}
                = -\frac{2y}{x} $$
        Using the method for separable equations,
            \begin{align*}
                \int -\frac{1}{2y} dy &= \int \frac{1}{x} dx \\
                -\frac{1}{2} \ln \lvert y \rvert
                    &= \ln \lvert x \rvert + c_1 \\
                \ln \lvert y \rvert &= \ln \lvert x \rvert^{-2} + c_2 \\
                y &= c_3 x^{-2} \\
                x^2 y &= c_3 .
            \end{align*}
    \end{enumerate}
    
\end{problem}

\begin{problem}{Problem set 1, Q1 (iii, iv)}{}
        
    \marginnote{Second-order ODEs (incl. complex solutions).}

    Find general solutions of:
    \begin{enumerate}
        \item $u''+ 2u' + 2u = 0$
        \item $u'' + 6u' + 9 = 0$
    \end{enumerate}

    \tcblower
    
    \begin{enumerate}
        \item Solutions to characteristic polynomial are
            $$ r = -1 \pm i , $$
        and so
            \begin{align*}
                u(x) &= c_1 e^{(-1 + i)x} + c_2 e^{(-1 - i)x} \\
                &= e^x \biggl( 
                    c_1 (\cos x + i\sin x) 
                    + c_2 (\cos x - i\sin x) \biggr) \\
                &= e^x \biggl( c_3 \cos x
                    + c_4 \sin x \biggr) \\
                &= c_5 e^x \sin(x - c_6) .
            \end{align*}
        \item Characteristic polynomial factorises to $(r + 3)^2 = 0$, so the solution is
            $$ u(x) = e^{-3x} (c_1 + c_2 x) . $$
    \end{enumerate}

\end{problem}

\begin{problem}{Problem set 1, Q3}{}

    \marginnote{Selected ODEs with initial conditions.}

    Solve:
    \begin{enumerate}
        \item $t \frac{dx}{dt} = x + \sqrt{t^2 + x^2}$ with $x(1) = 0$
        \item $2xe^y + e^x + (x^2 + 1) e^y \frac{dy}{dx} = 0$ with $y = 0$ when $x = 0$
    \end{enumerate}

    \tcblower

    \begin{enumerate}
        \item Use the substitution $x = v(t) \cdot t$, which yields
            $$ v' = \frac{1}{t} \sqrt{1 + v^2} . $$
        Separable method:
            \begin{align*}
                \int \frac{1}{\sqrt{1 + v^2}} dv 
                    &= \int \frac{1}{t} dt \\
                \arcsinh(v) &= \ln \lvert t \rvert + c_1 \\
                x &= t \sinh \bigl( \ln \lvert t \rvert + c_1 \bigr)
            \end{align*}
        Initial conditions give $c_1 = 0$, so
            \begin{align*}
                x &= t \sinh \bigl( \ln \lvert t \rvert \bigr) \\
                &= t \frac{e^{\ln \lvert t \rvert}
                    - e^{-\ln \lvert t \rvert}}{2} \\
                &= \frac{t^2 - 1}{2} .
            \end{align*}
        \item \marginnote{[Exact ODE.]} Note that the exactness criterion holds, so we have
            $$ \int \frac{d}{dx} \bigl( 
                    (x^2 + 1)e^y \bigr) dx 
                    = \int -e^x dx , $$
        which yields
            \begin{align*}
                (x^2 + 1) e^y = -e^x + C ,
            \end{align*}
        whereby we find $C = 2$.
    \end{enumerate}

\end{problem}

\begin{problem}{Problem set 1, Q5}{}

    \marginnote{Reduction of order.}

    Given that $u_1 = \sin x$ is a solution of
        $$ xu'' - u' + 4x^3 u = 0 , $$
    use reduction of order to find a second, linearly independent solution.

    \tcblower

    Reduction of order technique:
        \begin{align*}
            u &= \sin x^2
                \int \frac{1}{\sin^2 x^2 \operatorname{exp}
                    \bigl\{ \int \frac{1}{x} \bigr\} dx } dx \\
            &= \sin x^2 \int \frac{x}{\sin^2 x^2} dx \\
            &= \sin x^2 \cdot (-\frac{1}{2} \cot x^2 + c_1) \\
            &= -\frac{1}{2} \cos x^2 + c_1 \sin x^2 .
        \end{align*}
    We can therefore take $-\frac{1}{2} \cos x^2$ as the second solution.

\end{problem}

\begin{problem}{Problem set 1, Q6}{}

    \marginnote{Reduction of order, interesting IBP.}

    Given that $u_1 = 1 + x$ is a solution of
        $$ xu'' - (1 + x)u' + u = 0 , $$
    use reduction of order to find the general solution.

    \tcblower

    Using the method of reduction of order:
        \begin{align*}
            u &= (1 + x) \int \frac{1}{(1 + x)^2 \operatorname{exp}
                \bigl\{ \int -\frac{1+x}{x} dx \bigr\}} dx \\
            &= c_1 (1 + x) \int \frac{x e^x}{(1 + x)^2} dx \\
            &= c_1 (1 + x) \int x e^x \frac{d}{dx}
                \cdot \biggl( \frac{-1}{1 + x} \biggr) dx \\
            &= c_1 (1 + x) \biggl[ 
                \frac{-x e^x}{1 + x}
                + \int \frac{e^x (1 + x)}{1 + x} dx \biggr] 
                \qquad \text{(IBP)} \\
            &= e^x + c_2 (x + 1) .
        \end{align*}
    Hence another particular solution is $e^x$, so the general solution is
        $$ u_G(x) = Ae^x + B(1 + x) . $$

\end{problem}

\subsection{Wronskian}

    \begin{align*}
        \text{set of functions linearly dependent} &\implies W(x) = 0 \\
        \text{set of functions linearly independent} &\impliedby W(x) \neq 0 \ \text{at at least one point}
    \end{align*}

For solutions to a non-singular, linear, homogeneous, $m$th-order ODE $Lu = 0$ on $[a, b]$, there are one of two cases:
    \begin{align*}
        W(x) & = 0 \
            \text{and the $m$ solutions are linearly \textit{dependent}} \\
        W(x) &\neq 0 \
            \text{and the $m$ solutions are linearly \textit{independent}}
    \end{align*}

\begin{problem}{Problem set 1, Q10}{}

    \marginnote{Wronskian satisfying first-order ODE.}

    For each of the following ODEs, write down the linear, first-order ODE satisfied by the Wronskian $W$ of any two, linearly independent solutions, and hence find the form of $W$.

    \begin{enumerate}
        \item $(1 - x^2) u'' - 2 x u' + \nu (\nu + 1) = 0$ \quad (Legendre's equation of order $\nu$)
        \item $x^2 u'' + xu' + x^2 u = 0$ \quad (Bessel's equation of order $0$)
    \end{enumerate}

    \tcblower

    \begin{enumerate}
        \item The Wronskian $W$ satisfies
            $$ (1 - x^2)W' - 2xW = 0 , $$
        a separable equation that yields the solution
            $$ W = \frac{c}{1 - x^2} . $$
        \item The Wronskian satisfies
            $$ x^2 W' + xW = 0 , $$
        a separable equation that gives
            $$ W = \frac{c}{x} . $$
    \end{enumerate}

\end{problem}

\begin{problem}{Problem set 1, Q11}{}

    \marginnote{Wronskians: important edge cases.}

    Compute the Wronskian for each set of functions and then determine whether or not they are linearly independent on the given interval.

    \begin{enumerate}
        \item $x^3, \lvert x \rvert^3$ on $\mathbb{R}$
        \item $e^{\lambda x}, xe^{\lambda x}, x^2 e^{\lambda x}$ on $[a, b]$
    \end{enumerate}

    \tcblower

    \begin{enumerate}
        \item For $x \geq 0$ and $x < 0$, the Wronskians are respectively
            $$ 
                \begin{vmatrix}
                    x^3 & x^3 \\
                    3x^2 & 3x^2
                \end{vmatrix}
                \qquad \text{and} \qquad
                \begin{vmatrix}
                    x^3 & -x^3 \\
                    3x^2 & -3x^2
                \end{vmatrix} .
            $$
        In both cases the determinant is $0$, however we can set up the equation
            $$ ax^3 + b \lvert x \rvert^3 = 0 , $$
        and, substituting (for example) $x = 1$ and $x = -1$ shows $a = b = 0$ so the two functions are, in fact, linearly independent.
        \item Using the Wolfram Mathematica command
            \begin{center}
                \lstinline|Wronskian[{E^(d x), x E^(d x), x^2 E^(d x)}, x]|
            \end{center}
        gives the result $2 e^{\lambda x}$. Therefore, the functions are linearly independent.
    \end{enumerate}

\end{problem}

\subsection{Variation of parameters}

We seek $v_1$ and $v_2$ such that
    $$ u_P(x) = v_1(x) u_1(x) + v_2(x) u_2(x) , $$
with the following Ansätze:
    $$ v_1'(x) = \frac{-u_2(x) f(x)}{W(x)}
        \qquad \text{and} \qquad
        v_2'(x) = \frac{u_1(x) f(x)}{W(x)} , $$
where $W(x)$ is the Wronskian of $u_1$ and $u_2$.

\begin{problem}{Problem set 1, Q12}{}

    \marginnote{Variation of parameters.}

    Use the technique of \textit{variation of parameters} to find the general solution to each of the following inhomogeneous ODEs.

    \begin{enumerate}
        \item $u'' - u = \frac{2}{1 + e^x}$
        \item $u'' + u = \cosec x$
        \item $u'' + 2u' + u = e^{-x} \sec^2 x$
    \end{enumerate}

    \tcblower

    \begin{enumerate}
        \item \marginnote{[Easy.]} The characteristic equation gives solutions $\pm 1$, so the homogeneous solution is
            $$ u_H(x) = c_1 e^x + c_2 e^{-x} . $$
        Using variation of parameters then gives
            \begin{align*}
                v_1'(x) &= \frac{-e^{-x} \cdot
                \frac{2}{1 + e^x}}{1 + e^x}
                    = \frac{1}{e^x (1 + e^x)} \\
                v_2'(x) &= \frac{e^x \cdot
                    \frac{2}{1 + e^x}}{1 + e^x}
                    = \frac{-e^x}{1+e^x} .
            \end{align*}
        Integrating, we find
            $$ v_1(x) = -e^{-x} - x + \log(1 + e^x) + c_3 $$
        and
            $$ v_2(x) = -\log(1 + e^x) + c_4 . $$
        \underline{Note}: for $v_1(x)$, use partial fractions and an $1 + e^x - e^x$ trick.

        Hence
            $$ u_P(x) = e^x \bigl[ c_3 + \log(1 + e^x) - x \bigr]
                + e^{-x} \bigl[ c_4 - \log(1 + e^x) \bigr] - 1 , $$
        and of course $u_G(x) = u_P(x) + u_H(x)$.
        \item \marginnote{[Complex solutions.]} We find the homogeneous solution to be
            $$ u_H(x) = c_1 \cos x + c_2 \sin x , $$
        whereby the Wronskian is $1$. Using variation of parameters,
            \begin{align*}
                v_1'(x) &= -\sin x \cdot \cosec x = -1 \\
                v_2'(x) &= \cos x \cdot \cosec x = \cot x .
            \end{align*}
        and so
            $$ v_1(x) = -x + c_3 \qquad \text{and} \qquad
                v_2(x) = \log(\sin x) + c_4 . $$
        Therefore
            $$ u_P(x) = \cos x (-x + c_3) +
                \sin x (\log(\sin x) + c_4) $$
        and so
            $$ u_G(x) = -x \cos x + \sin x \log \lvert \cos x \rvert
                + A \cos x + B \sin x . $$
        \item \marginnote{[Difficult integral.]} Homogeneous solution:
            $$ u_H(x) = c_1 e^{-x} + c_2 x e^{-x} . $$
        Wronskian is $e^{-2x}$. We therefore have
            $$ v_1'(x) = -x \sec^2 x 
                \qquad \text{and} \qquad
                v_2'(x) = \sec^2 x , $$
        which gives
            $$ v_1(x) = -\log \lvert \cos x \rvert - x \tan x
                \qquad \text{and} \qquad
                v_2(x) = \tan x . $$
        Hence
            $$ u_G(x) = -e^{-x} \log \lvert \cos x \rvert 
                + Ae^{-x} + Bxe^{-x} . $$
    \end{enumerate}

\end{problem}

  
% \chapter{Abstract algebra}

\section{Cheat sheet}

$$ \text{fields} \subset \text{EDs} \subset \text{PIDs} \subset \text{UFDs}
    \subset \text{integral domains} \subset \text{rings} $$

\begin{center}
    \begin{tabular}{ |c||c|c|c|c|c|c|c| } 
        \hline
            & comm.
            & \begin{tabular}{@{}c@{}}no 0 \\ divs.\end{tabular}
            & \begin{tabular}{@{}c@{}}prime iff \\ irreduc.\end{tabular}
            & \begin{tabular}{@{}c@{}}ideals \\ p'pal\end{tabular}
            & \begin{tabular}{@{}c@{}}ideal max if\\$\neq$ 0, pr./irr.\end{tabular}
            & \begin{tabular}{@{}c@{}}Euclid. \\ alg.\end{tabular}
            & \begin{tabular}{@{}c@{}}unit if \\ $\neq$ 0\end{tabular} \\
        \hhline{|=||=|=|=|=|=|=|=|}
            ring & & & & & & & \\ 
        \hline
            domain & \cmark & \cmark 
            & \begin{tabular}{@{}c@{}} prime \\ $\implies$ irr.\end{tabular} 
            & & & & \\ 
        \hline
            UFD & \cmark & \cmark & \cmark & & & & \\ 
        \hline
            PID & \cmark & \cmark & \cmark & \cmark & \cmark & & \\
        \hline
            ED & \cmark & \cmark & \cmark & \cmark & \cmark & \cmark & \\
        \hline
            field & \cmark & \cmark & \cmark & \cmark & \cmark & \cmark & \cmark \\
        \hline
    \end{tabular}
    \newline
    \small Note: prime ideal has no `fundamental' definition
\end{center}

\begin{align*}
    r = ab &\implies a \ \text{or} \ b \ \text{a unit} &&\textbf{Irreducible} \\
    p \mid ab &\implies p \mid a \ \text{or} \ p \mid b &&\textbf{Prime}
\end{align*}

Rings:
    \begin{itemize}
        \item Not commutative: $M_n(\mathbb{R})$ ($n \times n$ matrices over $\mathbb{R}$)
        \item Quotient rings:
            \begin{itemize}
                \item Definition:
                    $$ R/I = \{ [r] = r + I : r \in R \} $$
                \item Intuition:
                    $$ \frac{\mathbb{Q}[x]}{x^4 - 2} = \{ \text{polynomials of degree $<4$, or reduced with $x^4 - 2 \equiv 0$} \} $$
                \item $R/R = \{0\}$
                \item $R/\{0\} = R$
            \end{itemize}
    \end{itemize}

Ideals:
    \begin{itemize}
        \item $\mathbb{Z}$: $n\mathbb{Z}$
        \item $\mathbb{Z}[i]$: $(a + bi) \mathbb{Z}[i]$
        \item Maximal ideals:
            \begin{itemize}
                \item $2\mathbb{Z}$, but not $10\mathbb{Z}$, in $\mathbb{Z}$
            \end{itemize}
        \item Prime ideal that is not maximal: $\{0\}$ in $\mathbb{Z}$ (see propositions \ref{maximal-ideal-field} and \ref{prime-ideal-domain})
        \item Commutative ring that is not a PID, but in which every ideal is principal: $\mathbb{Z}/4\mathbb{Z}$
    \end{itemize}

Fields:
    \begin{itemize}
        \item $\mathbb{Q}, \mathbb{R}, \mathbb{C}$ but \textit{not} $\mathbb{Z}$ (every element must have an inverse)
    \end{itemize}

Finite fields:
    \begin{itemize}
        \item $\mathbb{F}_p$ is a field with $p$ elements
            \begin{itemize}
                \item $\mathbb{Z}/p\mathbb{Z} = \mathbb{F}_p$ when $p$ is prime because $(p)$ is maximal.
            \end{itemize}
        \item Consider $f = x^2 + x + 1$, which is irreducible in $\mathbb{F}_2[x]$. Then $K = \mathbb{F}_2[x]/f$ is a field of order 4 (any $x^k$ with $k \geq 2$ is reduced somehow). Let $\alpha$ be the coset of $x$; then $K = \{ 0, 1, \alpha, \alpha + 1 \}$.
    \end{itemize}
    
Field extensions:
    \begin{itemize}
        \item Finitely generated but not finite: $\mathbb{Q}(\pi)/\mathbb{Q}$
    \end{itemize}
    
Fraction fields:
    \begin{itemize}
        \item $\mathbb{Q} = \operatorname{Frac}(\mathbb{Z})$
        \item $\mathbb{Q}[i] = \operatorname{Frac}(\mathbb{Z}[i])$
    \end{itemize}

Euclidean domains:
    \begin{itemize}
        \item \underline{Not a field}: $\mathbb{Z}$
    \end{itemize}

Principal ideal domains:
    \begin{itemize}
        \item \underline{Not an ED}: $\mathbb{Z}[(1 + \sqrt{19}) / 2]$
        \item Any field $\mathbb{F}$, including $\mathbb{Q}, \mathbb{R}, \mathbb{C}$
        \item $\mathbb{Z}$
        \item $\mathbb{F}[x]$: rings of polynomials in one variable with coefficients in a \textit{field}
        \item $\mathbb{Z}[i]$: the Gaussian integers
    \end{itemize}
    
Unique factorisation domains:
    \begin{itemize}
        \item \underline{Not a PID}: $\mathbb{Z}[x]$, $\mathbb{C}[x, y]$
    \end{itemize}
    
Integral domains:
    \begin{itemize}
        \item \underline{Not a UFD}: $\mathbb{Z}[\sqrt{-5}]$
        \item Commutative, not a field: $\mathbb{Z}$
    \end{itemize}
    
Irreducibility:
    \begin{itemize}
        \item $2x$ is irreducible over $\mathbb{Q}[x]$ but not over $\mathbb{Z}[x]$
        \item $2$ is irreducible but not prime in $\mathbb{Z}[\sqrt{5}]$
        \item Irreducible polynomial in $\mathbb{F}_3[x]$: the polynomial $x^3 - x + 1$ (no roots in $\mathbb{F}_2$, so it has no linear factors, so it must be irreducible)
    \end{itemize}
    
Units:
    \begin{itemize}
        \item $\pm 1$ in $\mathbb{Z}$
        \item $\mathbb{Q} - \{0\}$ in $\mathbb{Q}$
        \item $\{\pm 1, \pm i\}$ in $\mathbb{Z}[i]$
    \end{itemize}

Conjugacy classes:
    \begin{itemize}
        \item $S_4 : \{ 1, (12), (123), (1234), (12)(34) \}$
        \item $A_4 : \{ 1, (123), (132), (12)(34) \}$
    \end{itemize}

\newpage

\loadgeometry{margins}

\section{Group actions}

Let $G$ act on $A$, and let $J \subseteq G$.
\begin{itemize}
    \item \textbf{Kernel}: $\ker(G) = \{ g \in G : ga = a \quad \forall a \in A \}$
        \begin{itemize}
            \item \textbf{Centraliser} (kernel under conjugation): $C_G(A) = \{ g \in G : aga^{-1} = g \quad \forall a \in A \}$
                \begin{itemize}
                    \item \textbf{Centre}: $Z(G) = C_G(G) = \{ g \in G : hg = gh \quad \forall h \in G \}$
                \end{itemize}
        \end{itemize}
    \item \textbf{Stabiliser} (fixed $a$): $\operatorname{stab}_G(a) = G_a = \{ g \in G : g a = a \}$
    \item \textbf{Orbit}: $[a] = Ga = \{ ga : g \in G \}$
        \begin{itemize}
            \item \textbf{Conjugacy class} (orbit under conjugation): $[a] = Ga = \{ gag^{-1} : g \in G \}$
        \end{itemize}
    \item \textbf{Fixed point set} of $J$: $A^J = \{ a \in A : ja = a \quad j \in J \}$
\end{itemize}

\section{Ring homomorphisms}

Ring homomorphism $\phi : R \to R'$:
    \begin{itemize}
        \item $\phi(r + s) = \phi(r) + \phi(s)$
        \item $\phi(rs) = \phi(r) \phi(s)$
        \item $\phi(1_R) = 1_{R'}$
    \end{itemize}

\begin{itemize}
    \item \textbf{Kernel}: $\ker(\phi) = \{r \in R : \phi(r) = 0_{R'}\}$
\end{itemize}

\begin{problem}{Q11, wk8\textsuperscript{1}}{}

    \marginnote{Kernel of ring homomorphism.}

    Describe the kernel of the following ring homomorphisms:
    \begin{enumerate}[a)]
        \item $\varphi : \mathbb{R}[x] \to \mathbb{C}$ defined by $p \mapsto p(2 + i)$,
        \item $\psi : \mathbb{Z}[x] \to \mathbb{R}$ defined by $p \mapsto p(1 + \sqrt{2})$.
    \end{enumerate}

    \hrulefill

    \begin{enumerate}[a)]
        \item Let $z = 2 + i$. Then
                \begin{align}
                    z - 2 &= i \\
                    (z - 2)^2 &= -1 \\
                    z^2 - 4z + 5 &= 0 .
                \end{align}
            Define $q(z) = z^2 - 4z + 5$ so that $q \subseteq \ker(\varphi)$. Now $q$ is irreducible over $\mathbb{R}$, so $(q)$ is a maximal ideal of $\mathbb{R}[x]$ (see diagram).  $\langle q \rangle$ 
    \end{enumerate}

\end{problem}

\section{Ideals}

\begin{itemize}
    \item Prove: nonempty, closed under subtraction, closed under multiplication by all elements of $R$
\end{itemize}

\begin{proposition}{Maximal/prime ideal}{maximal-prime-ideal}
    \textsc{Suppose}:
        \begin{itemize}
            \item $R$ is a commutative ring
            \item $I$ is an ideal of $R$
        \end{itemize}
    \textsc{Then}:
        \begin{align*}
            I \ \text{is maximal} &\iff R/I \ \text{is a field} \\
            I \ \text{is prime}   &\iff R/I \ \text{is an integral domain}            
        \end{align*}
\end{proposition}

\begin{problem}{Q3 wk8\textsuperscript{1}}{}    

    \marginnote{Ideal zero on some set.}

    Let $Y \subseteq \mathbb{C}^n$. The \textit{ideal of polynomials zero on} $Y$ is
        $$ I(Y) = \{ f \in \mathbb{C}[x_1, \ldots, x_n] : f(\mathbf{y}) = 0 \ \text{for all} \ y \in Y \} . $$
    \begin{enumerate}[a)]
        \item Show that $I(Y) \trianglelefteq \mathbb{C}[x_1, \ldots, x_n]$.
        \item Let $Y = \{ (0, m) : m \in \mathbb{Z} \} \subset \mathbb{C}^2$. Compute the ideal $I(Y)$.
    \end{enumerate}

    \tcblower

    \begin{enumerate}[a)]
        \item
            \begin{itemize}
                \item For any $f, g \in I(Y)$, $(f - g)(y) = f(y) - g(y) = 0$ for all $y \in Y$, so $I(Y)$ is closed under subtraction and therefore a subgroup of $\mathbb{C}[x_1, \ldots, x_n]$.
                \item Now $I(Y)$ is an absorbing subset of $\mathbb{C}[x_1, \ldots, x_n]$ under multiplication: take $f \in \mathbb{C}[x_1, \ldots, x_n]$ and $g \in I(Y)$; then $(fg)(y) = f(y) g(y) = 0$ for all $y \in Y$.
            \end{itemize}
            Therefore $I(Y)$ is an ideal of $\mathbb{C}[x_1, \ldots, x_n]$.
        \item $I(Y) = \{ f \in \mathbb{C}[x, y] : f(0, m) = 0 \ \text{for all} \ m \in \mathbb{Z} \}$. All such $f$ must therefore have infinite zeros in $y$, which is only possible when $y = 0$. Hence all polynomials in $I(Y)$ are of the form $ax$ for some $a \in \mathbb{C}$.
    \end{enumerate}

\end{problem}

\begin{problem}{Q4 wk8\textsuperscript{1}}{}

    \marginnote{First/Second Isomorphism Theorems. Ideal zero on some set. Maximal ideal (Prop \ref{prop:maximal-prime-ideal}).}

    Let $Y \subseteq \mathbb{C}^n$ and $\mathbf{y} \in Y$.
        \begin{enumerate}[a)]
            \item Show that $I(\mathbf{y}) \supseteq I(Y)$.
            \item Show that $I(\mathbf{y})/I(Y)$ is a maximal ideal of $\mathbb{C}[Y] = \mathbb{C}[x_1, \ldots, x_n]/I(Y)$.
        \end{enumerate}

    \tcblower

    \begin{enumerate}[a)]
        \item Take any $f \in I(Y)$. Then $f(\mathbf{x}) = 0$ for all $\mathbf{x} \in Y$, so in particular $f \in I(\mathbf{y})$, and the statement follows.
        \item By the Second Isomorphism Theorem,
            $$ \cfrac{\ \cfrac{\mathbb{C}[Y]}{I(Y)} \ }{\cfrac{I(\mathbf{y})}{I(Y)}} \cong \frac{\mathbb{C}[Y]}{I(\mathbf{y})} . $$
        We want to show that this expression is a field. Consider the homomorphism
            $$ \varphi_\mathbf{y} : \mathbb{C}[Y] \to \mathbb{C} \quad \text{given by} \quad f \mapsto f(\mathbf{y}) . $$
        Now $\ker(\varphi_\mathbf{y}) = \{ f \in \mathbb{C}[Y] : f(\mathbf{y}) = 0 \} = I(\mathbf{y})$. By the First Isomorphism Theorem,
            $$ \frac{\mathbb{C}[Y]}{\ker(\varphi_\mathbf{y})} = \frac{\mathbb{C}[Y]}{I(\mathbf{y})} \cong \mathbb{C} , $$
        and so $\mathbb{C}[Y] / I(\mathbf{y})$ is a field. By Proposition 23.7, $I(\mathbf{y})/I(Y)$ is therefore the maximal ideal of $\mathbb{C}[Y]$.
    \end{enumerate}

\end{problem}

\begin{problem}{Q8, wk8\textsuperscript{1}}{}

    \marginnote{Generated subring.}

    Find the subring $\mathbb{C}[x^2, x^3]$ of $\mathbb{C}[x]$ generated by $\mathbb{C}$, $x^2$, and $x^3$. The nicest description is obtained by writing a basis for the underlying complex vector space of $\mathbb{C}[x^2, x^3]$.

    \tcblower

    The basis of $\mathbb{C}[x^2, x^3]$ is $\{ x^{2k + 3\ell} : k, \ell \in \mathbb{N}_0 \}$, so that
        $$ \mathbb{C}[x^2, x^3] = \Bigl\{ \sum_{k, \ell} a_{k, \ell} x^{2k + 3\ell} : i, j \in \mathbb{N}_0 \Bigr\} . $$

\end{problem}

\begin{problem}{Q9, wk8\textsuperscript{1}}{}

    \marginnote{Existence of nonzero integer in ideal of Gaussian intgers.}

    Prove that any nonzero ideal of $\mathbb{Z}[i]$ (the Gaussian integers) contains a nonzero integer.

    \tcblower

    The representatives of any ideal $I$ of $\mathbb{Z}[i]$ are of the form $a + bi$ where $a, b \in \mathbb{Z}$. If we stipulate that $I$ be nonzero, then at least one of $a, b$ must be nonzero. Now take $a + bi \in I$ and $a - bi \in \mathbb{Z}[i]$. Necessarily (by the properties of ideals), $(a + bi)(a - bi) = a^2 + b^2 \in \mathbb{Z}$ must be in $I$, so $I$ contains at least one nonzero integer.

\end{problem}

\section{Euclidean Algorithm}

Note that the greatest common denominator of two polynomials is unique \textit{up to multiplication by an invertible constant}.

\begin{problem}{Q10, wk8\textsuperscript{1}}{}

    \marginnote{Euclidean algorithm on polynomials.}

    Let $I$ be the ideal of $\mathbb{Z}[i]$ generated by $5$ and $-4 + 2i$. Find some $z \in \mathbb{Z}[i]$ such that $I$ is generated by $z$.

    \tcblower

    We will use the extended Euclidean Algorithm to find $\gcd(5, -4 + 2i)$. Using the norm $N(a + bi) = a^2 + b^2$, we find $N(5) > N(-4 + 2i)$. (In the calculations below, round any fractional quotients \textit{down} to the nearest Gaussian integer, before finding the remainder, similar to the algorithm on integers: $7 = 3.5 \times 2 = 3 \times 2 + 1$.)
    \begin{align*}
        5 &= (-1-i)(-4 + 2i) + (-1 - 2i) \\
        -4 + 2i &= (-2i) (-1 - 2i)
    \end{align*}
    The last nonzero remainder is $-1 - 2i$, which is the gcd we are looking for. (Note: we could also write the gcd as $1 + 2i$.) Hence $I = (-1 - 2i) = (1 + 2i)$.

\end{problem}

\begin{problem}{Q2b, 2015 final}{}

    \marginnote{Euclidean algorithm on $\mathbb{Z}[\sqrt{-2}]$.}

    Let $\alpha = \sqrt{-2}$ and let $\nu : \mathbb{Z}[\alpha] \to \mathbb{Z}$ be the usual Euclidean function given by $\nu(x + y \alpha) = x^2 + 2y^2$.

    Given $b = 7 - 2 \alpha$ and $c = 3 + \alpha$, find $q, r \in \mathbb{Z}[\alpha]$ such that $b = qc + r$ and $\nu(r) < \nu(c)$.

    \tcblower

    First we find
        $$ \frac{b}{c} = \frac{7 - 2\alpha}{3 + \alpha} \cdot \frac{3 - \alpha}{3 - \alpha}
            = \frac{17}{11} - \frac{13}{11} \alpha . $$
    Rounding to the \textit{nearest lattice point} gives $q = 2 - \alpha$, and so $r = b - qc = -1 - \alpha$, so that
        $$ 7 - 2 \alpha = (3 + 2 \alpha)(2 - \alpha) + (-1 - \alpha) . $$

\end{problem}

\section{Chinese remainder theorem}

\begin{theorem}{Chinese remainder theorem}{chinese-remainder-theorem}
    \textsc{Suppose}:
        \begin{itemize}
            \item $R$ is a ring
            \item $I_1, \ldots, I_n$ ideals of $R$
                \begin{itemize}
                    \item $I_i + I_j = R$ for each pair $i \neq j$
                    \item Let $I = \bigcap_j I_j$
                \end{itemize}
            \item Homomorphism $\phi_j : R/I \to R/I_j$
        \end{itemize}
    \textsc{Then}:
        $$ \phi : R/I \to R/I_1 \times \cdots \times R/I_n, \qquad
            r + I \mapsto (r + I_1, r + I_2, \ldots, r + I_n) $$
    is an isomorphism.
\end{theorem}

\section{Finite fields}

\begin{itemize}
    \item If $F$ is a field, then $\lvert F \rvert = p^n$, where $p$ is prime and called the \textbf{characteristic} of $F$.
\end{itemize}
% \section{Calculus}

\subsection{Continuity and differentiability}

\begin{shaded}
\textbf{Theorem (limit of function of several variables using vector sequences) \cite{math2111_notes}.} Given some function $f : \mathbb{R}^n \to \mathbb{R}$, the following identity holds:
$$
    \lim_{\mathbf{x} \to \mathbf{a}} f(\mathbf{x}) = b \quad
    \iff \quad \lim_{n \to \infty} f(\mathbf{x}_n) = b \qquad
    \forall \mathbf{x}_n \in \mathbb{R}^n : \lim_{n \to \infty} = \mathbf{a}
$$
\end{shaded}

\begin{shaded}
\textbf{Definition (continuous function) \cite{hubbard_hubbard}.} Let $X \subset \mathbb{R}^n$. Then a mapping $\mathbf{f} : X \to \mathbb{R}^m$ is continuous at $\mathbf{x}_0 \in X$ if
$$ \lim_{\mathbf{x} \to \mathbf{x}_0} \mathbf{f}(\mathbf{x}) = \mathbf{f}(\mathbf{x}_0); $$
$\mathbf{f}$ is continuous on $X$ if it is continuous at every point of $X$.

Equivalently, $\mathbf{f}: X \to \mathbf{R}^m$ is continuous at $\mathbf{x}_0$ if and only if for every $\epsilon > 0$, there exists $\delta > 0$ such that when $|\mathbf{x} - \mathbf{x}_0| < \delta$, then $|\mathbf{f}(\mathbf{x}) - \mathbf{f}(\mathbf{x}_0)| < \epsilon$.
\end{shaded}

\begin{shaded}
\textbf{Definition (continuity and discontinuity) \cite{math2111_notes}.} Suppose $[a, b] \subset \mathbb{R}$. Consider a function $f : [a, b] \to \mathbb{R}$ and a point $c \in \mathbb{R}$. Supposed the one-sided limits
$$ f(c^-) = \lim_{x \to c^-} f(x) \qquad \textnormal{and} \qquad f(c^+) = \lim_{x \to c^+} f(x) $$
exist.

(1) If
$$ f(c^-) = f(c^+) = f(c) , $$
then $f$ is \textbf{continuous} at $c$.

(2) If
$$ f(c^-) = f(c^+) \not= f(c) $$
--- regardless of whether or not $f(c)$ exists --- then $f$ has a \textbf{removable discontinuity} at $c$.

(3) If
$$ f(c^-) \not= f(c^+) , $$
then $f$ has a \textbf{jump discontinuity} at $c$.
\end{shaded}

\begin{shaded}
\textbf{Definition (piecewise continuous function) \cite{math2111_notes}.} A piecewise function is continuous on $[a, b] \subseteq \mathbb{R}$ if

(1) $f(x^-)$ exists for each $x \in (a, b]$;

(2) $f(x^+)$ exists for each $x \in [a, b)$;

(3) $f$ is continuous on $(a, b)$ except at (at most) a finite number of points where there exist jump discontinuities.

Moreover, $f$ is piecewise continuous on $\mathbb{R}$ if it is piecewise continuous on any finite interval $[a, b] \subseteq \mathbb{R}$.
\end{shaded}

\begin{shaded}
\textbf{Definition (differentiability) \cite{math2111_notes}.} Consider a function $ f : \mathbb{R} \to \mathbb{R} $ and a point $c \in \mathbb{R}$. Let
$$ f(c^-) = \lim_{x \to c^-} f(x) \qquad \textnormal{and} \qquad f(c^+) = \lim_{x \to c^+} f(x) , $$
and let
\begin{align*}
	D^+ f(c) &= \lim_{h \to 0^+} \frac{f(c + h) - f(c^+)}{h} \qquad \text{and} \\
	D^- f(c) &= \lim_{h \to 0^-} \frac{f(c + h) - f(c^-)}{h} .
\end{align*}
%Then $f$ is differentiable at $c$ if and only if $f(c^+) = f(c^-)$ and $D^+ f(c) = D^- f(c)$.

Warning: $D^{+/-} f(c)$ is not necessarily the same as $f'(c^{+/-}) = \lim_{x \to c^{+/-}} f'(x)$.
\end{shaded}

\begin{shaded}
\textbf{Definition (piecewise differentiable function) \cite{math2111_notes}.} A function $f$ is piecewise differentiable on $[a, b] \subseteq \mathbb{R}$ if

(1) $D^- f(x)$ exists for each $x \in (a, b]$;

(2) $D^+ f(x)$ exists for each $x \in [a, b)$;

(3) $f$ is differentiable on $(a, b)$ except at (at most) a finite number of points.

Moreover, $f$ is piecewise differentiable on $\mathbb{R}$ if it is piecewise differentiable on any finite interval $[a, b] \subseteq \mathbb{R}$.
\end{shaded}

\subsection{Fourier series and convergence}

\begin{shaded}
\textbf{Fourier series \cite{math2111_notes}.} Consider a function $f : \mathbb{R} \to \mathbb{R}$ which is $2 L$-periodic and is square integrable (i.e., $\int_{-\pi}^\pi f(x)^2 \ dx < \infty $). Its Fourier series is given by
$$ S_f(x) = \frac{a_0}{2} + \sum_{k = 1}^{\infty} \left( a_k \cos \left( \frac{k \pi}{L}x \right) + b_k \sin \left( \frac{k \pi}{L}x \right) \right) . $$
We can view the Fourier series as expression a function $f$ as a linear combination of the orthogonal functions $\frac{1}{2}, \cos(kx), ..., \sin(kx), k \geq 1$. Then, via vector decomposition, we can find the following formulas to yield the Fourier coefficients $a_k$ and $b_k$:
$$ a _k = \frac{1}{L} \int_{-L}^L f(x) \cos \left( \frac{k \pi}{L}x \right) \ dx, \quad k = 0, 1, 2, ... $$
and
$$ b _k = \frac{1}{L} \int_{-L}^L f(x) \sin \left( \frac{k \pi}{L}x \right) \ dx, \quad k = 1, 2, 3, ... $$
\end{shaded}

\begin{shaded}
\textbf{Definition (pointwise convergence) \cite{math2111_notes}.} Let $f_k : \mathbb{R} \to \mathbb{R}$. We say $f_k$ converges to $f$ on $[a, b]$ pointwisely if $f_k(x) \to f(x)$ for every $x \in [a, b]$ as $k \to \infty$.
\end{shaded}

\begin{shaded}
\textbf{Theorem (pointwise convergence of Fourier series) \cite{math2111_notes}.} Let $c \in \mathbb{R}$, and suppose that a function $f : \mathbb{R} \to \mathbb{R}$ has the following properties:
\begin{enumerate}
	\item $f$ is $2 \pi$-periodic;
	\item $f$ is piecewise continuous on $[-\pi, \pi]$;
	\item $D^+ f(c)$ and $D^- f(c)$ exist.
\end{enumerate}
If $f$ is continuous at $c$, then the Fourier series of $f$ agrees with $f$ at $c$, i.e.,
$$ S f(c) = f(c) . $$
On the other hand, if $f$ has a jump discontinuity at $c$, then
$$ S f(c) = \frac{1}{2} \left( f(c^+) + f(c^-) \right) . $$
\end{shaded}

\begin{shaded}
\textbf{Definition (uniform convergence) \cite{math2111_notes}.} Let $f_k : \mathbb{R} \to \mathbb{R}$. We say $f_k$ converges to $f$ on $[a, b]$ uniformly if for every $\epsilon > 0$, there exists a $K$ (depends on $\epsilon$ only), such that
$$ \sup_{x \in [a, b]} | f_k(x) - f(x) | \leq \epsilon \quad \text{for all} \quad k \geq K . $$
Moreover, we say $\sum_{k = 1}^{\infty} f_k$ converges uniformly to $f$ if the partial sum $\tilde{f}_n = \sum_{k = 1}^{\infty} f_k$ converges uniformly to $f$ as $n \to \infty$

\textbf{Theorem.} If $f_k : \mathbb{R} \to \mathbb{R}$ is continuous on $[a, b]$ for all $n$ and if $f_k$ converge to $f$ uniformly on $[a, b]$, then $f$ is continuous on $[a, b]$.
\end{shaded}

\begin{shaded}
\textbf{Theorem (Weierstrass test) \cite{math2111_notes}.} Let $f_k : \mathbb{R} \to \mathbb{R}$ be a sequence of functions defined on $[a, b]$. Suppose that there exists a sequence of numbers $c_k$ such that
$$ |f_k(x)| \leq c_k \quad \text{for all} \quad x \in [a, b] $$
and $\sum_{k = 1}^\infty c_k$ exists. Then, $\sum_{k = 1}^\infty f_k(x)$ converges uniformly to a function $f$ on $[a, b]$.
\end{shaded}

\begin{shaded}
\textbf{Definition (norm convergence) \cite{math2111_notes}.} Consider the maximum norm $\lVert f \rVert_\infty = \sup_{x \in [a, b]} |f(x)|$. The definition of uniform convergence can be equivalently rewritten as: for every $\epsilon > 0$, there exists $K > 0$ such that
$$ \lVert f_k - f \rVert \leq \epsilon \quad \text{for all} \quad k \geq K , $$
or simply
$$ \lim_{k \to \infty} \lVert f_k - f \rVert = 0 . $$
\end{shaded}

\begin{shaded}
\textbf{Theorem (Parseval's theorem).} Let $f$ be $2 \pi$-periodic, bounded and $\int_{-\pi}^\pi f(x)^2 \ dx < + \infty$. Then, the Fourier series of $f$ converges to $f$ in the mean square sense, i.e.
$$ \int_{- \pi}^\pi (S_n f(x) - f(x))^2 \ dx \to 0 \quad \text{as} \quad k \to \infty . $$
That is to say, $\lVert S_n f - f \rVert_2 \to 0$ as $n \to \infty$. Moreover, the following \textbf{Parseval's identity} holds:
$$ \int_{-\pi}^\pi f(x)^2 \ dx = \lVert f \rVert_2^2 = \frac{\pi}{2} a_0^2 + \pi \sum_{k = 1}^{\infty} (a_k^2 + b_k^2) . $$
\end{shaded}

\subsection{Grad, div, curl}

\begin{shaded}
\textbf{Definition (gradient) \cite{thomas_calculus}.} The \textbf{gradient vector} (or \textbf{gradient}) of $f(x, y, z)$ is the vector
$$ \nabla f = \diffp{f}{x} \mathbf{i} + \diffp{f}{y} \mathbf{j} + \diffp{f}{z} \mathbf{k} . $$
The value of this gradient vector obtained by evaluating the partial derivatives at a point $P_0(x_0, y_0, z_0)$ is written
$$ \nabla f |_{P_0} \qquad \text{or} \qquad \nabla f(x_0, y_0, z_0) . $$
\end{shaded}

\begin{shaded}
\textbf{Definition (divergence) \cite{math2111_notes}.} If $\mathbf{F} = F_1 \mathbf{i} + F_2 \mathbf{j} + F_3 \mathbf{k}$, the \textbf{divergence} of $\mathbf{F}$ is the scalar field
\begin{align*}
\text{div} \ \mathbf{F} &= \nabla \cdot \mathbf{F} \\
&= \begin{pmatrix}[2] \diffp{}{x} & \diffp{}{y} & \diffp{}{z} \end{pmatrix}^{T}
\cdot \begin{pmatrix} F_1 \\ F_2 \\ F_3 \end{pmatrix} \\
&= \diffp{F_1}{x} + \diffp{F_2}{y} + \diffp{F_3}{z} .
\end{align*}
\end{shaded}

An \textbf{incompressible} liquid is one that has zero divergence.

\begin{shaded}
\textbf{Definition (curl) \cite{math2111_notes}.} If $\mathbf{F} = F_1 \mathbf{i} + F_2 \mathbf{j} + F_3 \mathbf{k}$, the \textbf{curl} of $\mathbf{F}$ is the vector field
\begin{align*}
\text{curl} \ \mathbf{F} &= \nabla \times \mathbf{F} \\
&= \begin{vmatrix}[1.5]
	\mathbf{i} & \mathbf{j} & \mathbf{k} \\
	\diffp{}{x} & \diffp{}{y} & \diffp{}{z} \\
	F_1 & F_2 & F_3
\end{vmatrix} \\
&= \left( \diffp{F_3}{y} - \diffp{F_2}{z} \right) \mathbf{i} + \left( \diffp{F_1}{z} - \diffp{F_3}{x} \right) \mathbf{j} + \left( \diffp{F_2}{x} - \diffp{F_1}{y} \right) \mathbf{k} .
\end{align*}
\end{shaded}

\subsection{Line integrals}

\textsc{Notation \cite{marsden_vector_calculus} \cite{thomas_calculus}.}
\begin{itemize}
	\item \textbf{Velocity vector} of a particle on a path
	$$\mathbf{c}(t) = x(t) \mathbf{i} + y(t) \mathbf{j} + z(t) \mathbf{k} \qquad \text{or} \qquad \mathbf{s} = x \mathbf{i} + y \mathbf{j} + z \mathbf{k} $$
is
$$ \mathbf{c}'(t) = \mathbf{v} = \diff{\mathbf{s}}{t} = \diff{x}{t} \mathbf{i} + \diff{y}{t} \mathbf{j} + \diff{z}{t} \mathbf{k} . $$
The \textbf{speed} of the particle is its magnitude: $\lVert \mathbf{c}'(t) \rVert = \lVert \mathbf{v} \rVert$.
	\item An \textbf{infinitesimal displacement} of a particle following the path $\mathbf{c}$ is
\begin{align*}
d \mathbf{s} &= dx \mathbf{i} + dy \mathbf{j} + dz \mathbf{k} \\
&= \left( \diff{x}{t} \mathbf{i} + \diff{y}{t} \mathbf{j} + \diff{z}{t} \mathbf{k} \right) \ dt \\
&= \mathbf{c}'(t) \ dt
\end{align*}
and its length
\begin{align*}
ds &= \sqrt{dx^2 + dy^2 + dz^2} \\
&= \sqrt{\left(\diff{x}{t}\right)^2 + \left(\diff{y}{t}\right)^2 + \left(\diff{z}{t}\right)^2} \ dt \\
&= \lVert \mathbf{c}'(t) \rVert \ dt
\end{align*}
is the \textbf{differential of arc length}.
	\item The \textbf{unit tangent vector} is the velocity vector divided by its magnitude, i.e.
$$ \mathbf{T} = \frac{\mathbf{v}}{\lVert \mathbf{v} \rVert} . $$
	\item If $\mathbf{c}(t)$ is a smooth curve, then the \textbf{principal unit normal} is
$$ \mathbf{N} = \frac{d\mathbf{T} / dt}{\lVert d\mathbf{T} / dt \rVert} , $$
where $\mathbf{T} = \mathbf{v} / \lVert \mathbf{v} \rVert$ is the unit tangent vector. (More information about the derivation of this formula can be found in reference \cite{thomas_calculus} of the bibliography.)
	\item $dA$ is shorthand for $dx dy$
\end{itemize}

\begin{shaded}
\textbf{Definition (line integral of a scalar field) \cite{marsden_vector_calculus}.} The \textbf{line integral of a scalar field}, or \textbf{path integral}, or the \textbf{integral of $f(x, y, z)$ along the path $\mathbf{c}$}, is defined when $\mathbf{c}: I = [a, b] \to \mathbb{R}^3$ is of class $C^1$ and when the composite function $t \mapsto f(x(t), y(t), z(t))$ is continuous on $I$. We define this integral by the equation
$$ \int_{\mathbf{c}} f \ ds = \int_a^b f(x(t), y(t), z(t)) \ \lVert \mathbf{c}'(t) \rVert \ dt . $$
Sometimes $\int_\mathbf{c} f \ ds$ is denoted
$$ \int_\mathbf{c} f(x, y, z) \ ds $$
or
$$ \int_\mathbf{c} f(\mathbf{c}(t)) \ \lVert \mathbf{c}'(t) \rVert \ dt . $$

If $\mathbf{c}(t)$ is only piecewise $C^1$ or $f(\mathbf{c}(t))$ is piecewise continuous, we define $\int_\mathbf{c} f \ ds$ by breaking $[a, b]$ into pieces over which $f(\mathbf{c}(t)) \ \lVert \mathbf{c}'(t) \rVert$ is continuous, and summing the integrals over the pieces.
\end{shaded}

\begin{shaded}
\textbf{Definition (line integral of a vector field) \cite{thomas_calculus}.} Let $\mathbf{F}$ be a vector field with continuous components defined along a smooth curve $C$ parametrised by $\mathbf{c}(t), a \leq t \leq b$. Then the \textbf{line integral of} $\mathbf{F}$ along $\mathbf{C}$ is
\begin{align*}
\int_C F_1 \ dx + F_2 \ dy + F_3 \ dz &= \int_a^b \left( F_1 \diff{x}{t} + F_2 \diff{y}{t} + F_3 \diff{z}{t} \right) \ dt \\
&= \int_C \mathbf{F} \cdot d \mathbf{s} \\
&= \int_C \left( \mathbf{F} \cdot \diff{\mathbf{s}}{s} \right) \ ds \\
&= \int_C \mathbf{F} \cdot \mathbf{T} \ ds \\
&= \int_a^b \mathbf{F}(\mathbf{c}(t)) \cdot \mathbf{c}'(t) \ dt
\end{align*}

\textbf{Definition (flow integral, circulation) \cite{thomas_calculus}.} If $\mathbf{c}(t)$ parametrises a smooth curve $C$ in the domain of a continuous velocity field $\mathbf{F}$, the \textbf{flow} along the curve from $A = \mathbf{c}(a) to B = \mathbf{r}(b)$ is
$$ \text{Flow} = \int_C \mathbf{F} \cdot \mathbf{T} \ ds . $$
The integral is called a \textbf{flow integral}. If the curve starts and ends at the same point, so that $A = B$, the flow is called the \textbf{circulation} around the curve.
\end{shaded}

\begin{shaded}
\textbf{Definition (flux in the plane) \cite{thomas_calculus}.} If $C$ is a smooth  simple closed curve in the domain of a continuous vector field $F = P(x, y) \mathbf{i} + Q(x, y) \mathbf{j}$ in the plane, and if $\mathbf{n}$ is the outward-pointing unit normal vector on $C$, the \textbf{flux} of $\mathbf{F}$ across $C$ is
\begin{align*}
\text{Flux of } \mathbf{F} \text{ across } C &= \int_C \mathbf{F} \cdot \mathbf{n} \ ds \\
&= \oint_C P \ dy - Q \ dx .
\end{align*}
The integral can be evaluated from any smooth parametrisation $x = g(t), y = h(t), a \leq t \leq b$, that traces $C$ anticlockwise exactly once.
\end{shaded}

\subsection{Definitions and theorems about line integrals}

\begin{shaded}
\textbf{Definition (arc length in $\mathbb{R}^n$) \cite{marsden_vector_calculus}.} Let $\mathbf{c}: [t_0, t_1] \to \mathbb{R}^n$ be a piecewise $C^1$ path. Its \textbf{length} is defined to be
$$ L(\mathbf{c} = \int_{t_0}^{t_1} \lVert \mathbf{c}'(t) \rVert \ dt . $$
The integrand is the square root of the sume of the squares of the coordinate functions of $\mathbf{c}'(t)$: If
$$ \mathbf{c}(t) = (x_1(t), x_2(t), ..., x_n(t)) , $$
then
$$ L(\mathbf{c}) = \sqrt{[x'_1(t)]^2 + [x'_2(t)]^2 + ... + [x'_n(t)]^2} \ dt . $$
\end{shaded}

\begin{shaded}
\textbf{Theorem (fundamental theorem of line integrals) \cite{thomas_calculus}.} Let $C$ be a smooth curve joining the point $A$ to the point $B$ in the plane or in space and parametrised by $\mathbf{c}(t)$. Let $f$ be a differentiable function with a continuous gradient vector $\mathbf{F} = \nabla f$ on a domain $D$ containing $C$. Then
$$ \int_C \mathbf{F} \cdot d \mathbf{r} = f(B) - f(A) . $$
\end{shaded}

\begin{shaded}
\textbf{Theorem (cross partials of gradient vector fields) \cite{math2111_notes}.} Let
$$ \mathbf{F} = (F_1, F_2, F_3) $$
be a gradient vector field whose components have continuous partial derivatives. Then the cross partials are equal:
$$ \diffp{F_1}{y} = \diffp{F_2}{x}, \quad \diffp{F_2}{z} = \diffp{F_3}{y}, \quad \diffp{F_3}{x} = \diffp{F_1}{z} . $$

Similarly, if the vector field in the plane
$$ \mathbf{F} = (F_1, F_2) $$
is a gradient vector field, then
$$ \diffp{F_1}{y} = \diffp{F_2}{x} . $$
\end{shaded}

\begin{theorem}{Green's theorem \cite{thomas_calculus}}{greens_theorem}
Let $C$ be a piecewise smooth, simple closed curve enclosing a region $R$ in the
plane. Let $\mathbf{F} = P \mathbf{i} + Q \mathbf{j}$ be a vector field with $P$ and $Q$ having continuous first partial derivatives in an open region containing $R$. Then:
\begin{itemize}
	\item \textsc{Circulation-curl or tangential form} --- the anticlockwise circulation of $\mathbf{F}$ around $C$ equals the double integral of $(\text{curl} \ \mathbf{F}) \cdot \mathbf{k}$ over $R$.
\begin{align*}
\oint_C \mathbf{F} \cdot \mathbf{T} &= \oint_C P \ dx + Q \ dy \\
&= \iint_R (\text{curl} \ \mathbf{F}) \cdot \mathbf{k} \ dA \\
&= \iint_R \left( \diffp{Q}{x} - \diffp{P}{y} \right) \ dx dy
\end{align*}
	\item \textsc{Flux-divergence or normal form} --- the outward flux of $\mathbf{F}$ across $C$ equals the double integral of $\text{div} \ \mathbf{F}$ over the region $R$ enclosed by $C$.
\begin{align*}
\oint_C \mathbf{F} \cdot \mathbf{n} &= \oint_C P \ dy - Q \ dx \\
&= \iint_R \text{div} \ \mathbf{F} \ dA \\
&= \iint_R \left( \diffp{P}{x} + \diffp{Q}{y} \right) \ dx dy
\end{align*}
\end{itemize}
\end{theorem}

\subsection{Miscellaneous}
\textbf{Even and odd functions.}
\begin{itemize}
	\item The product of two even functions is an even function.
	\item The product of two odd functions is an even function.
	\item The product of and even function and an odd function is an odd function.
\end{itemize}

This can be helpful when solving integrals of the form $\int f(x)g(x) \ dx$, where $f$ and $g$ are each even or odd.
% \section{Linear algebra}

\subsection{Matrices}

\subsubsection{Inverse matrices}

\begin{definition}{Inverse matrix \cite{math1141_notes}}{inverse_matrix}
A matrix $X$ is said to be the \textbf{inverse} of a matrix $A$ if both
$$ AX = I \qquad \text{and} \qquad XA = I , $$
where $I$ is the identity matrix of appropriate size.
\end{definition}

If a matrix $A$ has an inverse, then $A$ is said to be an \textbf{invertible matrix}. If a matrix $A$ is not an invertible matrix, then it is called a \textbf{singular} matrix.

\begin{lemma}{\cite{math1141_notes}}{}
All invertible matrices are square.
\end{lemma}

\begin{lemma}{}{singular_matrix}
A (square) matrix is singular if its determinant is $0$.
\end{lemma}

\begin{definition}{Right/left inverse \cite{math1141_notes}}{right_left_inverse}
\begin{itemize}
	\item An $n \times m$ matrix $X$ is said to be a \textbf{right inverse} of the $m \times n$ matrix $A$ if
$$ AX = I_m ; $$

	\item An $n \times m$ matrix $Y$ is said to be a \textbf{left inverse} of the $m \times n$ matrix $A$ if
$$ YA = I_n . $$
\end{itemize}
\end{definition}

\textbf{Finding the inverse of a matrix}. A matrix $A$ is invertible if and only if it can be reduced by elementary row operations to an identity matrix $I$ and if $(A|I)$ can be reduced to $(I|B)$, in which case, $B = A^{-1}$.

\begin{formula}{Inverse of a $2 \times 2$ matrix}{inverse_2b2_matrix}
$$
\begin{pmatrix}
	a & b \\
	c & d
\end{pmatrix}^{-1}
= \dfrac{1}{ad - bc}
\begin{pmatrix}
	d & -b \\
	-c & a
\end{pmatrix} , \quad
\text{provided } ad - bc \not = 0.
$$
\end{formula}

\subsubsection{Determinants}

\begin{definition}{Determinant of a $2 \times 2$ matrix}{determinant_2b2}
Given the matrix
$$ A = \begin{pmatrix}
a_{11} & a_{12} \\
a_{21} & a_{22}
\end{pmatrix} , $$
its \textbf{determinant} is
$$ \det (A) = a_{11} a_{22} - a_{12} a_{21} . $$
\end{definition}

\begin{definition}{Determinant}{determinant}
First, define the row $i$, column $j$ \textbf{minor} of a matrix $X$ to be the resulting matrix obtained by deleting row $i$ and column $j$ from $X$, denoted by $ \lvert X_{ij} \rvert$.

Then the \textbf{determinant} of an $n \times n$ matrix $A$ is
\begin{align*}
\lvert A \rvert &=
a_{11} \lvert A_{11} \rvert 
- a_{12} \lvert A_{12} \rvert
+ a_{13} \lvert A_{13} \rvert
- \dots +
(-1)^{1+n} a_{1n} \lvert A_{1n} \rvert \\
&= \sum_{k = 1}^n (-1)^{1+k} a_{1k} \lvert A_{1k} \rvert .
\end{align*}
\end{definition}

\textbf{Properties of determinants} \cite{math1141_notes}:
\begin{enumerate}
	\item $\det (A) = \det (A^T)$
	\item If any two rows (or any two columns) of $A$ are interchanged, then the sign of the determinant is reversed.
	\item If a matrix contains a zero row or column then its determinant is zero.
	\item If a row (or column) of $A$ is multiplied by a scalar, then the value of $\det A$ is multiplied by the same scalar.
	\item If any column of a matrix is a multiple of another column of the matrix (or any row is a multiple of another row), then the value of $\det (A)$ is zero.
	\item If a multiple of one row (or column) is added to another row (or column), then the value of the determinant is not changed.
	\item If $A$ and $B$ are square matrices such that the product $AB$ exists, then
$$ \det (AB) = \det (A) \det (B) . $$
\end{enumerate}

\subsection{Groups and fields}

\begin{definition}{Group \cite{math2601_notes}}{group}
	A \textbf{group} is a a set, $G$, together with an operation $*$ (called the \textbf{group law} of $G$) that combines any two elements $a$ and $b$ to form another element, denoted $a * b$ or $ab$. To qualify as a group, the set and operation, $(G, *)$, must satisfy four requirements known as the \textbf{group axioms}:
	\begin{itemize}
		\item \textbf{Closure.} For all $a, b$ in $G$, the result of the operation, $a * b$, is also in $G$.
		\item \textbf{Associativity.} For all $a, b$ and $c$ in $G, (a * b) * c = a * (b * c)$.
		\item \textbf{Identity element.} There exists an element $e$ in $G$ such that, for every element $a$ in $G$, the equation $e * a = a * e = a$ holds. Such an element is unique, and thus one speaks of \textit{the} identity element.
		\item \textbf{Inverse element.} For each $a$ in $G$, there exists an element $b$ in $G$, commonly denoted $a^{-1}$ (or $-a$, if the operation is denoted ``$+$''), such that $a * b = b * a = e$, where $e$ is the identity element.
	\end{itemize}
	If $G$ is a finite set then the \textbf{order} of $G$ is $\lvert G \rvert$, the number of elements in $G$.
\end{definition}

A group $G$ is \textbf{abelian} if the operation satisfies the \textbf{commutative law}
$$ a * b = b * a \qquad \text{for all} \ a, b \in G . $$

We often write the \textbf{cyclic group of order $m$} as
$$ C_m = \langle a : a^m = e \rangle , $$
and say it is \textit{generated} by $a$.

\begin{definition}{Subgroup \cite{math2601_notes}}{subgroup}
Let $(G, *)$ be a group and $H$ a non-empty subset of $G$. If $H$ is a group under the restriction of $*$ to $H$, we call it a \textbf{subgroup} of $G$. We write this as $H \leq G$ and say $H$ \textit{inherits} the group structure from $G$.
\end{definition}

\begin{lemma}{\cite{math2601_notes}}{subgroup}
Let $(G, *)$ be a group and $H$ a non-empty subset of $G$. Then $H$ is a subgroup of $G$ if and only if
\begin{itemize}
	\item for all $a, b \in H, \ a * b \in H$;
	\item for all $a \in H, \ a^{-1} \in H$,
\end{itemize}
i.e. $H$ is closed under $*$ and $^{-1}$.
\end{lemma}

\begin{definition}{Field \cite{math2601_notes}}{field}
A \textbf{field}, $(\mathbb{F}, +, \times)$ is a set $\mathbb{F}$ with two binary operations on it --- addition ($+$) and multiplication ($\times$) --- where
\begin{enumerate}
	\item $(\mathbb{F}, +)$ is an abelian group,
	\item $\mathbb{F}^{*} = \mathbb{F} \setminus \{0\}$ (where 0 is the additive identity) is an abelian group under multiplication,
	\item the distributive laws $a \times (b + c) = a \times b + a \times c$ and $(a + b) \times c = a \times c + b \times c$ hold.
\end{enumerate}
\end{definition}

\begin{lemma}{\cite{math2601_notes}}{}
	Let $\mathbb{F}$ be a field and $a, b, c \in \mathbb{F}$. Then
	\begin{enumerate}[a)]
		\item $a0 = 0$
		\item $a(-b) = -(ab)$
		\item $a(b - c) = ab - ac$
		\item if $ab = 0$ then either $a = 0$ or $b = 0$.
	\end{enumerate}
\end{lemma}

\begin{definition}{Subfield \cite{math2601_notes}}{subfield}
If $(\mathbb{F}, +, \times)$ is a field and $\mathbb{E} \subset \mathbb{F}$ is also a field under the same operations (restricted to $\mathbb{E}$), then $(\mathbb{E}, +, \times)$ is a \textbf{subfield} of $(\mathbb{F}, +, \times)$, usually written $\mathbb{E} \leq \mathbb{F}$.
\end{definition}

\begin{lemma}{\cite{math2601_notes}}{subfield}
Let $\mathbb{E} \not = \{0\}$ be a non-empty subset of field $\mathbb{F}$. Then $\mathbb{E}$ is a subfield of $\mathbb{F}$ if and only if for all $a, b \in \mathbb{E}$:
$$ a + b \in \mathbb{E}, \quad -b \in \mathbb{E}, \quad a \times b \in \mathbb{E}, \quad b^{-1} \in \mathbb{E} \text{ if } b \not = 0 . $$

\begin{proof}
The distributive laws are inherited from $\mathbb{F}$ to $\mathbb{E}$. The rest of the proof follows from applying the subgroup lemma \ref{lem:subgroup} to both $(\mathbb{E}, +)$ and $(\mathbb{E}^*, \times)$.
\end{proof}
\end{lemma}

\begin{definition}{General linear group \cite{math2601_notes}}{general_linear_group}
Let $n \geq 1$ be any integer. The \textbf{general linear group}, $\operatorname{GL}(n, \mathbb{F})$ is the set of invertible $n \times n$ matrices over field $\mathbb{F}$ under matrix multiplication, and is non-abelian if $n > 1$.
\end{definition}

Subgroups of $\operatorname{GL}(n, \mathbb{F})$ include
\begin{itemize}
	\item the \textbf{special linear groups} $\operatorname{SL}(n, \mathbb{R})$ and $\operatorname{SL}(n, \mathbb{C})$ of matrices with determinant 1;
	\item $\operatorname{O}(n) \leq \operatorname{GL}(n, \mathbb{R})$, the group of \textbf{orthogonal matrices};
	\item $\operatorname{SO}(n) = \operatorname{O}(n) \cap \operatorname{SL}(n, \mathbb{R})$, the group of \textbf{special orthogonal matrices}.
\end{itemize}

\begin{definition}{Group homomorphism \cite{math2601_notes}}{group_homomorphism}
Let $(G, *)$ and $(H, \circ)$ be two groups. A \textbf{(group) homomorphism} from $G$ to $H$ is a map $\phi : G \to H$ that respects the two operations, that is where
$$ \phi (a * b) = \phi (a) \circ \phi (b) \qquad \text{for all } a, b \in G . $$
A bijective homomorphism $\phi : G \to H$ is called an \textbf{isomorphism}: the groups are then said to be \textbf{isomorphic}.
\end{definition}

\begin{lemma}{\cite{math2601_notes}}{homomorphism}
Let $(G, *)$ and $(H, \circ)$ be two groups and $\phi$ a homomorphism between them. Then
\begin{itemize}
	\item $\phi$ maps the identity of $G$ to the identity of $H$;
	\item $\phi$	 maps inverse to inverses, i.e. $\phi (a^{-1}) = (\phi(a))^{-1}$ for all $a \in G$;
	\item if $\phi$ is an isomorphism from $G$ to $H$ then $\phi^{-1}$ is an isomorphism from $H$ to $G$.
\end{itemize}
\end{lemma}

\begin{definition}{Kernel and image \cite{math2601_notes}}{kernel_and_image}
Let $\phi : G \to H$ be a group homomorphism, with $e'$ the identity of $H$.

The \textbf{kernel} of $\phi$ is the set
$$ \ker (\phi) = \{ g \in G : \phi (g) = e' \} . $$

The \textbf{image} of $\phi$ is the set
$$ \operatorname{im} (\phi) = \{ h \in H : h = \phi (g), \text{ some } g \in G \} . $$
\end{definition}

\begin{lemma}{\cite{math2601_notes}}{kernel_image_subgroups}
For $\phi : G \to H$, a group homomorphism, $\ker \phi \leq G$ and $\operatorname{im} \phi \leq H$.

\begin{proof}
Let $e$ be the identity of $G$. From lemma \ref{lem:homomorphism}, $e \in \ker \phi$, so the kernel is non-empty.

If $a, b \in \ker \phi$ then
$$ \phi (a * b) = \phi (a) \circ \phi (b) = e' \circ e' = e' , $$
so $a * b \in \ker \phi$.

If $a \in \ker \phi$ then from lemma \ref{lem:homomorphism},
$$ \phi (a^{-1}) = (\phi (a))^{-1} = (e')^{-1} = e' , $$
and so $a^{-1} \in \ker \phi$.

Thus by the subgroup lemma \ref{lem:homomorphism}, $\ker \phi \leq G$.

\textit{The proof that $\operatorname{im} \phi \leq H$ is omitted.}
\end{proof}
\end{lemma}

\begin{lemma}{\cite{math2601_notes}}{}
A group homomorphism $\phi : G \to H$ is one-to-one if and only if $\ker \phi = \{e\}$, with $e$ the identity of $G$; if $\phi$ is one-to-one then $\operatorname{im} \phi$ is isomorphic to $G$.

\begin{proof}
From lemma \ref{lem:kernel_image_subgroups}, $e \in \ker \phi$, and if $\phi$ is one-to-one, $e$ is the only element that maps to $e'$, the identity of $H$.

Conversely, suppose $\ker \phi = \{e\}$ and $\phi (a) = \phi (b)$, where $a, b \in G$. Then
$$ \phi (a * b^{-1}) = \phi (a) \circ (\phi (b))^{-1} = e' $$
and so $a * b^{-1} = e$ and $a = b$.

If $\phi$ is one-to-one, it is a bijection from $G$ to $\operatorname{im} \phi$, and hence an isomorphism.
\end{proof}
\end{lemma}

A common use of group homomorphisms is to look for a homomorphism $\phi : G \to \operatorname{GL}(n, \mathbb{F})$ for some $n$ and some field $F$. The group $\operatorname{im} (\phi)$ is called a \textbf{(linear) representation of $G$ on $\mathbb{F}^n$}.

If $\phi$ is one-to-one (so every element maps to a distinct matrix), we call the representation \textbf{faithful}.

\subsection{Vector spaces}

\begin{definition}{Vector space \cite{math2601_notes}}{vector_space}
Let $\mathbb{F}$ be a field. A \textbf{vector space over the field} $\mathbb{F}$ consists of an abelian group $(V, +)$ plus a function from $\mathbb{F} \times V$ to $V$ called \textbf{scalar multiplication} and written $\alpha \mathbf{v}$, where
\begin{itemize}
	\item $\alpha (\beta \mathbf{v}) = (\alpha \beta) \mathbf{v}$ for all $\alpha, \beta \in \mathbb{F}$ for all $\mathbf{v} \in V$.
	\item $1 \mathbf{v} = \mathbf{v}$ for all $\mathbf{v} \in V$.
	\item $\alpha (\mathbf{u} + \mathbf{v}) = \alpha \mathbf{u} + \alpha \mathbf{v}$ for all $\alpha \in \mathbb{F}$ and for all $\mathbf{u}, \mathbf{v} \in V$.
	\item $(\alpha + \beta) \mathbf{u} = \alpha \mathbf{u} + \beta \mathbf{u}$ for all $\alpha, \beta \in \mathbb{F}$ for all $\mathbf{u} \in V$.
\end{itemize}
\end{definition}

\begin{lemma}{\cite{math2601_notes}}{}
Let $V$ be a vector space over field $\mathbb{F}$. For all $\mathbf{v}, \mathbf{w} \in V$ and $\lambda \in \mathbb{F}$:
\begin{itemize}
	\item $0 \mathbf{v} = \mathbf{0}$ and $\lambda \mathbf{0} = \mathbf{0}$;
	\item $(-1) \mathbf{v} = - \mathbf{v}$;\
	\item $\lambda \mathbf{v} = \mathbf{0}$ implies either $\lambda = 0$ or $\mathbf{v} = \mathbf{0}$;
	\item if $\lambda \mathbf{v} = \lambda \mathbf{w}$ and $\lambda \not = 0$ then $\mathbf{v} = \mathbf{w}$.
\end{itemize}
\end{lemma}

\subsubsection{Subspaces}

\begin{definition}{Linear (vector) subspace \cite{math2601_notes}}{linear_subspace}
If $V$ is a vector space over $\mathbb{F}$ and $U \subseteq V$, then $U$ is a \textbf{subspace} of $V$, written $U \leq V$, if it is a vector space over $\mathbb{F}$ with the same addition and scalar multiplication as in $V$.
\end{definition}

\begin{lemma}{(Linear) subspace test}{subspace_test}
Suppose $V$ is a vector space over the field $\mathbb{F}$ and $U$ is a non-empty subset of $V$. Then $U$ is a subspace of $V$ if and only if for all $\mathbf{u}, \mathbf{v}, \in U$ and $\alpha \in \mathbb{F}$, it holds that $\alpha \mathbf{u} + \mathbf{v} \in U$.

\begin{proof}
Most of the axioms will be simply inherited from $V$: all we really need to prove is closure and that the zero vector is in $U$.

But setting $\alpha = 1$ proves $U$ is closed under addition.

With $\alpha = -1$ and $\mathbf{v} = \mathbf{u}$ we get $\mathbf{0} \in U$.

Hence we can set $\mathbf{v} = \mathbf{0}$ to get closure under scalar multiplication.

``The rest is easy.''
\end{proof}
\end{lemma}

Examples:
\begin{itemize}
	\item Every vector space has ${\mathbf{0}}$ (the \textbf{trivial subspace}) and itself as subspaces.
	\item Polynomials and functions: $\mathcal{P}_n(\mathbb{R})$ is a subspace of $\mathcal{P}(\mathbb{R})$, which is a subspace of $\mathcal{R}[\mathbb{R}]$.
\end{itemize}

\subsubsection{Linear combinations, spans and independence}

\begin{definition}{Linear combination \cite{math2601_notes}}{linear_combination}
Let $V$ be a vector space over $\mathbb{F}$. A \textbf{(finite) linear combination} of vectors $\mathbf{v}_1 + \mathbf{v}_2, \ldots, \mathbf{v}_n$ in $V$ is any vector which can be expressed
$$ \alpha_1 \mathbf{v}_1 + \alpha_2 \mathbf{v}_2 + \dots + \alpha_n \mathbf{v}_n , $$
where the $\alpha_k$ are scalars.
\end{definition}

\begin{definition}{Span \cite{math2601_notes}}{span}
If $S$ is a subset of $V$, then the \textbf{span} of $S$ is
$$ \operatorname{span}(S) = \{ \text{all finite linear combinations of vectors in } S \} . $$
If $\operatorname{span}(S) = V$, we say that $S$ \textbf{spans} $V$, or \textbf{is a spanning set for} $V$.
\end{definition}

\begin{lemma}{\cite{math2601_notes}}{}
If $S$ is a non-empty subset of a vector space $V$, then $\operatorname{span}(S)$ is a subspace of $V$.

\begin{proof}
Since $S \subseteq \operatorname{span}(S)$, we know that $\operatorname{span}(S)$ is non-empty.

But if $\mathbf{v}, \mathbf{w} \in \operatorname{span}(S)$, then $\lambda \mathbf{v} + \mathbf{w}$ is a linear combination of two linear combinations of elements of $S$, and so is a linear combination of elements of $S$.

Thus $\lambda \mathbf{v} + \mathbf{w} \in \operatorname{span}(S)$ and $\operatorname{span}(S) \leq V$.
\end{proof}
\end{lemma}

\begin{definition}{Linear independence \cite{math2601_notes}}{linear_independence}
A subset $S$ of a vector space $V$ is \textbf{linearly independent} if for all vectors $\mathbf{v}_1, \mathbf{v}_2, \ldots, \mathbf{v}_n$ in $S$ (with $n \geq 1$) the equation
$$ \alpha_1 \mathbf{v}_1 + \alpha_2 \mathbf{v}_2 + \dots + \alpha_n \mathbf{v}_n = \mathbf{0} , $$
with $\alpha_i \in \mathbb{F}$, implies $\alpha_i = 0$ for all $i = 1 \ldots n$.

A set which is not linearly independent is said to be \textbf{linearly dependent}.
\end{definition}

\begin{lemma}{Linear dependence \cite{math2601_notes}}{linear_dependence}
If $S = \{ \mathbf{v}_1, \ldots, \mathbf{v}_n \}$ is a linearly dependent set in $V$, then there is an $i$ where $2 \leq i \leq n$ such that
$$ \mathbf{v}_i = \sum_{j = 1}^{i - 1} \beta_j \mathbf{v}_j , $$
where $\beta_j \in \mathbb{F}$ for $j = 1 \ldots i - 1$.

In other words, in an ordered linearly dependent set, at least one vector is a linear combination of its predecessors.
\end{lemma}

\subsubsection{Bases}

\begin{definition}{Basis \cite{math2601_notes}}{basis}
Let $S \subseteq V$. The set $S$ is a \textbf{basis} for $V$ over $\mathbb{F}$ if and only $V = \operatorname{span}(S)$, and $S$ is a linearly independent set.
\end{definition}

\subsubsection{Dimension}

\begin{theorem}{Existence of finite basis \cite{math2601_notes}}{existence_finite_basis}
Let $V$ be a vector space over $\mathbb{F}$ and $S$ a finite set that spans $V$. Then $S$ contains a finite basis for $V$.
\end{theorem}

\begin{lemma}{Exchange Lemma \cite{math2601_notes}}{exchange}
Suppose that $S$ is a finite spanning set for $V$ and that $T$ is a (finite) linearly independent subset of $V$ with $\lvert T \rvert \leq \lvert S \rvert$. Then there is a spanning set $S'$ of $V$ such that
$$ T \subseteq S' \quad \text{and} \quad \lvert S' \rvert = \lvert S \rvert . $$
\end{lemma}

\begin{corollary}{\cite{math2601_notes}}{size_independent_spanning}
If $S$ is a finite spanning set for a vector space $V$, and $T$ is a linearly independent subset of $V$, then $T$ is finite and $\lvert T \rvert \leq \lvert S \rvert$.

In other words, \textit{independent sets are no larger than spanning sets (if there is a finite spanning set}.

\begin{proof}
Suppose that $\lvert T \rvert > \lvert S \rvert$, and consider any subset $T_0$ of $T$ with exactly $\lvert S \rvert$ elements.

Let $\mathbf{v}$ be an element of $T \setminus T_0$.

Applying the Exchange Lemma (lemma \ref{lem:exchange}) to $T_0$ and $S$ gives a spanning set $S'$ for which
$$ T_0 \subseteq S' \quad \text{and} \quad \lvert S' \rvert = \lvert T_0 \rvert . $$
But this implies that $T_0 = S'$ and so $T_0$ is a spanning set for $V$.

But then $v$ is in $\operatorname{span}(T_0)$, implying that $T$ is linearly dependent. This is a contradiction, so in fact $\lvert T \rvert \leq \lvert S \rvert$.
\end{proof}
\end{corollary}

\begin{theorem}{\cite{math2601_notes}}{}
Let $V$ be a vector space over $\mathbb{F}$ with a finite spanning set and $T$ a linearly independent subset of $V$. Then there is a basis $B$ of $V$ which contains $T$.

In other words, \textit{any linearly independent set can be extended to a basis (if there is a finite spanning set)}.

\begin{proof}
By the Exchange Lemma (lemma \ref{lem:exchange}), there is a finite spanning set $S'$ with $T \subseteq S'$.

Consider the (non-empty) collection of all the linearly independent subsets of $S'$ that contain $T$.

This collection must be finite (as $S'$ is), so there must be least one that is maximal; let $B$ be one such.

But for every $\mathbf{v} \in S'$, the set $B \cup \{\mathbf{v}\}$ is linearly dependent (by maximality of $B$) and so $\mathbf{v} \in \operatorname{span}(B)$. Therefore
$$ V = \operatorname{span}(S') \subseteq \operatorname{span}B) . $$
So $B$, which by construction contains $T$, is a basis for $V$.
\end{proof}
\end{theorem}

\begin{theorem}{Size of all bases \cite{math2601_notes}}{size_all_bases}
If vector space $V$ admits a finite spanning set, it admits a finite basis and all bases contain the same number of elements.

\begin{proof}
By theorem \ref{th:existence_finite_basis}, $V$ must contain at least one finite basis, $\mathcal{B}_1$, say.

Suppose that $\mathcal{B}_2$ is a second basis for $V$.

Since $\mathcal{B}_2$ is linearly independent and $\mathcal{B}_1$ spans $V$, corollary \ref{cor:size_independent_spanning} says that $\lvert \mathcal{B}_2 \rvert \leq \lvert \mathcal{B}_1 \rvert$, so $\mathcal{B}_2$ is also finite.

But then similarly, $\lvert \mathcal{B}_1 \rvert \leq \lvert \mathcal{B}_2 \rvert$, and this completes the proof.
\end{proof}
\end{theorem}

\begin{definition}{Dimension \cite{math2601_notes}}{dimension}
The \textbf{dimension} of a vector space $V$ is the size of a basis if $V$ has a finite basis or infinity otherwise.

The notation is $\dim (V) = n$ or $\dim(V) = \infty$.

If we need to emphasis the field, we use the notation $\dim_\mathbb{F}(V)$.
\end{definition}

\begin{lemma}{\cite{math2601_notes}}{}
Let $V$ be a finite dimensional vector space and suppose $\dim (V) = n$.
\begin{enumerate}[a)]
	\item The number of elements in any spanning set is at least $n$.
	\item The number of elements in any independent set is no more than $n$.
	\item If $\operatorname{span} (S) = V$ and $\lvert S \rvert = n$ then $S$ is a basis.
	\item If $S$ is a linearly independent set and $\lvert S \rvert = n$ then $S$ is a basis.
\end{enumerate}
\end{lemma}

\begin{theorem}{\cite{math2601_notes}}{basis_written_uniquely}
	Let $V$ be a finite dimensional vector space over $\mathbb{F}$.

	Then $\mathcal{B} = \{ \mathbf{v}_1, \ldots, \mathbf{v}_n \}$ is a basis for $V$ if and only if every $\mathbf{x} \in V$ can be written uniquely as
	$$ \mathbf{x} = \sum_{i = 1}^{n} \alpha_i \mathbf{v}_i, \quad \alpha_i \in \mathbb{F} $$

	\begin{proof}
		Suppose that $\lvert S \rvert = q$; we shall prove the result by induction on $\lvert T \rvert = p \leq q$.

		For the base case $p = 0$, simply choose $S' = S$.

		Now suppose that the result is true for some specific non-negative integer $p < q$, and consider a linearly independent set
		$$ T = \{ \mathbf{u}_1, \mathbf{u}_2, \ldots, \mathbf{u}_{p+1} \} . $$

		As $\{ \mathbf{u}_1, \mathbf{u}_2, \ldots, \mathbf{u}_{p} \}$ is also linearly independent, there is a set
		$$ S_1 = \{ \mathbf{u}_1, \mathbf{u}_2, \ldots, \mathbf{u}_{p}, \mathbf{v}_{p+1}, \ldots, \mathbf{v}_q \} $$
		that spans $V$ by our inductive hypothesis. Thus
		$$ S_2 = \{ \mathbf{u}_1, \mathbf{u}_2, \ldots, \mathbf{u}_{p}, \mathbf{u}_{p+1}, \mathbf{v}_{p+1}, \ldots, \mathbf{v}_q \} $$
		is a linearly \textit{dependent} spanning set of $V$ (by the properties of dependent sets).

		Now for $j = 1 \ldots q - p$, define the set
		$$ S_2 = \{ \mathbf{u}_1, \mathbf{u}_2, \ldots, \mathbf{u}_{p}, \mathbf{u}_{p+1}, \mathbf{v}_{p+1}, \ldots, \mathbf{v}_{p+j} \} . $$
		Let $T_k$ be the first of these sets that is linearly dependent. Then from our previous results we have
			\begin{align*}
				\mathbf{v}_{p+k} &\in \operatorname{span} (T_k \setminus \{ \mathbf{v}_{p+k} \}) \\
				\implies \mathbf{v}_{p+k} &\in \operatorname{span} (S_2 \setminus \{ \mathbf{v}_{p+k} \}) \\
				\\
				\implies \operatorname{span} (S_2 \setminus \{ \mathbf{v}_{p+k} \}) &= \operatorname{span} (S_2) \\
				&= V .
			\end{align*}
		Now take $S' = S_2 \setminus \{ \mathbf{v}_{p+k} \}$. Then we have shown that $S'$ spans $V$. Also, $T \subseteq S'$.

		Moreoever, $S'$ has one element fewer than $S_2$, and as $S_2$ has one element more than $S_1$, we have that $\lvert S' \rvert = \lvert S_1 \rvert = q$. This completes the proof of the inductive step.

		The result follows by induction.
	\end{proof}
\end{theorem}

\subsubsection{Coordinates}

\begin{definition}{\cite{math2601_notes}}{}
	Suppose $V$ is a vector space of dimension $n$ over $\mathbb{F}$ and suppose $\mathcal{B} = \{ \mathbf{v}_1, \ldots, \mathbf{v}_n \}$ is an ordered basis for $V$ over $\mathbb{F}$.

	If $\mathbf{v} \in V$ then $\mathbf{v} = \sum_{i = 1}^{n} \alpha_i \mathbf{v}_i$, with $\alpha_i$ unique, by theorem \ref{th:basis_written_uniquely}.

	We call
	$$ \boldsymbol{\alpha} = 
		\begin{pmatrix}
			\alpha_1 \\
			\vdots \\
			\alpha_n \\
		\end{pmatrix}
	$$
	the \textbf{coordinate vector} of $\mathbf{v}$ with respect to $\mathcal{B}$, and refer to the $\alpha_i$ as the \textbf{coordinates} of $\mathbf{v}$.
\end{definition}

A useful notation is
$$ \boldsymbol{\alpha} = [\mathbf{v}]_\mathcal{B} \quad
	\text{if} \quad \mathbf{v} = \sum_{i = 1}^{n} \alpha_i \mathbf{v}_i . $$

\subsubsection{Sums and Direct Sums}

\begin{definition}{Sum of subspaces \cite{math2601_notes}}{sum_subspaces}
	The \textbf{sum} $S + T$ of two subspaces is defined as
	$$ S + T = \{ \mathbf{a} + \mathbf{b} : \mathbf{a} \in S, \mathbf{b} \in T \} . $$
	If $S \cap T = \{ \mathbf{0} \}$, we call the sum a \textbf{direct sum} and denote it as $S \oplus T$.
\end{definition}

\begin{theorem}{\cite{math2601_notes}}{}
	Suppose $S$ and $T$ are finite dimensional subspaces of vector space $V$. Then
	$$ \dim (S) + \dim (T) = \dim (S + T) + \dim (S \cap T) . $$
\end{theorem}

\begin{corollary}{\cite{math2601_notes}}{}
	For a direct sum of finite dimensional spaces,
	$$ \dim (S) + \dim (T) = \dim (S \oplus T) . $$
\end{corollary}

\begin{lemma}{\cite{math2601_notes}}{}
	Let $V$ be a finite dimensional vector space and $X \leq V$. Then there is a subspace $Y$ for which $V = X \oplus Y$.
\end{lemma}

\begin{definition}{External direct sum \cite{math2601_notes}}{external_direct_sum}
	Let $X$ and $Y$ be two vector									 spaces over the same field $\mathbb{F}$. The Cartesian product $X \times Y$ can be made into a vector space over $\mathbb{F}$ with the obvious definitions
	$$
		(\mathbf{x}_1, \mathbf{y}_1) + (\mathbf{x}_2, \mathbf{y}_2)
		= (\mathbf{x}_1 + \mathbf{x}_2, \mathbf{y}_1 + \mathbf{y}_2)
	$$
	and
	$$ \lambda (\mathbf{x}_1, \mathbf{y}_1) = (\lambda \mathbf{x}_1, \lambda \mathbf{y}_1) . $$
	With this structure we call the Cartesian product the \textbf{(external) direct sum} of $X$ and $Y$ --- $X \oplus Y$.
\end{definition}

\subsection{Linear transformations}

\begin{definition}{Linear transformation \cite{math2601_notes}}{linear_transformation}
	Suppose $V$ and $W$ are vector spaces over the field $\mathbb{F}$. A function $T : V \to W$ is a \textbf{linear transformation} or a \textbf{linear map} (or simply \textbf{linear}) if
	\begin{itemize}
		\item $T(\mathbf{u} + \mathbf{v}) = T(\mathbf{u}) + T(\mathbf{v})$, and
		\item $T(\lambda \mathbf{v}) = \lambda T(\mathbf{u})$,
	\end{itemize}
	for all $\mathbf{u}, \mathbf{v} \in V$ and for all $\lambda \in \mathbb{F}$.
\end{definition}

Note that we only define linear maps between vector spaces over the same field.

\begin{lemma}{\cite{math2601_notes}}{}
	Let $V$ and $W$ be vector spaces over field $\mathbb{F}$.
	\begin{enumerate}[a)]
		\item The identity map $\operatorname{id} : V \to V$ defined by $\operatorname{id} (\mathbf{v}) = \mathbf{v}$ is linear.
		\item If $T : V \to W$ is linear then $T(\mathbf{0}) = \mathbf{0}$ and $T(-\mathbf{v}) = -T (\mathbf{v})$.
	\end{enumerate}
\end{lemma}

\begin{lemma}{Linearity test \cite{math2601_notes}}{linearity_test}
	A function $T : V \to W$ between vector spaces over the same field $\mathbb{F}$ is linear if and only if
	$$ T(\lambda \mathbf{u} + \mathbf{v}) = \lambda T(\mathbf{u}) + T(\mathbf{v}) $$
	for all $\lambda \in \mathbb{F}$ and $\mathbf{u},\mathbf{v} \in V$.
\end{lemma}

\begin{theorem}{\cite{math2601_notes}}{}
	Let $V$ and $W$ be two vector spaces over field $\mathbb{F}$. The set $L(V, W)$ of all linear transformations from $V$ to $W$ is a vector space under the operations
	$$
		(S + T)(\mathbf{v}) = S(\mathbf{v}) + T(\mathbf{v}), \qquad
		(\lambda S)(\mathbf{v}) = \lambda S(\mathbf{v}) .
	$$
\end{theorem}

\begin{lemma}{\cite{math2601_notes}}{}
	Let $T : V \to W$ and $S : W \to X$ be linear maps between vector spaces. Then $S \circ T : V \to X$ is also linear.
\end{lemma}

\begin{lemma}{\cite{math2601_notes}}{}
	Let $T : V \to W$ be an invertible linear map between two vector spaces over field $\mathbb{F}$. Then $T^{-1} : W \to V$ is linear.
\end{lemma}

\begin{theorem}{\cite{math2601_notes}}{}
	The invertible linear maps in $L(V, V)$ form a group under composition.
\end{theorem}

\begin{lemma}{Linearity of taking coordinates \cite{math2601_notes}}{linearity_taking_coordinates}
	Let $V$ be a (finite-dimensional) vector space over $\mathbb{F}$ with a basis $\mathcal{B} = \{ \mathbf{v}_1, \ldots, \mathbf{v}_p \}$.

	Then the function $S : V \to \mathbb{F}^p$ defined by $S(x) = [x]_{\mathcal{B}}$ is linear.
\end{lemma}

\subsubsection{Kernel and image}

\begin{definition}{Kernel and image \cite{math2601_notes}}{kernel_image}
	Let $T : V \to W$ be a linear transformation.
	
	The \textbf{kernel} (or \textbf{nullspace}) of $T$ is the set
	$$ \operatorname{ker}(T) = \{ \mathbf{v} \in V : T(\mathbf{v}) = \mathbf{0} \} . $$

	If $U \leq V$ then the \textbf{image} of $U$ is the set
	$$ T(U) = \{ T(\mathbf{u}) : \mathbf{u} \in U \} . $$

	We also define the image of $T$ (or \textbf{range} $T$), $\operatorname{im}(T)$ as the image of all of $V$; that is, $\operatorname{im}(T) = T(V)$.
\end{definition}

\begin{theorem}{\cite{math2601_notes}}{}
	Let $T : V \to W$ be a linear transformation between vector spaces over $\mathbb{F}$ and $U \leq V$. Then
	\begin{enumerate}[a)]
		\item $\operatorname{ker}(T)$ is a subspace of $V$.
		\item $T(U)$ is a subspace of $W$, and so $\operatorname{im}(T) \leq W$.
		\item If $U$ is finite-dimensional, so is $T(U)$, so if $V$ is finie dimensional, so is $\operatorname{im}(T)$.
	\end{enumerate}
\end{theorem}

\begin{definition}{Nullity and rank \cite{math2601_notes}}{nullity_rank}
	If $T$ is a linear transformation, then the dimension of the kernel of $T$ is called the \textbf{nullity} of $T$, and the dimension of its image is called the \textbf{rank} of $T$.
\end{definition}

\begin{lemma}{\cite{math2601_notes}}{}
	A linear map $T : V \to W$ is one-to-one if and only if $\operatorname{nullity}(T) = 0$.
\end{lemma}

\begin{theorem}{Rank-Nullity Theorem \cite{math2601_notes}}{rank_nullity}
	If $V$ is a finite dimensional vectors space over $\mathbb{F}$ and $T : V \to W$ is linear then
	$$ \operatorname{rank}(T) + \operatorname{nullity}(T) = \dim (V) . $$
\end{theorem}

\begin{theorem}{}{}
	Let $V, W$ be vector spaces over $\mathbb{F}$ with $\dim (V) = \dim (W)$ finite and $T : V \to W$ be linear. The following are equivalent:
	\begin{enumerate}[a)]
		\item $T$ is invertible (bijective).
		\item $T$ is one-to-one (injective), i.e. $\operatorname{nullity}(T) = 0$.
		\item $T$ is onto (surjective), i.e. $\operatorname{rank}(T) = \dim (V)$.
	\end{enumerate}
\end{theorem}

\begin{definition}{Isomorphism (vector spaces) \cite{math2601_notes}}{isomorphism_vector_spaces}
	An invertible linear map $T : V \to W$ is called an \textbf{isomorphism} of the vector spaces $V$ and $W$.
	
	If there is an isomorphism between $V$ and $W$ we call the two spaces \textbf{isomorphic}.
\end{definition}

Note that this isomorphism isn't strictly the same as for groups.

The isomorphism is an equivalence relation on vector spaces:
\begin{itemize}
	\item \textbf{Reflexive}: The identity map $\operatorname{id} : V \to V$ is an isomorphism from $V$ to itself.
	\item \textbf{Symmetric}: If $T : V \to W$ is an isomorphism, then $T^{-1}$ is an isomorphism from $W$ to $V$.
	\item \textbf{Transitive}: If $T : V \to W$ and $S : W \ to X$ are isomorphism, then $S \circ T$ is an isomorphism from $V$ to $X$.
\end{itemize}

\begin{theorem}{\cite{math2601_notes}}{}
	Finite dimension vector spaces $V$ and $W$ over $\mathbb{F}$ are isomorphic if and only if they have the same dimension.
\end{theorem}

\subsubsection{Spaces associated to matrices}

Let $A$ be a $m \times n$ matrix over field $\mathbb{F}$, and define a map $T : \mathbb{F}^n \to \mathbb{F}^m$ by $T(\mathbf{x}) = A \mathbf{x}$.

The kernel, image, nullity, and rank of $A$ are by definition the same as those of this map $T$.

Now suppose $A$ has columns $\mathbf{c}_1, \ldots, \mathbf{c}_n$ (all in $\mathbb{F}^m$). Then
\begin{align*}
	\operatorname{im}(A) &= \{ A \mathbf{x} : \mathbf{x} \in \mathbb{F}^n \} \\
	&= \{ x_1 \mathbf{c}_1 + \cdots + x_n \mathbf{c}_n : x_i \in \mathbb{F} \} \\
	&= \operatorname{span}(\{ \mathbf{c}_1, \ldots, \mathbf{c}_n \})
\end{align*}
That is, $\operatorname{im}(A)$ is the space spanned by the columns of $A$: the \textbf{column space} of $A$, $\operatorname{col}(A)$, a subspace of $\mathbb{F}^m$.

The rank of $A$ is thus the dimension of the column space of $A$.

As an immediate corollary of the Rank-Nullity Theorem \cite{th:rank_nullity} for maps, we have the Rank-Nullity Theorem for matrices:

\begin{theorem}{Rank-Nullity Theorem (matrices) \cite{math2601_notes}}{rank_nullity_matrices}
	For $A \in M_{m,n}(\mathbb{F})$, $\operatorname{rank}(A) + \operatorname{nullity}(A) = n$, the number of columns of $A$.
\end{theorem}

The \textbf{row space} of $A$, $\operatorname{row}(A)$, is defined similarly as the space spanned by the rows: it is a subspace of $\mathbb{F}^n$.

\begin{theorem}{\cite{math2601_notes}}{}
	Let $A \in M_{m,n}(\mathbb{F})$. The spaces $\operatorname{row}(A)$ and $\operatorname{col}(A)$ have the same dimension.
\end{theorem}

\subsubsection{Matrix of a linear map}

\begin{theorem}{\cite{math2601_notes}}{linear_unique_matrix}
	Let $V, W$ be two finite dimensional vector spaces over $\mathbb{F}$. Suppose $\dim (V) = q$ and $V$ has basis $\mathcal{B}$ and also $\dim (W) = p$ and $W$ has basis $\mathcal{C}$.

	If $T : V \to W$ is linear then there is a unique $A \in M_{p,q}(\mathbb{F})$ with
	\begin{equation}
		[T(\mathbf{v})]_{\mathcal{C}} = A [\mathbf{v}]_{\mathcal{B}} . \label{matrix_linear_map}
	\end{equation}

	Conversely, for any $A \in M_{p,q}(\mathbb{F})$, equation \eqref{matrix_linear_map} defines a unique linear map from $V$ to $W$.
\end{theorem}

We call $A$ in the above theorem \ref{th:linear_unique_matrix} the \textbf{matrix of $T$ with respect to $\mathcal{B}$ and $\mathcal{C}$}.

A useful notation is to denote this matrix by $[T]_{\mathcal{C}}^{\mathcal{B}}$ and then equation \eqref{matrix_linear_map} takes the form
$$ [T(\mathbf{v})]_\mathcal{C} = [T]_{\mathcal{C}}^{\mathcal{B}} [\mathbf{v}]_{\mathcal{B}} . $$

\begin{corollary}{\cite{math2601_notes}}{}
	If $\dim (V) = q$ and $\dim (W) = p$ then $\dim (L(V, W)) = pq$.
\end{corollary}

\begin{theorem}{\cite{math2601_notes}}{}
	Let $T : V \to W$ and $S : W \to X$ be linear maps between vector spaces and suppose $V$, $W$, and $X$ have bases $\mathcal{A}$, $\mathcal{B}$, and $\mathcal{C}$ respectively.

	Then the matrix of $S \circ T : V \to X$ is the product of the matrices of $T$ and $S$, all taken with respect to the appropriate bases:
	$$ [S \circ T]_\mathcal{C}^\mathcal{A} = [S]_\mathcal{C}^\mathcal{B} \cdot [T]_\mathcal{B}^\mathcal{A} . $$
\end{theorem}

\begin{corollary}{\cite{math2601_notes}}{}
	If $T : V \to W$ is linear and invertible, the matrix of $T^{-1}$ is the inverse of the matrix of $T$.

	Thus the group of invertible linear maps on an $n$-dimensional vector space over $\mathbb{F}$ is isomorphic to $\operatorname{GL}(n, \mathbb{F})$.
\end{corollary}

\begin{definition}{Change of basis \cite{math2601_notes}}{change_basis}
	If vector space $V$ has two bases $\mathcal{B}$ and $\mathcal{C}$, the matrix $[\operatorname{id}]_\mathcal{C}^\mathcal{B}$ of the identity map is called the \textbf{change of basis matrix (from $\mathcal{B}$ to $\mathcal{C}$)}.
\end{definition}

The point here is that a change of basis matrix can be used to change coordinates:
$$ [\mathbf{v}]_\mathcal{C} = [\operatorname{id}]_\mathcal{C}^\mathcal{B} [\mathbf{v}]_\mathcal{B} . $$

We can visualise a change of basis with a \textbf{commutative diagram} like so:
$$
	\begin{tikzcd}[row sep=huge, column sep=huge]
		\begin{tabular}{c}
			basis $\mathcal{S}$ \\
			$T : V$
		\end{tabular} \arrow[r, "A"]
			& \begin{tabular}{c}
				basis $\mathcal{S}'$ \\
				$W$
			\end{tabular} \\
		\begin{tabular}{c}
			$T : V$ \\
			basis $\mathcal{B}$
		\end{tabular} \arrow[u, "P"] \arrow[r, "M"']
			&  \begin{tabular}{c}
				$W$ \\
				basis $\mathcal{C}$
			\end{tabular} \arrow[u, "Q"']
	\end{tikzcd}
$$

In the diagram,
\begin{enumerate}[a)]
	\item $\mathcal{S}$ and $\mathcal{S}'$ are standard bases.
	\item The vertical arrows represent identity maps, so $P = [\operatorname{id}_V]_\mathcal{S}^\mathcal{B}$ and $Q = [\operatorname{id}_W]_{\mathcal{S}'}^\mathcal{C}$.
	\item The matrix $A = [T]_{\mathcal{S}'}^\mathcal{S}$, which we assume is easy to find.
	\item The matrix $M = [T]_\mathcal{C}^\mathcal{B}$, and
	\item $M = Q^{-1} A P$, i.e. $[T]_\mathcal{C}^\mathcal{B} = [\operatorname{id}_W]_\mathcal{C}^{\mathcal{S}'} [T]_{\mathcal{S}'}^\mathcal{S} [\operatorname{id}_V]_\mathcal{S}^\mathcal{B}$.
\end{enumerate}

\begin{lemma}{\cite{math2601_notes}}{}
	Let $T : V \to W$ be a linear map between finite dimensional vector spaces over $\mathbb{F}$ and $A$ its matrix with respect to any two bases in $V$ and $W$.

	Then
	$$ \operatorname{nullity} (A) = \operatorname{nullity} (T) \quad \text{and} \quad \operatorname{rank} (A) = \operatorname{rank} (T) . $$
\end{lemma}

\begin{definition}{Invariant subspace \cite{math2601_notes}}{invariant_subspace}
	Let $V$ be a vector space over $\mathbb{F}$ and $T : V \to V$ a linear map.

	If $X \leq V$ such that $T(X) \leq X$, we call $X$ an \textbf{invariant subspace of $T$}.
\end{definition}

\begin{theorem}{\cite{math2601_notes}}{}
	Let $T : V \to V$ be a linear map on a finite dimensional vector space. Suppose $V = X \oplus Y$ with both $X$ and $Y$ invariant subspaces of $T$ with dimensions $p$ and $q$ respectively.

	Then there is a basis $\mathcal{B}$ for $V$ in which the matrix $[T]_\mathcal{B}^\mathcal{B}$ of $T$ is of the form
	$$ 
		[T]_\mathcal{B}^\mathcal{B} =
		\begin{pmatrix}
			A & \mathbf{0} \\
			\mathbf{0} & B
		\end{pmatrix},
	$$
	with $A$ a $p \times p$ and $B$ a $q \times q$ matrix.
\end{theorem}

\subsubsection{Isomorphisms}

\begin{theorem}{\cite{math2601_notes}}{}
	Let $T : V \to W$ be a linear map between finite-dimensional spaces, and let $A$ be the matrix of $T$ (with respect to any bases).

	Then $T$ is invertible if and only if $A$ is invertible.

	If this is the case, then the matrix of $T^{-1}$ (with respect to the same bases) is $A^{-1}$.
\end{theorem}

\begin{theorem}{Normal form \cite{math2601_notes}}{normal_form}
	Let $T : V \to W$ be a linear map between vector spaces over $\mathbb{F}$ where $\dim V = p$, $\dim W = q$, and $\operatorname{rank}(T) = r$.

	Then there are bases $\mathcal{B}$ of $V$ and $\mathcal{C}$ of $W$ with respect to which the matrix of $T$ takes the form
	$$
		N_{q,p;r} =
		\begin{pmatrix}
			I_r & \mathbf{0} \\
			\mathbf{0} & \mathbf{0}
		\end{pmatrix}
		\in M_{q,p} (\mathbb{F}) ,
	$$
	where $I_r$ is the $r \times r$ identity matrix and $\mathbf{0}$ are suitably sized zero matrices.
\end{theorem}

\begin{corollary}{\cite{math2601_notes}}{}
	Let $A \in M_{q,p} (\mathbb{F})$ be rank $r$. Then there is a $q \times q$ invertible matrix $Q$ and a $p \times p$ invertible matrix $P$ such that
	$$ Q^{-1} A P = N_{q,p;r} . $$
\end{corollary}

The $q \times p$ rank $r$ matrix $N_{q,p;r}$ in these resuls is called the \textbf{normal form} of the map or matrix.
% \section{Statistics}

\subsection{Probability}

\subsubsection{Probability spaces}

\begin{definition}{Probability space \cite{math2901_notes}}{probability_space}
In probability theory, a \textbf{probability space} or a \textbf{probability triple} $(\Omega, \mathcal{F}, P)$ is a mathematical construct that provides a formal model of a random process or ``experiment''.

A probability space consists of three elements:
\begin{enumerate}
	\item A \textbf{sample space} $\Omega$, which is the set of all possible outcomes.
	\item An \textbf{event space}, which is a set of events $\mathcal{F}$, an event being a set of outcomes in the sample space.
	\item A \textbf{probability function}, which assigns each event in the event space a probability, which is a number between 0 and 1.
\end{enumerate}
\end{definition}

The minimal assumption that we impose on $\mathcal{F}$ is that it should be an object called a $\sigma$-algebra.

\begin{definition}{$\sigma$-algebra \cite{wikipedia_sigma_algebra}}{sigma-algebra}
Let $X$ be some set, and let $\mathcal{P}(X)$ represent its power set. Then a subset $\Sigma \subseteq \mathcal{P}(X)$ is called a $\bm{\sigma}$\textbf{-algebra} if it satisfies the following three properties:

\begin{enumerate}
	\item $X$ is in $\Sigma$, and $X$ is considered to be the universal set in the following context;
	\item $\Sigma$ is closed under complementation: if $A$ is in $\Sigma$, then so is its complement $X \setminus A$;
	\item $\Sigma$ is closed under countable unions: if $A_1, A_2, A_3, \ldots$ are in $\Sigma$, then so is $A = A_1 \cup A_2 \cup A_3 \cup \ldots$
\end{enumerate}
\end{definition}

Given a sample space $(\Sigma, \mathcal{F})$, a probability function $\mathbb{P}$ can be defined in the following way: to every event $A \in \mathcal{F}$ we assign a number $\mathbb{P} (A)$, the \textbf{probability that $A$ occurs}. The function $\mathbb{P}$ must satisfy the axioms

\begin{enumerate}
	\item $\mathbb{P} (A) \geq 0$ for each $A \subset \Omega$;
	\item $\mathbb{P} (\Omega) = 1$;
	\item if $A_1, A_2, \ldots$ are mutually exclusive (disjoint), i.e.
	$$ A_i \cap A_j = \emptyset \text{ for all } i, j \text{ with } i \not = j  ,$$
	then
	$$ \mathbb{P} \left( \bigcup_{i = 1}^{\infty} A_i \right) = \sum_{i = 1}^{\infty} \mathbb{P} (A_i) . $$
\end{enumerate}

\begin{lemma}{\cite{math2901_notes}}{}
\begin{enumerate}
	\item If $A_1, A_2, \ldots, A_k$ are mutually exclusive, then
	$$ \mathbb{P} \left( \bigcup_{i = 1}^{k} A_i \right) = \sum_{i = 1}^{k} \mathbb{P} (A_i) ; $$
	\item $\mathbb{P} (\emptyset) = 0$;
	\item for any $A \subseteq \Omega$, it holds that $0 \leq \mathbb{P} (A) \leq 1$ and $\mathbb{P} (\bar A) = 1 - \mathbb{P} (A)$;
	\item if $B \subset A$, then $\mathbb{P} (B) \leq \mathbb{P} (A)$.
\end{enumerate}
\end{lemma}

\subsubsection{Monotonic sequences of events}

\begin{theorem}{Continuity properties \cite{math2901_notes}}{continuity_properties}
If $A_1, A_2, \ldots$ is an increasing sequence of events, i.e., $A_1 \subset A_2 \subset \dots$, then
$$ \lim_{n \to \infty} \mathbb{P} (A_n) = \mathbb{P} \left( \bigcup_{n = 1}^{\infty} \right) . $$
We say that $\mathbb{P}$ is \textbf{continuous from below}.

If $A_1, A_2, \ldots$ is a decreasing sequence of events, i.e., $A_1 \supseteq A_2 \supseteq \dots$, then
$$ \lim_{n \to \infty} \mathbb{P} (A_n) = \mathbb{P} \left( \bigcap_{n = 1}^{\infty} \right) . $$
We say that $\mathbb{P}$ is \textbf{continuous from above}.
\end{theorem}

\subsubsection{Conditional probability}

\begin{definition}{Conditional probability \cite{math2901_notes}}{conditional_probability}
The \textbf{conditional probability} that an event $A$ occurs, given that an event $B$ has occurred, is
$$ \mathbb{P} (A | B) = \dfrac{\mathbb{P} (A \cap B)}{\mathbb{P} (B)} \text{ if } \mathbb{P} (B) \not = 0 . $$
\end{definition}

\begin{lemma}{\cite{math2901_notes}}{conditional_probability}
$$ \mathbb{P} (A|B) = \mathbb{P} (A) \iff \mathbb{P} (B|A) = \mathbb{P} (B) $$
\end{lemma}

\subsubsection{Independent events}

\begin{definition}{Independent events \cite{math2901_notes}}{independent_events}
For a countable sequence of events $\{A_i\}$, the events are  
\begin{itemize}
	\item \textbf{pairwise independent} if
$$ \mathbb{P} (A_i \cap A_j) = \mathbb{P} (A_i) \mathbb{P} (A_j) \text{ for all } i \not = j ; $$

	\item \textbf{(mutually) independent} if for any sub-collection $A_{i_1}, \ldots, A_{i_n}$ we have
$$ \mathbb{P} \left( \bigcap_{j = 1}^n A_{i_j} \right) = \prod_{j = 1}^n \mathbb{P} (A_{i_j}) . $$
\end{itemize}
\end{definition}

Note that for any two events $A$ and $B$, $\mathbb{P} (A \cap B) = \mathbb{P} (A|B) \mathbb{P} (B)$, so $A$ and $B$ are independent if and only if the equalities in lemma \ref{lem:conditional_probability} are true.

Also note that mutual independence implies pairwise independence (but not vice versa).
% \restoregeometry

\chapter{Stochastic processes}

\section{The exponential distribution and the Poisson process}

\textbf{Common random variables}:
\begin{itemize}
    \item $X_i$, inter-arrival times, exponentially distributed
    \item $S_n = \sum_{i = 1}^n X_i$, time of $n$th arrival, gamma-distributed (see property 1 below)
    \item $N(t)$, number of arrivals by time $t$, Poisson-distributed.
\end{itemize}

\textbf{Properties of the exponential distribution}:
    \begin{enumerate}
        \item Suppose we have $n$ lots of i.i.d. random variables
            $$ X_i \sim \operatorname{Exp}(\lambda) . $$
        Then
            $$ S_n = \sum_{i = 1}^n X_i \sim \Gamma(n, \lambda) . $$
        \item Let $X_1$ and $X_2$ be independent exponential random variables with respective rates $\lambda_1$ and $\lambda_2$. Then
            $$ \mathbb{P}(X_1 < X_2) = \frac{\lambda_1}{\lambda_1 + \lambda_2} . $$
        \item \textbf{Minimum exponential random variable}: suppose we have $n$ lots of independently distributed random variables
            $$ X_i \sim \operatorname{Exp}(\lambda_i) . $$
        Then
            $$ X_{(1)} \sim \operatorname{Exp}\biggl( \sum_{i = 1}^n \lambda_i \biggr) , $$
        where $X_{(1)} = \min_i X_i$.
    \end{enumerate}

\textbf{Other facts}:
    \begin{enumerate}
        \item From Q1, Week 5: if $X \sim \operatorname{Exp}(\lambda_1)$ then
            $$ \frac{X}{\lambda_2} \sim \operatorname{Exp}(\lambda_1 \lambda_2) . $$
    \end{enumerate}
    
\textbf{Stationary increment property}: for every $0 \leq t_1 \leq t_2$, we have
    $$ N(t_2) - N(t_1) \sim N(t_2 - t_1) . $$
\textbf{Independent increment property}: if $(t_1, t_2]$ and $(t_3, t_4]$ are disjoint intervals, then $(N(t_2) - N(t_1))$ and $(N(t_4) - N(t_3))$ are independent.

\loadgeometry{margins}

\begin{problem}{Q4, wk5}{}

    \marginnote{Minimum exponential random variable.}

    One hundred items are simultaneously put on a life test. Suppose that the lifetime of the individual items are independent exponential random variables with mean 200 hours. The test will end when there have been a total of 5 failures. If $T$ is the time at which the test ends, find
    \begin{enumerate}[a)]
        \item $\mathbb{E}(T)$, and
        \item $\operatorname{Var}(T)$.
    \end{enumerate}

    \tcblower

    Let $T_i$ be the time between the $(i -1)$th and $i$th failure. By property 3 in the preamble, the minimum variable out of 100 items is
        $$ T_1 \sim \operatorname{Exp} \biggl( \frac{100}{200} \biggr) , $$
    and generalising,
        $$ T_i \sim \operatorname{Exp} \biggl( \frac{101 - i}{200} \biggr) . $$
    Then
        \begin{enumerate}[a)]
            \item
                $$ \mathbb{E}(T) = \sum_{i = 1}^5 \mathbb{E}(T_i)
                    = \sum_{i = 1}^5 \frac{200}{101 - i}
                    \approx 10.21 \ \text{hours} $$
            \item 
                $$ \operatorname{Var}(T) = \sum_{i = 1}^5 \operatorname{Var}(T_i)
                    = \sum_{i = 1}^5 \frac{200^2}{(101 - i)^2}
                    \approx 20.84 \ \text{hours}^2 $$
    \end{enumerate}

\end{problem}

\begin{problem}{Q7, wk5}{}

    \marginnote{Relationship between $X_i$, $S_n$, and $N(t)$.}

    Let $\{ N(t), t \geq 0 \}$ be a Poisson process with rate $\lambda$. Let $S_n$ denote the time of the $n$th arrival. Find
    \begin{enumerate}[a)]
        \item $\mathbb{E}(S_4)$
        \item $\mathbb{E}(S_4 \ \vert \ N(1) = 2)$
        \item $\mathbb{E}(N(4) - N(2) \ \vert \ N(1) = 3)$
    \end{enumerate}

    \tcblower

    \begin{enumerate}[a)]
        \item By property 1, $S_4 \sim \Gamma(4, \lambda)$, so $\mathbb{E}(S_4) = 4/\lambda$.
        \item By the memoryless property,
            $$ \mathbb{E}(S_4 \ \vert \ N(1) = 2) = 1 + \mathbb{E}(S_2) = 1 + 2/\lambda . $$
        \item We have
            \begin{align*}
                \mathbb{E}(N(4) - N(2) \ \vert \ N(1) = 3)
                    &= \mathbb{E}(N(4) - N(2)) \qquad &&\text{(ind. inc. prop.)} \\
                &= \mathbb{E}(N(4 - 2)) \qquad &&\text{(stat. inc. prop.)} \\
                &= 2 \lambda \qquad &&\text{(mean of Poiss. dist.)}
            \end{align*}
    \end{enumerate}

\end{problem}

\begin{problem}{}{}

    \marginnote{Maximising chances of winning a Poisson game.}

    Events occur according to a Poisson process with rate $\lambda$. Each time an event occurs, we must decide whether or not to stop, with our objective being to stop at the last event to occur prior to some specified time $\tau$, where $\tau > 1/\lambda$.  That is, if an event occurs at time $t, 0 \leq t \leq \tau$,and we decide to stop, then we win if there are no additional events by time $\tau$, and we lose otherwise. If we do not stop when an event occurs and no additional events occur by time $\tau$, then we lose. Also, if no events occur by time $\tau$, then we lose. Consider the strategy that stops at the first event to occur after some fixed time $s, 0 \leq s \leq \tau$.
    \begin{enumerate}[a)]
        \item Using this strategy, what is the probability of winning?
        \item What value of $s$ maximises the probability of winning?
        \item Show that one's probability of winning when using the preceding strategy with the optimal value of $s$ is $1/e$.
    \end{enumerate}

    \tcblower

    \begin{enumerate}[a)]
        \item We win if there is one and only occurrence between times $s$ and $\tau$. Note that $N(t) \sim \mathcal{P}(\lambda t)$, so
            \begin{align*}
                \mathbb{P}(N(\tau) - N(s) = 1) &= \mathbb{P}(N(\tau - s) = 1) \\
                &= \lambda (\tau - s) e^{-\lambda (\tau - s)} .
            \end{align*}
        \item Differentiating the expression in part a) with respect to $s$, and setting to $0$, gives
            $$ e^{-\lambda (\tau - s)} = \lambda (\tau - s) e^{-\lambda (\tau - s)} . $$
        To balance this equation we find that $s = \tau - 1/\lambda$, which is the value of $s$ that maximises our probability of winning.
        \item Setting $s = \tau - 1/\lambda$ in the answer to part a) gives $e^{-1} \approx 0.37$.
    \end{enumerate}

\end{problem}

\begin{problem}{Q21, wk5}{}

    \marginnote{Waiting times at two servers with exponentially distributed waiting times.}

    In a certain system, a customer must first be served by server 1 and then by server 2, with exponentially distributed service times with rate $\mu_1$ and $\mu_2$ respectively.  An arrival finding server 1 busy waits in line for that server. Upon completion of service at server 1, a customer either enters service with server 2 if that server is free or else remains with server 1 (blocking any other customer from entering the service) until server 2 is free. Customers depart the system after being served by server 2.
    \begin{enumerate}[a)]
        \item Suppose that when you arrive there is one customer in the system and that customer is being served by server 1. What is the expected total time you spend in the system?
        \item Suppose now that you arrive to find two others in the system, one being served by server1 and the other by server 2. What is the expected time you spend in the system?
    \end{enumerate}

    \tcblower

    \begin{enumerate}[a)]
        \item Let $T$ be the total time spent in the system, and let $T_1$ and $T_2$ be the respective waiting times at server 1 and 2. Then
            $$ T = 2T_1 + T_2 \mathbf{1}_{\{T_2 > T_1\}} + T_2 , $$
        so
            \begin{align*}
                \mathbb{E}(T) &= 2 \mathbb{E}(T_1) + \mathbb{E}(T_2) \mathbb{P}(T_1 > T_2)
                    + \mathbb{E}(T_2) \\
                    &= \frac{2}{\mu_1} + \frac{1}{\mu_2} \cdot \frac{\mu_1}{\mu_1 + \mu_2}
                        + \frac{1}{\mu_2} \\
                    &= \frac{2}{\mu_1} + \frac{2\mu_1 + \mu_2}{\mu_1 + \mu_2} .
            \end{align*}
        \item Let $T'$ be the total time spent in the system for this particular scenario. Then
            \begin{align*}
                T' &= T \mathbf{1}_{\{T_1 > T_2\}} + (T_2 + T) \mathbf{1}_{\{T_1 < T_2\}} \\
                &= T (\mathbf{1}_{\{T_1 > T_2\}} + \mathbf{1}_{\{T_1 < T_2\}})
                    + T_2 \mathbf{1}_{\{T_1 < T_2\}}
            \end{align*}
        where $T$ is the total time from part a). Therefore
            \begin{align*}
                \mathbb{E}(T') &= \mathbb{E}(T) \times 1 + \mathbb{E}(T_2) \mathbb{P}(T_1 < T2) \\
                &= \frac{2}{\mu_1} + \frac{2 \mu_1}{\mu_2(\mu_1 + \mu_2)} + \frac{1}{\mu_2} .
            \end{align*}
    \end{enumerate}

\end{problem}

\section{Continuous-time Markov chains}

\subsection{Generators}

Generator in terms of probability matrix: $q_{ij} = v_i P_{ij}$.

Note: \underline{rows} of a generator sum to 0.

\begin{problem}{Q14, wk6}{}

    \marginnote{Relationship between probability matrix and generator.}

    Consider the probability transition matrix
        $$ \mathbf{P} = 
            \begin{pmatrix}
                0 & 1/3 & 2/3 \\
                1/3 & 0 & 2/3 \\
                1 & 0 & 0
            \end{pmatrix} .
        $$
    The time spent in states 0, 1, and 2 are exponentially distributed with parameters $1/3$, $1/3$, and $1/6$ respectively. Let $\{X(t), t \geq 0\}$ be the continuous-time Markov chain describing the states over time. Assume the time units are days.
        \begin{enumerate}[a)]
            \item What is the generator matrix of this chain?
            \item Find the vector of long-run proportions.
        \end{enumerate}
    
    \tcblower

    \begin{enumerate}[a)]
        \item We have
                $$ v_0 = \frac{1}{3} \qquad v_1 = \frac{1}{3} \qquad v_2 = \frac{1}{6} $$
            Using the formula in the preamble (e.g. $G_{0, 1} = v_0 P_{0, 1}$) gives
                $$ \mathbf{G} =
                    \begin{pmatrix}
                        -1/3 & 1/9 & 2/9 \\
                        1/9 & -1/3 & 2/9 \\
                        1/6 & 0 & -1/6
                    \end{pmatrix} .
                $$
        \item We want to solve the equation $\mathbf{\pi} \mathbf{G} = \mathbf{0}$, where $\mathbf{\pi}$ is a horizontal vector and $\mathbf{0}$ a vertical one. We get the system of equations
            \begin{align*}
                -\frac{1}{3} \pi_0 + \frac{1}{9} \pi_1 + \frac{1}{6} \pi_2 &= 0 \\
                \frac{1}{9} \pi_0 - \frac{1}{3} \pi_1 &= 0 \\
                \pi_0 + \pi_1 + \pi_2 &= 1 .
            \end{align*}
        (Note that the equation using the third column of $\mathbf{G}$ is redundant.) Solving yields
            $$ \mathbf{\pi} = \biggl( \frac{9}{28}, \frac{3}{28}, \frac{16}{28} \biggr) . $$
    \end{enumerate}

\end{problem}

\subsection{Balance equations}

The balance equation says that
    $$ \text{long run rate into state $i$} = \text{long run rate out of state $i$} . $$
When we consider a birth and death process, this can be greatly simplified to the following:
    $$ \text{long run rate from $i$ to $i + 1$} = \text{long run rate from $i + 1$ to $i$} . $$

\begin{problem}{Q20, wk6}{}

    \marginnote{Balance equations of birth and death processes.}

    A job shop consists of three machines and two repairmen. The amount of time a machine works before breaking down is exponentially distributed with mean 10, and the amount of time it takes a single repairman to fix a machine is exponentially distributed with mean 8.
        \begin{enumerate}[a)]
            \item What is the average number of machines not in use at any given point in time?
            \item What proportion of time are both repairmen busy?
        \end{enumerate}

    \tcblower

    \begin{enumerate}[a)]
        \item We let state $i$ represent the state where $i$ machines have broken down. Then we have (draw a diagram)
            $$ \begin{matrix}
                    q_{01} = \frac{3}{10} & q_{12} = \frac{2}{10} & q_{23} = \frac{1}{10} \\ \\
                    q_{10} = \frac{1}{8} & q_{21} = \frac{2}{8} & q_{32} = \frac{2}{8} .
                \end{matrix}
            $$
        This (see preamble) gives the system of equations
            $$
                \frac{3}{10} \pi_0 = \frac{1}{8} \pi_1 \qquad
                \frac{2}{10} \pi_1 = \frac{2}{8} \pi_2 \qquad
                \frac{1}{10} \pi_2 = \frac{2}{8} \pi_3 \qquad 
                \sum_i \pi_i = 1
            $$
        which yields
            $$
                \pi_0 = \frac{125}{761} \qquad
                \pi_1 = \frac{300}{761} \qquad
                \pi_2 = \frac{240}{761} \qquad
                \pi_3 = \frac{96}{761} .
            $$
        Hence the average number of machines not in use is
            $$ \pi_1 + 2 \pi_2 + 3 \pi_3 = \frac{1068}{761} \approx 1.40 . $$
        \item Only in states 2 and 3 are both repairmen busy, so we are looking for the quantity
            $$ \pi_2 + \pi_3 \approx 0.44 . $$
    \end{enumerate}

\end{problem}

\begin{problem}{}{}

    \marginnote{Long-run proportions of a population that can be totally wiped out in an instant}

    Consider a population for which new members are born according to a Poisson process with constant  rate $\lambda$ (the  birth  rate  does  not  depend  on  the  population  size).   After  a  random amount of time, which is independent of the population size, a disaster occurs and the whole population immediately dies out.  Then the population starts to grow all over again, that is, after an exponential time with rate $\lambda$ the first member is born, and so on until the next disaster occurs. Disasters occur according to a Poisson process with rate $\mu$.
    \begin{enumerate}[a)]
        \item Find the long-run fraction of time the population consists of $n$ members.
        \item What is the long-run average population size?
        \item Determine the mean population size just before a disaster.
    \end{enumerate}

    \tcblower

    \begin{enumerate}[a)]
        \item Using the balance equations (see preamble),
            \begin{align*}
                \lambda \pi_0 &= \mu \biggl( \sum_{k = 1}^\infty \pi_k \biggr) \\
                &= \mu (1 - \pi_0) .
            \end{align*}
        This gives
            $$ \pi_0 = \frac{\mu}{\lambda + \mu} . $$
        Again, using the balance equations we find
            $$ \pi_n = \biggl( \frac{\lambda}{\lambda + \mu} \biggr)^n \biggl( \frac{\mu}{\lambda + \mu} \biggr) . $$
        \item Let $N$ be the size of the population. We have
            \begin{align*}
                \mathbb{P}(N = n) &= \pi_n \\
                &= \biggl( 1 - \frac{\mu}{\lambda + \mu} \biggr)^n \biggl( \frac{\mu}{\lambda + \mu} \biggr)
            \end{align*}
            We can therefore conclude that
                $$ N + 1 \sim \operatorname{Geom}\biggl( \frac{\mu}{\lambda + \mu} \biggr) , $$
            so
                $$ \mathbb{E}(N + 1) = \frac{\lambda + \mu}{\mu} , $$
            which gives
                $$ \mathbb{E}(N) = \frac{\lambda}{\mu} . $$
        \item
            $$ 1 + \frac{\lambda}{\mu} $$
    \end{enumerate}

\end{problem}

\section{Brownian motion}

\begin{problem}{Q2, wk10}{}

    \marginnote{Stationary and independent increment properties.}

    Consider $\{B(t), \ t \geq 0\}$, a standard Brownian motion process. What is the distribution of $B(s) + B(t)$. where $0 \leq s \leq t$?

    \tcblower

    \textit{Note}: try to algebraically \underline{force} independent increments.

    Rewrite as follows:
        \begin{align*}
            B(s) + B(t) &= B(s) + B(t) + B(s) - B(s) \\
            &= 2B(s) + B(t) - B(s) \\
            &= 2B(s) + B(t - s) \qquad \text{(stat. inc. prop.)}
        \end{align*}
    Note that $2B(s)$ and $B(t - s)$ are independent. In addition $\mathbb{E}(2B(s)) = 0$, $\operatorname{Var}(2B(s)) = 4s$, $\mathbb{E}(B(t - s)) = 0$, and $\operatorname{Var}(B(t - s)) = t - s$. Hence
        \begin{align*}
            B(s) + B(t) &\sim \mathcal{N}(0, 4s + t - s) \\
            &\sim \mathcal{N}(0, 3s + t) .
        \end{align*}

\end{problem}


\begin{problem}{Q4, wk10}{}

    \marginnote{Non-symmetrical (?) hitting times.}

    Define $T_a$ as the time it takes a standard Brownian motion to hit the value $a$. What is $\mathbb{P}(T_1 < T_{-1} < T_2)$?

    \tcblower

    Write
    \begin{align*}
        \mathbb{P}(T_1 < T_{-1} < T_2) &= \mathbb{P}(T_1 < T_{-1}) \mathbb{P}(T_{-1} < T_2 \ \vert \ T_1 < T_{-1}) \\
        &= \frac{1}{2} \times \frac{1}{3}  \\
        &= \frac{1}{6} .
    \end{align*}
    In the second-last equality, let $p = \mathbb{P}(T_{-1} < T_2 \ \vert \ T_1 < T_{-1})$. Then, starting at 1, we want to find the probability of going down two steps before going up one, so $p = \frac{1}{2}(\frac{1}{2} + \frac{p}{2})$ which yields $p = \frac{1}{3}$.

\end{problem}

\subsection{Maximum value }

Let $T_a$ be the first time $\{ B(t), \ t \geq 0 \}$ attains $a$. We have
    $$ \mathbb{P}(T_a \leq t) = 2 \phi \biggl( \frac{a}{\sqrt{t}} \biggr) . $$
Let $M(t) = \max_{0 \leq s \leq t} B(s)$ (note that we are considering the interval $[0, t]$). Then for $a > 0$,
    $$ \mathbb{P}(M(t) \geq a) = \mathbb{P}(T_a \leq t) , $$
so
    \begin{align*}
        \mathbb{P}(M(t) \leq a) &= 1 - \mathbb{P}(T_a \leq t) \\
        &= 2\phi \biggl( \frac{a}{\sqrt{t}} \biggr) - 1 . \qquad \text{(VERY STRANGE)}
    \end{align*}

\begin{problem}{Q5, wk10}{}

    \marginnote{Hitting times, maximum variables, absolute value of variables.}

    \begin{enumerate}[a)]
        \item For a fixed $t > 0$, show that $M(t) = \max_{0 \leq s \leq t} B(s)$ and $\lvert B(t) \rvert$ have the same probability distribution.
        \item Deduce that $\mathbb{E}(M(t)) = \mathbb{E}(\lvert B(t) \rvert) = \sqrt{2 t / \pi}$. (\textit{Hint}: $\int_0^\infty z \phi(z) \ dz = \frac{1}{\sqrt{2 \pi}}$.)
    \end{enumerate}

    \tcblower

    \begin{enumerate}[a)]
        \item Consider the following manipulations:
                \begin{align*}
                    \mathbb{P}(\lvert B(t) \rvert \leq a) &= 1 - \mathbb{P}(\vert B(t) \rvert > a) \\
                    &= 1 - \mathbb{P}(\{B > a\} \cup \{B < -a\}) \\
                    &= 1 - 2\mathbb{P}(B > a) \\
                    &= 1 - 2(1 - \mathbb{P}(B \leq a)) \\
                    &= 2 \mathbb{P}(B \leq a) - 1 \\
                    &= 2 \phi \biggl( \frac{a}{\sqrt{t}} \biggr) - 1 .
                \end{align*}
            This final expression is the same distribution function as that of $M(t)$ (see preamble).
        \item Differentiating the distribution function from part a) gives
            $$ \frac{2}{\sqrt{t}} \phi \biggl( \frac{a}{\sqrt{t}} \biggr) . $$
            Note that $a > 0$. Then
                \begin{align*}
                    \mathbb{E}(M(t)) = \int_0^\infty \frac{2 a}{\sqrt{t}} \phi \biggl( \frac{a}{\sqrt{t}} \biggr) \ da
                \end{align*}
            Substituting $z = a / \sqrt{t}$ and using the hint gives
                \begin{align*}
                    \mathbb{E}(M(t)) &= 2 \sqrt{t} \int_0^\infty z \phi (z) \ dz \\
                    &= 2 \sqrt{t} \frac{1}{\sqrt{2 \pi}} \\
                    &= \sqrt{\frac{2t}{\pi}} ,
                \end{align*}
            as required.
    \end{enumerate}
        
\end{problem}




\chapter{Computing}

\section{Graphics}

\subsection{Basics}

\begin{itemize}
	\item Display rate --- number of distinct frames shown per second
	\item Refresh rate --- number of times frames are updated (might be a lot higher than display rate)
\end{itemize}

\section{Algorithms}
\subsection{Dynamic programming}

\begin{shaded}
\textbf{Definition \cite{clrs_algorithms}.} We say that a problem exhibits \textbf{optimal substructure} if optimal solutions to related sub-problems (which may be solved independently) are incorporated into optimal solutions of the problem itself.
\end{shaded}

\begin{shaded}
\textbf{Definition (memoisation (top-down method)) \cite{clrs_algorithms}.} In this approach, we write the procedure recursively in a natural manner, but modified to save the result of each sub-problem (usually in an array or hash table). The procedure now first checks to see whether it has previously solved this sub-problem. If so, it returns the saved value, saving further computation at this level; if not, the procedure computes the value in the usual manner. We say that the recursive procedure has been \textbf{memoised}; it ``remembers'' what results it has computed previously.
\end{shaded}

\begin{shaded}
\textbf{Definition (bottom-up method) \cite{clrs_algorithms}.} This approach typically depends on some natural notion of the ``size'' of a sub-problem, such that solving any particular sub-problem depends only on solving ``smaller'' sub-problems. We sort the sub-problems by size and solve them in size order, smallest first. When solving a particular sub-problem, we have already solved all the smaller sub-problems its solution depends upon, and we have saved their solutions. We  solve each sub-problem only once, and when we first see it, we have already solved all of its prerequisite sub-problems.
\end{shaded}

\subsection{Linear programming}
\begin{shaded}
\textbf{Definition (standard form) \cite{clrs_algorithms}.} We are given a vector $\mathbf{c}$ of coefficients $c_1, c_2, \ldots, c_n$, which is associated with a vector $\mathbf{x}$ of variables $x_1, x_2, \ldots, x_n$. The \textbf{objective function} is formed by taking the dot product of the two vectors:
\begin{align*}
\text{objective function} &= \mathbf{c} \cdot \mathbf{x} \\
&= \sum_{j=1}^{n} c_j x_j \\
&= c_1 x_1 + c_2 x_2 + \ldots + c_n x_n .
\end{align*}

We want to find values for each $x_j$ that maximises the objective function, subject to the constraints
$$
\begin{pmatrix}
a_{11} & a_{12} & \ldots & a_{1n} \\
a_{21} & a_{22} & \ldots & a_{2n} \\
\vdots & \vdots & \ddots & a_{1n} \\
a_{m1} & a_{m2} & \ldots & a_{mn} \\
\end{pmatrix}
\begin{pmatrix} x_1 \\ x_2 \\ \vdots \\ x_n \end{pmatrix}
\begin{matrix} \leq \\ \leq \\ \vdots \\ \leq \end{matrix}
\begin{pmatrix} b_2 \\ b_2 \\ \vdots \\ b_m \end{pmatrix}
$$
and
$$ x_j \geq 0 , $$
for $i = 1, 2, \ldots, m$ and $j = 1, 2, \ldots, n$, where $a_{ij}, b_i \in \mathbb{R}$.

\end{shaded}

\section{Cloud computing}

\subsection{Basics}

\subsubsection{Service models}

From Wikipedia \cite{wikipedia_cloud_computing}:
\begin{itemize}
	\item \textbf{SaaS} (application level) --- CRM, email, virtual desktop, communication, games, ...
	\item \textbf{PaaS} (platform level) --- execution runtime, database, web server, development tools, ...
	\item \textbf{IaaS} (infrastructure level) --- virtual machines, servers, storage, load balancers, network, ...
\end{itemize}
\chapter{Finance}

\section{Investing}
\begin{shaded}
\textbf{Definition (bonds) \cite{investopedia_bond}.} A bond is a fixed income instrument that represents a loan made by an investor to a borrower (typically corporate or governmental).
\begin{itemize}
	\item Bonds are units of corporate debt issued by companies and securitized as tradeable assets.
	\item A bond is referred to as a fixed income instrument since bonds traditionally paid a fixed interest rate (coupon) to debtholders. Variable or floating interest rates are also now quite common.
	\item Bond prices are inversely correlated with interest rates: when rates go up, bond prices fall and vice-versa.
	\item Bonds have maturity dates at which point the principal amount must be paid back in full or risk default.
\end{itemize}
\end{shaded}

\section{Personal finance}

\subsection{Miscellaneous terminology}
\begin{shaded}
\textbf{Definition (term deposit) \cite{investopedia_term_deposit}.} A term deposit is a fixed-term investment that includes the deposit of money into an account at a financial institution. Term deposit investments usually carry short-term maturities ranging from one month to a few years and will have varying levels of required minimum deposits.
\end{shaded}

\subsection{Banking}

\subsubsection{Terminology}
\begin{itemize}
	\item Retail banking --- ``Retail banking, also known as consumer banking, is the typical mass-market banking in which individual customers use local branches of larger commercial banks. Services offered include savings and checking accounts, mortgages, personal loans, debit/credit cards and certificates of deposit (CDs). In retail banking, the focus is on the individual consumer.'' --- Investopedia \cite{investopedia_retail_banking}
	\item Direct banks (such as ING) don't have branch networks and operate remotely. This means they can significantly reduce costs.
	\item Transaction (cheque) vs savings account:
	\begin{itemize}
		\item Transaction: short term, modest interest rates --- used for everyday transactions and paying bills
		\item Savings: long term, higher interest rates --- used for growing savings
	\end{itemize}
\end{itemize}

Avoid Big Four banks and their multitudinous fees.

\subsubsection{ING Orange Everyday}
Deposit \$1000+ every month and make 5+ (settled, not pending) card purchases, and you get
\begin{itemize}
	\item \$0 ING international transaction fees on online or overseas transactions,
	\item free ATMs around Australia and around the world, and
	\item (for Saving Maximiser) up to 1.95\% p.a. variable rate (limited to balances up to \$100,000).
\end{itemize}

In addition, ING Orange Everday charges no monthly fees.

\subsection{Superannuation}

\textit{Note: the following information mostly comes from the Moneysmart \cite{moneysmart_super} and ATO \cite{ato_super} websites.}

\subsubsection{Basics}
\begin{itemize}
	\item employers make compulsory payments to employees' superannuation funds, on top of wages and salary
	\item tax benefits apply
\end{itemize}

\subsubsection{Eligibility}
Must be paid over \$450 per month (before tax) to be eligible.

\subsubsection{Types of super funds}
\begin{itemize}
	\item Accumulation fund --- it...accumulates
	\item Defined benefit fund --- determined by a formula, mostly corporate or public sector funds
\end{itemize}

\subsubsection{Super fund categories}
\begin{itemize}
	\item Retail fund
	\begin{itemize}
		\item often have a wide range of options
		\item may be recommended by financial advisers who get paid a commission
		\item usually range from medium to high cost (may have low cost MySuper alternative)
		\item fund company makes profit
	\end{itemize}
	\item Industry fund
	\begin{itemize}
		\item mostly accumulation funds
		\item usually range from low to medium cost and offer MySuper option
		\item generally not-for-profit
	\end{itemize}
	\item Public sector fund --- for government employees
 	\item Corporate fund --- arranged by employer for employees
	\item SMSF (self-managed super fund)
\end{itemize}

\subsubsection{Super guarantee (SG) contributions}
Employers are required to pay at least 9.5\% of an employee's \textit{ordinary time earnings} into his super account, at least once every three months. Ordinary time earnings include:
\begin{itemize}
	\item over-award payments
	\item commissions
	\item allowances
	\item bonuses
	\item paid leave
\end{itemize}

\subsubsection{Salary sacrifices}
Salary sacrifices are considered employer contributions rather than employee contributions, and are taxed at a maximum rate of 15\% (generally less than marginal tax rate). There is no limit to how much can be contributed, however if contributions exceed a certain threshold, the concessional tax rate will not apply. (This threshold has been \$25,000 since 2017-2018.)

\subsubsection{Personal contributions}
Personal contributions come from after-tax income. There is a non-concessional contributions cap of \$100,000.

\subsubsection{Withdrawal}
Super can be withdrawn:
\begin{itemize}
	\item at age 65
	\item when preservation age is reached (60 for me)
	\item under transition to retirement rules
	\item if experiencing extraordinarily severe conditions financially, medically etc.
\end{itemize}
Some withdrawn money is taxable and some isn't:
\begin{itemize}
	\item Non-concessional (after-tax) contributions --- not taxable upon withdrawal
	\item Concessional (before-tax) contributions --- taxable. (Contributions include employer contributions, salary sacrificed, and tax-deducted personal contributions.) The amount of tax payable depends on and whether tax was paid for it before contribution --- taxable super is separated into taxed and untaxed elements.
\end{itemize}
Super can be withdrawn a number of ways, including as an income stream or as a lump sum. After an income stream starts, no more contributions can be made.

\subsubsection{Miscellaneous}
Having more than one super fund means there are avoidable fees. Better to have all superannuation money wrapped up in a single fund.

\subsubsection{Student Super}
Zero switching fees for \textbf{any} balance.
\begin{itemize}
	\item Balance under \$1,000 --- zero fees
	\item Balance between \$1,000 and \$4,999 --- flat \$39 fee per year and administration fee of 0.99\% p.a.
\end{itemize}

\section{Insurance}
\begin{itemize}
	\item Health (for income over \$90,000)
	\item Income insurance
\end{itemize}

\section{Transportation}

\textbf{Opal card (concession)} --- must carry proof of entitlement (student card). Benefits:

\begin{itemize}
	\item Daily travel cap of \$8
	\item Weekly travel cap of \$25
	\item Sunday travel cap of \$2.8
	\item Weekly travel award --- After 8 journeys, all remaining fares for the week are half price; note that a tap-on must be 60 minutes after the last tap-off to be considered a new journey (Manly ferries exception of 2hrs 10 min).
	\item A 30\% discount on metro/train fares outside of peak hours.
\end{itemize}

\printbibliography

\end{document}