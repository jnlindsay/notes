\documentclass[oneside]{book}

% PACKAGES
\usepackage[utf8]{inputenc}
\usepackage{biblatex}				% bibliography
%\usepackage{xeCJK}					% Chinese
\usepackage{csquotes}				% quotes
\usepackage{parskip}                 % paragraph indentation
\usepackage{fancyhdr}				% fancy headers
\usepackage[yyyymmdd]{datetime}		% proper date format
\usepackage{tgschola}				% Tex Gyre Schola (based on New Century Schoolbook)
\usepackage{amsmath}
\usepackage{amsfonts}
\usepackage{amssymb}					% symbols
\usepackage{amsthm}
\usepackage{tcolorbox}				% boxes for theorems, definitions, etc.
\usepackage{xcolor}
\usepackage{framed}    				% paragraph shading
\usepackage[hidelinks]{hyperref}	% clickable table of contents
\usepackage[italic]{esdiff}			% differentiation notation

% SETTINGS
\colorlet{shadecolor}{lightgray!20}
\setlength{\parindent}{0em}
\addbibresource{bibliography.bib}
\tcbuselibrary{most}

% 'THEOREM' ENVIRONMENTS
\newtcbtheorem[number within=section]{definition}{Definition}%
    {colback=blue!5, colframe=blue!35!black, sharp corners,
    	 fonttitle=\bfseries, parbox=false,
    	 separator sign={\ ---}}{def}

\newtcbtheorem[number within=section]{theorem}{Theorem}%
    {colback=red!5, colframe=red!35!black, sharp corners,
     fonttitle=\bfseries, parbox=false,
     separator sign={\ ---}}{th}
     
\newtcbtheorem[number within=section,
               use counter from=theorem]{lemma}{Lemma}%
    {colback=red!5, colframe=red!35!black, sharp corners,
     fonttitle=\bfseries, parbox=false,
     separator sign={\ ---}}{lem}
     
\tcolorboxenvironment{proof}{% 'proof' from 'amsthm'
	blanker,breakable,left=3mm,
	before skip=10pt,after skip=10pt, parbox=false,
	borderline west={0.5mm}{0pt}{black}}

% REDEFINITIONS

% Credit to @StefanKottwitz on Tex Stack Exchange
% This redefinition allows matrices to be stretched on the fly like so:
% \begin{pmatrix}[1.5] ... \end{pmatrix}
\makeatletter
\renewcommand*\env@matrix[1][\arraystretch]{
  \edef\arraystretch{#1}%
  \hskip -\arraycolsep
  \let\@ifnextchar\new@ifnextchar
  \array{*\c@MaxMatrixCols c}}
\makeatother

% #################################################################

\begin{document}

\tableofcontents

\chapter{Mathematics}
\section{Calculus}

\begin{shaded}
\textbf{Definition (Continuous function) \cite{hubbard_hubbard}.} Let $X \subset \mathbb{R}^n$. Then a mapping $\mathbf{f} : X \to \mathbb{R}^m$ is continuous at $\mathbf{x}_0 \in X$ if
$$ \lim_{\mathbf{x} \to \mathbf{x}_0} \mathbf{f}(\mathbf{x}) = \mathbf{f}(\mathbf{x}_0); $$
$\mathbf{f}$ is continuous on $X$ if it is continuous at every point of $X$.

Equivalently, $\mathbf{f}: X \to \mathbf{R}^m$ is continuous at $\mathbf{x}_0$ if and only if for every $\epsilon > 0$, there exists $\delta > 0$ such that when $|\mathbf{x} - \mathbf{x}_0| < \delta$, then $|\mathbf{f}(\mathbf{x}) - \mathbf{f}(\mathbf{x}_0)| < \epsilon$.
\end{shaded}

\begin{shaded}
\textbf{Definition (Continuity and discontinuity).} Suppose $[a, b] \subset \mathbb{R}$. Consider a function $f : [a, b] \to \mathbb{R}$ and a point $c \in \mathbb{R}$. Supposed the one-sided limits
$$ f(c^-) = \lim_{x \to c^-} f(x) \qquad \textnormal{and} \qquad f(c^+) = \lim_{x \to c^+} f(x) $$
exist.

(1) If
$$ f(c^-) = f(c^+) = f(c) , $$
then $f$ is \textbf{continuous} at $c$.

(2) If
$$ f(c^-) = f(c^+) \not= f(c) $$
--- regardless of whether or not $f(c)$ exists --- then $f$ has a \textbf{removable discontinuity} at $c$.

(3) If
$$ f(c^-) \not= f(c^+) , $$
then $f$ has a \textbf{jump discontinuity} at $c$.
\end{shaded}

\begin{shaded}
\textbf{Definition (Piecewise continuous function) \cite{wikipedia_piecewise}.} A piecewise function is continuous on $[a, b] \subseteq \mathbb{R}$ if

(1) $f(x^-)$ exists for each $x \in (a, b]$;

(2) $f(x^+)$ exists for each $x \in [a, b)$;

(3) $f$ is continuous on $(a, b)$ except at (at most) a finite number of points where there exist jump discontinuities.

Moreover, $f$ is piecewise continuous on $\mathbb{R}$ if it is piecewise continuous on any finite interval $[a, b] \subseteq \mathbb{R}$.
\end{shaded}

\begin{shaded}
\textbf{Definition (Differentiability).} Consider a function $ f : \mathbb{R} \to \mathbb{R} $ and a point $c \in \mathbb{R}$. Let
$$ f(c^-) = \lim_{x \to c^-} f(x) \qquad \textnormal{and} \qquad f(c^+) = \lim_{x \to c^+} f(x) , $$
and let
\begin{align*}
	D^+ f(c) &= \lim_{x \to 0^+} \frac{f(c + h) - f(c^+)}{h} \qquad \text{and} \\
	D^- f(c) &= \lim_{x \to 0^-} \frac{f(c + h) - f(c^-)}{h} .
\end{align*}
Then $f$ is differentiable at $c$ if and only if $f(c^+) = f(c^-)$ and $D^+ f(c) = D^- f(c)$.

Warning: $D^{+/-} f(c)$ is not necessarily the same as $f'(c^{+/-}) = \lim_{x \to c^{+/-}} f'(x)$.
\end{shaded}

\begin{shaded}
\textbf{Definition (Piecewise differentiable function).} A function $f$ is piecewise differentiable on $[a, b] \subseteq \mathbb{R}$ if

(1) $D^- f(x)$ exists for each $x \in (a, b]$;

(2) $D^+ f(x)$ exists for each $x \in [a, b)$;

(3) $f$ is differentiable on $(a, b)$ except at (at most) a finite number of points.

Moreover, $f$ is piecewise differentiable on $\mathbb{R}$ if it is piecewise differentiable on any finite interval $[a, b] \subseteq \mathbb{R}$.
\end{shaded}
\chapter{Computing}

\section{Graphics}

\subsection{Basics}

\begin{itemize}
	\item Display rate --- number of distinct frames shown per second
	\item Refresh rate --- number of times frames are updated (might be a lot higher than display rate)
\end{itemize}

\section{Algorithms}
\subsection{Dynamic programming}

\begin{shaded}
\textbf{Definition \cite{clrs_algorithms}.} We say that a problem exhibits \textbf{optimal substructure} if optimal solutions to related sub-problems (which may be solved independently) are incorporated into optimal solutions of the problem itself.
\end{shaded}

\begin{shaded}
\textbf{Definition (memoisation (top-down method)) \cite{clrs_algorithms}.} In this approach, we write the procedure recursively in a natural manner, but modified to save the result of each sub-problem (usually in an array or hash table). The procedure now first checks to see whether it has previously solved this sub-problem. If so, it returns the saved value, saving further computation at this level; if not, the procedure computes the value in the usual manner. We say that the recursive procedure has been \textbf{memoised}; it ``remembers'' what results it has computed previously.
\end{shaded}


\section{Cloud computing}

\subsection{Basics}

\subsubsection{Service models}

From Wikipedia \cite{wikipedia_cloud_computing}:
\begin{itemize}
	\item \textbf{SaaS} (application level) --- CRM, email, virtual desktop, communication, games, ...
	\item \textbf{PaaS} (platform level) --- execution runtime, database, web server, development tools, ...
	\item \textbf{IaaS} (infrastructure level) --- virtual machines, servers, storage, load balancers, network, ...
\end{itemize}
\chapter{Finance}

\section{Investing}
\begin{shaded}
\textbf{Definition (bonds) \cite{investopedia_bond}.} A bond is a fixed income instrument that represents a loan made by an investor to a borrower (typically corporate or governmental).
\begin{itemize}
	\item Bonds are units of corporate debt issued by companies and securitized as tradeable assets.
	\item A bond is referred to as a fixed income instrument since bonds traditionally paid a fixed interest rate (coupon) to debtholders. Variable or floating interest rates are also now quite common.
	\item Bond prices are inversely correlated with interest rates: when rates go up, bond prices fall and vice-versa.
	\item Bonds have maturity dates at which point the principal amount must be paid back in full or risk default.
\end{itemize}
\end{shaded}

\section{Personal finance}

\subsection{Miscellaneous terminology}
\begin{shaded}
\textbf{Definition (term deposit) \cite{investopedia_term_deposit}.} A term deposit is a fixed-term investment that includes the deposit of money into an account at a financial institution. Term deposit investments usually carry short-term maturities ranging from one month to a few years and will have varying levels of required minimum deposits.
\end{shaded}

\subsection{Banking}

\subsubsection{Terminology}
\begin{itemize}
	\item Retail banking --- ``Retail banking, also known as consumer banking, is the typical mass-market banking in which individual customers use local branches of larger commercial banks. Services offered include savings and checking accounts, mortgages, personal loans, debit/credit cards and certificates of deposit (CDs). In retail banking, the focus is on the individual consumer.'' --- Investopedia \cite{investopedia_retail_banking}
	\item Direct banks (such as ING) don't have branch networks and operate remotely. This means they can significantly reduce costs.
	\item Transaction (cheque) vs savings account:
	\begin{itemize}
		\item Transaction: short term, modest interest rates --- used for everyday transactions and paying bills
		\item Savings: long term, higher interest rates --- used for growing savings
	\end{itemize}
\end{itemize}

Avoid Big Four banks and their multitudinous fees.

\subsubsection{ING Orange Everyday}
Deposit \$1000+ every month and make 5+ (settled, not pending) card purchases, and you get
\begin{itemize}
	\item \$0 ING international transaction fees on online or overseas transactions,
	\item free ATMs around Australia and around the world, and
	\item (for Saving Maximiser) up to 1.95\% p.a. variable rate (limited to balances up to \$100,000).
\end{itemize}

In addition, ING Orange Everday charges no monthly fees.

\subsection{Superannuation}

\textit{Note: the following information mostly comes from the Moneysmart \cite{moneysmart_super} and ATO \cite{ato_super} websites.}

\subsubsection{Basics}
\begin{itemize}
	\item employers make compulsory payments to employees' superannuation funds, on top of wages and salary
	\item tax benefits apply
\end{itemize}

\subsubsection{Eligibility}
Must be paid over \$450 per month (before tax) to be eligible.

\subsubsection{Types of super funds}
\begin{itemize}
	\item Accumulation fund --- it...accumulates
	\item Defined benefit fund --- determined by a formula, mostly corporate or public sector funds
\end{itemize}

\subsubsection{Super fund categories}
\begin{itemize}
	\item Retail fund
	\begin{itemize}
		\item often have a wide range of options
		\item may be recommended by financial advisers who get paid a commission
		\item usually range from medium to high cost (may have low cost MySuper alternative)
		\item fund company makes profit
	\end{itemize}
	\item Industry fund
	\begin{itemize}
		\item mostly accumulation funds
		\item usually range from low to medium cost and offer MySuper option
		\item generally not-for-profit
	\end{itemize}
	\item Public sector fund --- for government employees
 	\item Corporate fund --- arranged by employer for employees
	\item SMSF (self-managed super fund)
\end{itemize}

\subsubsection{Super guarantee (SG) contributions}
Employers are required to pay at least 9.5\% of an employee's \textit{ordinary time earnings} into his super account, at least once every three months. Ordinary time earnings include:
\begin{itemize}
	\item over-award payments
	\item commissions
	\item allowances
	\item bonuses
	\item paid leave
\end{itemize}

\subsubsection{Salary sacrifices}
Salary sacrifices are considered employer contributions rather than employee contributions, and are taxed at a maximum rate of 15\% (generally less than marginal tax rate). There is no limit to how much can be contributed, however if contributions exceed a certain threshold, the concessional tax rate will not apply. (This threshold has been \$25,000 since 2017-2018.)

\subsubsection{Personal contributions}
Personal contributions come from after-tax income. There is a non-concessional contributions cap of \$100,000.

\subsubsection{Withdrawal}
Super can be withdrawn:
\begin{itemize}
	\item at age 65
	\item when preservation age is reached (60 for me)
	\item under transition to retirement rules
	\item if experiencing extraordinarily severe conditions financially, medically etc.
\end{itemize}
Some withdrawn money is taxable and some isn't:
\begin{itemize}
	\item Non-concessional (after-tax) contributions --- not taxable upon withdrawal
	\item Concessional (before-tax) contributions --- taxable. (Contributions include employer contributions, salary sacrificed, and tax-deducted personal contributions.) The amount of tax payable depends on and whether tax was paid for it before contribution --- taxable super is separated into taxed and untaxed elements.
\end{itemize}
Super can be withdrawn a number of ways, including as an income stream or as a lump sum. After an income stream starts, no more contributions can be made.

\subsubsection{Miscellaneous}
Having more than one super fund means there are avoidable fees. Better to have all superannuation money wrapped up in a single fund.

\subsubsection{Student Super}
Zero switching fees for \textbf{any} balance.
\begin{itemize}
	\item Balance under \$1,000 --- zero fees
	\item Balance between \$1,000 and \$4,999 --- flat \$39 fee per year and administration fee of 0.99\% p.a.
\end{itemize}

\section{Insurance}
\begin{itemize}
	\item Health (for income over \$90,000)
	\item Income insurance
\end{itemize}

\section{Transportation}

\textbf{Opal card (concession)} --- must carry proof of entitlement (student card). Benefits:

\begin{itemize}
	\item Daily travel cap of \$8
	\item Weekly travel cap of \$25
	\item Sunday travel cap of \$2.8
	\item Weekly travel award --- After 8 journeys, all remaining fares for the week are half price; note that a tap-on must be 60 minutes after the last tap-off to be considered a new journey (Manly ferries exception of 2hrs 10 min).
	\item A 30\% discount on metro/train fares outside of peak hours.
\end{itemize}

\printbibliography

\end{document}