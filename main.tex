\documentclass[oneside]{book}

% PACKAGES
\usepackage[utf8]{inputenc}
\usepackage{biblatex}				% bibliography
%\usepackage{xeCJK}					% Chinese
\usepackage{csquotes}				% quotes
\usepackage{parskip}				% paragraph indentation
\usepackage{fancyhdr}				% fancy headers
\usepackage[yyyymmdd]{datetime}		% proper date format
\usepackage{tgschola}				% Tex Gyre Schola (based on New Century Schoolbook)
\usepackage{amsmath}
\usepackage{amsfonts}
\usepackage{xcolor}
\usepackage{framed}					% paragraph shading
\usepackage[hidelinks]{hyperref}	% clickable table of contents
\usepackage[italic]{esdiff}			% differentiation notation

% SETTINGS
\colorlet{shadecolor}{lightgray!20}
\setlength{\parindent}{0em}
\addbibresource{bibliography.bib}

% REDEFINITIONS

% Credit to @StefanKottwitz on Tex Stack Exchange
% This redefinition allows matrices to be stretched on the fly like so:
% \begin{pmatrix}[1.5] ... \end{pmatrix}
\makeatletter
\renewcommand*\env@matrix[1][\arraystretch]{
  \edef\arraystretch{#1}%
  \hskip -\arraycolsep
  \let\@ifnextchar\new@ifnextchar
  \array{*\c@MaxMatrixCols c}}
\makeatother

\begin{document}

\tableofcontents

\chapter{Mathematics}
\section{Calculus}

\begin{shaded}
\textbf{Definition (Continuous function) \cite{hubbard_hubbard}.} Let $X \subset \mathbb{R}^n$. Then a mapping $\mathbf{f} : X \to \mathbb{R}^m$ is continuous at $\mathbf{x}_0 \in X$ if
$$ \lim_{\mathbf{x} \to \mathbf{x}_0} \mathbf{f}(\mathbf{x}) = \mathbf{f}(\mathbf{x}_0); $$
$\mathbf{f}$ is continuous on $X$ if it is continuous at every point of $X$.

Equivalently, $\mathbf{f}: X \to \mathbf{R}^m$ is continuous at $\mathbf{x}_0$ if and only if for every $\epsilon > 0$, there exists $\delta > 0$ such that when $|\mathbf{x} - \mathbf{x}_0| < \delta$, then $|\mathbf{f}(\mathbf{x}) - \mathbf{f}(\mathbf{x}_0)| < \epsilon$.
\end{shaded}

\begin{shaded}
\textbf{Definition (Continuity and discontinuity).} Suppose $[a, b] \subset \mathbb{R}$. Consider a function $f : [a, b] \to \mathbb{R}$ and a point $c \in \mathbb{R}$. Supposed the one-sided limits
$$ f(c^-) = \lim_{x \to c^-} f(x) \qquad \textnormal{and} \qquad f(c^+) = \lim_{x \to c^+} f(x) $$
exist.

(1) If
$$ f(c^-) = f(c^+) = f(c) , $$
then $f$ is \textbf{continuous} at $c$.

(2) If
$$ f(c^-) = f(c^+) \not= f(c) $$
--- regardless of whether or not $f(c)$ exists --- then $f$ has a \textbf{removable discontinuity} at $c$.

(3) If
$$ f(c^-) \not= f(c^+) , $$
then $f$ has a \textbf{jump discontinuity} at $c$.
\end{shaded}

\begin{shaded}
\textbf{Definition (Piecewise continuous function) \cite{wikipedia_piecewise}.} A piecewise function is continuous on $[a, b] \subseteq \mathbb{R}$ if

(1) $f(x^-)$ exists for each $x \in (a, b]$;

(2) $f(x^+)$ exists for each $x \in [a, b)$;

(3) $f$ is continuous on $(a, b)$ except at (at most) a finite number of points where there exist jump discontinuities.

Moreover, $f$ is piecewise continuous on $\mathbb{R}$ if it is piecewise continuous on any finite interval $[a, b] \subseteq \mathbb{R}$.
\end{shaded}

\begin{shaded}
\textbf{Definition (Differentiability).} Consider a function $ f : \mathbb{R} \to \mathbb{R} $ and a point $c \in \mathbb{R}$. Let
$$ f(c^-) = \lim_{x \to c^-} f(x) \qquad \textnormal{and} \qquad f(c^+) = \lim_{x \to c^+} f(x) , $$
and let
\begin{align*}
	D^+ f(c) &= \lim_{x \to 0^+} \frac{f(c + h) - f(c^+)}{h} \qquad \text{and} \\
	D^- f(c) &= \lim_{x \to 0^-} \frac{f(c + h) - f(c^-)}{h} .
\end{align*}
Then $f$ is differentiable at $c$ if and only if $f(c^+) = f(c^-)$ and $D^+ f(c) = D^- f(c)$.

Warning: $D^{+/-} f(c)$ is not necessarily the same as $f'(c^{+/-}) = \lim_{x \to c^{+/-}} f'(x)$.
\end{shaded}

\begin{shaded}
\textbf{Definition (Piecewise differentiable function).} A function $f$ is piecewise differentiable on $[a, b] \subseteq \mathbb{R}$ if

(1) $D^- f(x)$ exists for each $x \in (a, b]$;

(2) $D^+ f(x)$ exists for each $x \in [a, b)$;

(3) $f$ is differentiable on $(a, b)$ except at (at most) a finite number of points.

Moreover, $f$ is piecewise differentiable on $\mathbb{R}$ if it is piecewise differentiable on any finite interval $[a, b] \subseteq \mathbb{R}$.
\end{shaded}
\chapter{Computing}

\section{Graphics}

\subsection{Basics}

\begin{itemize}
	\item Display rate --- number of distinct frames shown per second
	\item Refresh rate --- number of times frames are updated (might be a lot higher than display rate)
\end{itemize}

\section{Algorithms}
\subsection{Dynamic programming}

\begin{shaded}
\textbf{Definition \cite{clrs_algorithms}.} We say that a problem exhibits \textbf{optimal substructure} if optimal solutions to related sub-problems (which may be solved independently) are incorporated into optimal solutions of the problem itself.
\end{shaded}

\begin{shaded}
\textbf{Definition (memoisation (top-down method)) \cite{clrs_algorithms}.} In this approach, we write the procedure recursively in a natural manner, but modified to save the result of each sub-problem (usually in an array or hash table). The procedure now first checks to see whether it has previously solved this sub-problem. If so, it returns the saved value, saving further computation at this level; if not, the procedure computes the value in the usual manner. We say that the recursive procedure has been \textbf{memoised}; it ``remembers'' what results it has computed previously.
\end{shaded}


\section{Cloud computing}

\subsection{Basics}

\subsubsection{Service models}

From Wikipedia \cite{wikipedia_cloud_computing}:
\begin{itemize}
	\item \textbf{SaaS} (application level) --- CRM, email, virtual desktop, communication, games, ...
	\item \textbf{PaaS} (platform level) --- execution runtime, database, web server, development tools, ...
	\item \textbf{IaaS} (infrastructure level) --- virtual machines, servers, storage, load balancers, network, ...
\end{itemize}
\include{personal_finance}

\printbibliography

\end{document}