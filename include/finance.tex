\chapter{Finance}

\section{Investing}
\begin{shaded}
\textbf{Definition (bonds) \cite{investopedia_bond}.} A bond is a fixed income instrument that represents a loan made by an investor to a borrower (typically corporate or governmental).
\begin{itemize}
	\item Bonds are units of corporate debt issued by companies and securitized as tradeable assets.
	\item A bond is referred to as a fixed income instrument since bonds traditionally paid a fixed interest rate (coupon) to debtholders. Variable or floating interest rates are also now quite common.
	\item Bond prices are inversely correlated with interest rates: when rates go up, bond prices fall and vice-versa.
	\item Bonds have maturity dates at which point the principal amount must be paid back in full or risk default.
\end{itemize}
\end{shaded}

\section{Personal finance}

\subsection{Miscellaneous terminology}
\begin{shaded}
\textbf{Definition (term deposit) \cite{investopedia_term_deposit}.} A term deposit is a fixed-term investment that includes the deposit of money into an account at a financial institution. Term deposit investments usually carry short-term maturities ranging from one month to a few years and will have varying levels of required minimum deposits.
\end{shaded}

\subsection{Banking}

\subsubsection{Terminology}
\begin{itemize}
	\item Retail banking --- ``Retail banking, also known as consumer banking, is the typical mass-market banking in which individual customers use local branches of larger commercial banks. Services offered include savings and checking accounts, mortgages, personal loans, debit/credit cards and certificates of deposit (CDs). In retail banking, the focus is on the individual consumer.'' --- Investopedia \cite{investopedia_retail_banking}
	\item Direct banks (such as ING) don't have branch networks and operate remotely. This means they can significantly reduce costs.
	\item Transaction (cheque) vs savings account:
	\begin{itemize}
		\item Transaction: short term, modest interest rates --- used for everyday transactions and paying bills
		\item Savings: long term, higher interest rates --- used for growing savings
	\end{itemize}
\end{itemize}

Avoid Big Four banks and their multitudinous fees.

\subsubsection{ING Orange Everyday}
Deposit \$1000+ every month and make 5+ (settled, not pending) card purchases, and you get
\begin{itemize}
	\item \$0 ING international transaction fees on online or overseas transactions,
	\item free ATMs around Australia and around the world, and
	\item (for Saving Maximiser) up to 1.95\% p.a. variable rate (limited to balances up to \$100,000).
\end{itemize}

In addition, ING Orange Everday charges no monthly fees.

\subsection{Superannuation}

\textit{Note: the following information mostly comes from the Moneysmart \cite{moneysmart_super} and ATO \cite{ato_super} websites.}

\subsubsection{Basics}
\begin{itemize}
	\item employers make compulsory payments to employees' superannuation funds, on top of wages and salary
	\item tax benefits apply
\end{itemize}

\subsubsection{Eligibility}
Must be paid over \$450 per month (before tax) to be eligible.

\subsubsection{Types of super funds}
\begin{itemize}
	\item Accumulation fund --- it...accumulates
	\item Defined benefit fund --- determined by a formula, mostly corporate or public sector funds
\end{itemize}

\subsubsection{Super fund categories}
\begin{itemize}
	\item Retail fund
	\begin{itemize}
		\item often have a wide range of options
		\item may be recommended by financial advisers who get paid a commission
		\item usually range from medium to high cost (may have low cost MySuper alternative)
		\item fund company makes profit
	\end{itemize}
	\item Industry fund
	\begin{itemize}
		\item mostly accumulation funds
		\item usually range from low to medium cost and offer MySuper option
		\item generally not-for-profit
	\end{itemize}
	\item Public sector fund --- for government employees
 	\item Corporate fund --- arranged by employer for employees
	\item SMSF (self-managed super fund)
\end{itemize}

\subsubsection{Super guarantee (SG) contributions}
Employers are required to pay at least 9.5\% of an employee's \textit{ordinary time earnings} into his super account, at least once every three months. Ordinary time earnings include:
\begin{itemize}
	\item over-award payments
	\item commissions
	\item allowances
	\item bonuses
	\item paid leave
\end{itemize}

\subsubsection{Salary sacrifices}
Salary sacrifices are considered employer contributions rather than employee contributions, and are taxed at a maximum rate of 15\% (generally less than marginal tax rate). There is no limit to how much can be contributed, however if contributions exceed a certain threshold, the concessional tax rate will not apply. (This threshold has been \$25,000 since 2017-2018.)

\subsubsection{Personal contributions}
Personal contributions come from after-tax income. There is a non-concessional contributions cap of \$100,000.

\subsubsection{Withdrawal}
Super can be withdrawn:
\begin{itemize}
	\item at age 65
	\item when preservation age is reached (60 for me)
	\item under transition to retirement rules
	\item if experiencing extraordinarily severe conditions financially, medically etc.
\end{itemize}
Some withdrawn money is taxable and some isn't:
\begin{itemize}
	\item Non-concessional (after-tax) contributions --- not taxable upon withdrawal
	\item Concessional (before-tax) contributions --- taxable. (Contributions include employer contributions, salary sacrificed, and tax-deducted personal contributions.) The amount of tax payable depends on and whether tax was paid for it before contribution --- taxable super is separated into taxed and untaxed elements.
\end{itemize}
Super can be withdrawn a number of ways, including as an income stream or as a lump sum. After an income stream starts, no more contributions can be made.

\subsubsection{Miscellaneous}
Having more than one super fund means there are avoidable fees. Better to have all superannuation money wrapped up in a single fund.

\subsubsection{Student Super}
Zero switching fees for \textbf{any} balance.
\begin{itemize}
	\item Balance under \$1,000 --- zero fees
	\item Balance between \$1,000 and \$4,999 --- flat \$39 fee per year and administration fee of 0.99\% p.a.
\end{itemize}

\section{Insurance}
\begin{itemize}
	\item Health (for income over \$90,000)
	\item Income insurance
\end{itemize}

\section{Transportation}

\textbf{Opal card (concession)} --- must carry proof of entitlement (student card). Benefits:

\begin{itemize}
	\item Daily travel cap of \$8
	\item Weekly travel cap of \$25
	\item Sunday travel cap of \$2.8
	\item Weekly travel award --- After 8 journeys, all remaining fares for the week are half price; note that a tap-on must be 60 minutes after the last tap-off to be considered a new journey (Manly ferries exception of 2hrs 10 min).
	\item A 30\% discount on metro/train fares outside of peak hours.
\end{itemize}