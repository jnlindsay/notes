\chapter{Computing}

\section{Graphics}

\subsection{Basics}

\begin{itemize}
	\item Display rate --- number of distinct frames shown per second
	\item Refresh rate --- number of times frames are updated (might be a lot higher than display rate)
\end{itemize}

\section{Algorithms}
\subsection{Dynamic programming}

\begin{shaded}
\textbf{Definition \cite{clrs_algorithms}.} We say that a problem exhibits \textbf{optimal substructure} if optimal solutions to related sub-problems (which may be solved independently) are incorporated into optimal solutions of the problem itself.
\end{shaded}

\begin{shaded}
\textbf{Definition (memoisation (top-down method)) \cite{clrs_algorithms}.} In this approach, we write the procedure recursively in a natural manner, but modified to save the result of each sub-problem (usually in an array or hash table). The procedure now first checks to see whether it has previously solved this sub-problem. If so, it returns the saved value, saving further computation at this level; if not, the procedure computes the value in the usual manner. We say that the recursive procedure has been \textbf{memoised}; it ``remembers'' what results it has computed previously.
\end{shaded}

\begin{shaded}
\textbf{Definition (bottom-up method) \cite{clrs_algorithms}.} This approach typically depends on some natural notion of the ``size'' of a sub-problem, such that solving any particular sub-problem depends only on solving ``smaller'' sub-problems. We sort the sub-problems by size and solve them in size order, smallest first. When solving a particular sub-problem, we have already solved all the smaller sub-problems its solution depends upon, and we have saved their solutions. We  solve each sub-problem only once, and when we first see it, we have already solved all of its prerequisite sub-problems.
\end{shaded}

\subsection{Linear programming}
\begin{shaded}
\textbf{Definition (standard form) \cite{clrs_algorithms}.} We are given a vector $\mathbf{c}$ of coefficients $c_1, c_2, \ldots, c_n$, which is associated with a vector $\mathbf{x}$ of variables $x_1, x_2, \ldots, x_n$. The \textbf{objective function} is formed by taking the dot product of the two vectors:
\begin{align*}
\text{objective function} &= \mathbf{c} \cdot \mathbf{x} \\
&= \sum_{j=1}^{n} c_j x_j \\
&= c_1 x_1 + c_2 x_2 + \ldots + c_n x_n .
\end{align*}

We want to find values for each $x_j$ that maximises the objective function, subject to the constraints
$$
\begin{pmatrix}
a_{11} & a_{12} & \ldots & a_{1n} \\
a_{21} & a_{22} & \ldots & a_{2n} \\
\vdots & \vdots & \ddots & a_{1n} \\
a_{m1} & a_{m2} & \ldots & a_{mn} \\
\end{pmatrix}
\begin{pmatrix} x_1 \\ x_2 \\ \vdots \\ x_n \end{pmatrix}
\begin{matrix} \leq \\ \leq \\ \vdots \\ \leq \end{matrix}
\begin{pmatrix} b_2 \\ b_2 \\ \vdots \\ b_m \end{pmatrix}
$$
and
$$ x_j \geq 0 , $$
for $i = 1, 2, \ldots, m$ and $j = 1, 2, \ldots, n$, where $a_{ij}, b_i \in \mathbb{R}$.

\end{shaded}

\section{Cloud computing}

\subsection{Basics}

\subsubsection{Service models}

From Wikipedia \cite{wikipedia_cloud_computing}:
\begin{itemize}
	\item \textbf{SaaS} (application level) --- CRM, email, virtual desktop, communication, games, ...
	\item \textbf{PaaS} (platform level) --- execution runtime, database, web server, development tools, ...
	\item \textbf{IaaS} (infrastructure level) --- virtual machines, servers, storage, load balancers, network, ...
\end{itemize}