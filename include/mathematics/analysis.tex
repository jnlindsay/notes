\chapter{Analysis}

\section{Introduction}

\begin{problem}{p. 2}{}

    \marginnote{Newton's method.}

    We want to approximate the solution of the equation $x^2 - 2 = 0$. Newton's method uses the following calculation:
        $$ x_{n+1} = x_n - \frac{f(x_n)}{f'(x_n)} . $$
    So we have
        \begin{align*}
            f(x) &= x^2 - 2 \\
            f'(x) &= 2x \\
            x_0 &= 1 \qquad \text{(initial guess)} \\
            x_1 &= 3/2 \\
            x_2 &= \ldots
        \end{align*}
\end{problem}

\begin{itemize}
    \item $L^p$ metric on $\mathbb{R}^n$:
        $$ d_p(\mathbf{x}, \mathbf{y})
            = \biggl( \sum_{i = 1}^n 
            \lvert y_i - x_i \rvert^p \biggr)^\frac{1}{p} $$
    \item $L^\infty$ metric on $\mathbb{R}^n$:
        $$ d_\infty(\mathbf{x}, \mathbf{y}) = \max\{ \lvert x_1 - y_1 \rvert,
            \lvert x_2 - y_2 \rvert, \ldots, 
            \lvert x_n - y_n \rvert \} . $$
\end{itemize}

\begin{problem}{p. 6}{}

    \marginnote{$L^p$ metrics on $\mathbb{R}^n$.}

    What does the set
        $$ \{ (x, y) : d_p((x, y), (0, 0)) \leq 1 \} $$
    look like for $p = 1, 2, 3, \infty$?

    \tcblower

    \begin{align*}
        p < 1 \qquad &\text{`pushed in' diamond} \\
        p = 1 \qquad &\text{diamond} \\
        p = 2 \qquad &\text{circle} \\
        p = 3 \qquad &\text{`pushed out' circle} \\
        \vdots \\
        p = \infty \qquad &\text{square}
    \end{align*}

\end{problem}

\begin{itemize}
    \item $L^p$ metric on $C[0, 1]$:
        $$ d_p(f, g) = \biggl( \int_0^1 \lvert f(x) - g(x) \rvert^p dx \biggr)^\frac{1}{p} $$
    \item $L^\infty$ metric on $C[0, 1]$:
        $$ d_\infty(f, g) = \sup_{x \in [0, 1]} \lvert f(x) - g(x) \rvert $$
\end{itemize}

\underline{See lecture notes.} \marginnote{$L^p$ metrics on $C[0, 1]$.}

\section{Sets and cardinality}

\begin{itemize}
    \item \textbf{Cartesian product}. Given $\{A_i\}_{i \in I}$,
        $$ \prod_{i \in I} A_i = \{ (a_i)_{i \in I}
            : a_i \in A_i, \forall i \in I \} $$
        \underline{Note} that $\{A_i\}_{i \in I}$ is a \textit{set of sets}, whereas $(a_i)_{i \in I}$ is a \textit{sequence}.
    \item Formally, the \textbf{Cartesian product} is
        $$ \{ f : I \to \cup_{i \in I} A_i
            : f(i) \in A_i, \forall i \in I \} . $$
        (Try $\mathbb{R} \times \mathbb{N}$.)
\end{itemize}

\begin{problem}{N/A}{}

    \marginnote{Formal definition of Cartesian product (easy).}

    Describe the formal definition of $\mathbb{R} \times \mathbb{N}$.

    \tcblower

    The set of all $f : \{1, 2\} \to \mathbb{R}$ such that
        $$ f(1) \in \mathbb{R} \quad 
            \text{and} \quad f(2) \in \mathbb{N} . $$
\end{problem}

\begin{problem}{N/A}{}

    \marginnote{Formal definition of Cartesian product (hard).}

    Describe the Cartesian product $\prod_{i \in I} A_i$ where $I \in [0, 1]$ and
        $$ A_i = 
            \begin{cases}
                \mathbb{R} \quad \text{for} \quad i \in [0, \frac{1}{2}] \\
                \mathbb{N} \quad \text{for} \quad i \in (\frac{1}{2}, 1]
            \end{cases} $$

    \tcblower

    The set of functions $f : [0, 1] \to \mathbb{R}$ with the restriction that $f(x) \in \mathbb{N}$ when $x > \frac{1}{2}$.
\end{problem}

\begin{itemize}
    \item Set of \textbf{functions} $f : A \to B$:
        \begin{itemize}
            \item $\{ \forall x \in A, \exists ! y \in B : f(x) = y \}$
            \item Set theorist's interpretation: $\{ (x, y) \in A \times B :
                f(x) = y \subseteq A \times B \}$
        \end{itemize}
\end{itemize}

\begin{problem}{p. 17}{}

    \marginnote{Show $\mathbb{N} \sim \mathbb{N} \times \mathbb{N}$ (tedious method).}

    \begin{enumerate}[a)]
        \item Show $\mathbb{N} \sim \mathbb{N} \times \mathbb{N}$.
        \item Show $\mathbb{N} \sim \mathbb{N} \times \mathbb{N} \times \ldots \times \mathbb{N}$.
    \end{enumerate}

    \tcblower

    \begin{enumerate}[a)]
        \item 
    \end{enumerate}

\end{problem}

\begin{problem}{p. 18}{}
    
    \marginnote{Proof of Cantor's theorem. (Similar to Russell's paradox.)}

    Prove Cantor's theorem: $\lvert A \rvert \neq \lvert \mathcal{P}(A) \rvert$.

    \tcblower

    Suppose by contradiction that $\lvert A \rvert = \lvert \mathcal{P}(A) \rvert$. Then there is a bijection $f : A \to \mathcal{P}(A)$.

    Consider the set $S = \{ a \in A : a \not \in f(a) \}$. (\underline{Example}: suppose $f(a) = \{a, b, c\}$ and $f(b) = \{a, c\}$. Then $a \not \in S$ but $b \in S$.)

    Since $f$ is surjective, there exists a $y \in A$ such that $f(y) = S$, since $S \in \mathcal{P}(A)$. But
        \begin{itemize}
            \item $y \in S \implies y \in \mathcal{P}(A) \implies y \not \in S$, and
            \item $y \not \in S \implies y \in S$,
        \end{itemize}
    so we have a paradoxical contradiction
        $$ y \in S \iff y \not \in S . $$
    
\end{problem}

\begin{itemize}
    \item $\lvert A \rvert \leq \lvert B \rvert$ if there is an \textit{injection} $f : A \to B$.
    \item \textbf{Injection} notation: $f : A \xhookrightarrow{} B$
    \item \textbf{Surjection} notation: $f : A \twoheadrightarrow B$
    \item \textbf{Schröder-Bernstein theorem}: let $A$ and $B$ be sets. If there exists injections $f : A \to B$ and $g : B \to A$, then there exists a bijection $h : A \to B$.
\end{itemize}

\begin{problem}{24}{}

    \marginnote{Show that $[0, 1]$ and $[0, 1)$ have the same cardinality.}

    See margin.

    \tcblower

    Consider the two injections:
    \begin{itemize}
        \item $f : [0, 1] \to [0, 1)$ \quad with \quad $x \mapsto \frac{x}{2}$
        \item $g : [0, 1) \to [0, 1]$ \quad with \quad $x \mapsto x$
    \end{itemize}
    By the Schröder-Bernstein theorem, there exists a bijection between $[0, 1]$ and $[0, 1)$, so they have the same cardinality.

\end{problem}

\begin{problem}{N/A}{cardinality_reals_0_1}

    \marginnote{$\lvert [0, 1] \rvert = \lvert \mathbb{R} \rvert$.}

    Prove the margin.

    \tcblower

    See problem \ref{prob:cardinality_set_functions}.

\end{problem}

\begin{problem}{p. 25}{}

    \marginnote{Show that $\mathbb{N}$ and $\mathbb{N} \times \mathbb{N}$ have the same cardinality.}

    See margin.

    \tcblower

    Consider the following injections:
    \begin{itemize}
        \item $f : \mathbb{N} \to \mathbb{N} \times \mathbb{N}$ \quad with \quad $n \mapsto (0, n)$
        \item $g : \mathbb{N} \times \mathbb{N} \to \mathbb{N}$ \quad with \quad $(m, n) \mapsto 2^m 3^n$
    \end{itemize}

    By Schröder-Bernstein, $\lvert \mathbb{N} \rvert = \lvert \mathbb{N} \times \mathbb{N} \rvert$.

\end{problem}

\begin{itemize}
    \item A set $S$ is \textbf{Dedekind-infinite} if there is a bijection from $S$ to a \textit{proper} subset of itself. (Otherwise, $S$ is Dedekind-finite).
        \begin{itemize}
            \item \textit{Assuming AC},
                \begin{align*}
                    \text{Dedekind-infinite} &\iff \text{infinite} \\
                    \text{Dedekind-finite} &\iff \text{finite}
                \end{align*} 
        \end{itemize}
\end{itemize}

\begin{problem}{p. 33}{}

    \marginnote{Show $\mathbb{Q}$ is countable.}

    See margin.

    \tcblower

    Take any $q = \frac{a}{b}$ (simplified) and write it as $(a, b)$. Therefore
        $$ \mathbb{Q} \sim \mathbb{Z} \times \mathbb{Z}
            \sim \mathbb{N} \times \mathbb{N} \sim \mathbb{N} . $$
    Hence $\mathbb{Q}$ is countable.
    
\end{problem}

\begin{itemize}
    \item The set $\mathcal{P}(\mathbb{N})$ can be identified with the set of \textbf{infinite binary bitstrings}. These sets are uncountably infinite.
\end{itemize}

\begin{problem}{p. 36}{}
    
    \marginnote{Show that $[0, 1]$ is uncountable using bitstrings.}

    See margin.

    \tcblower

    We can inject infinite binary bitstrings into $[0, 1]$. But the set of infinite binary bitstrings has the same cardinality as $\mathcal{P}(\mathbb{N})$, which is uncountably infinite.
    
    Therefore $\lvert [0, 1] \rvert \leq \lvert \mathcal{P}(\mathbb{N}) \rvert$, so $[0, 1]$ is uncountably infinite.

\end{problem}

See lecture notes: Theorem 1.3.6. Nice proof.

\begin{problem}{p. 39}{}

    \marginnote{Countable union of countable sets is countable.}

    Show that the set $S$ of finite subsets of $\mathbb{N}$ is countable.

    \tcblower

    Let $S_n$ be the set of finite subsets of $S$ of size $n$. Then show $S = \cup_{n \in \mathbb{N}} S_n$.

    We want to show that $S_n$ is comparable to $\mathbb{N}^n$; i.e. it is countable.
    
    Using the theorem above, we conclude $S$ is countable.
    
\end{problem}

\begin{problem}{Problem sheet 1, Q2}{}

    \marginnote{Explicit bijection between $\mathbb{N}$ and $\mathbb{N} \times \mathbb{N}$.}

    See margin.

    \tcblower

    Use diagonals from upper right to lower left. Then note (where $T_n$ is the $n$th triangular number):

    $$ A_{m, 0} = T_{m+1} = \sum_{i=1}^{m+1} = \frac{(m+1)(m+2)}{2} $$

    Therefore:

    \begin{align*}
        A_{m, n} &= A_{m, n-1} + m + n \\
        &= A_{m, n-2} + (m + n - 1) + (m + n) \\
        &\ \ \vdots \\
        &= A_{m, 0} + (m + 1) + (m + 2) + \cdots + (m + n) \\
        &= A_{m, 0} + nm + \frac{n (n+1)}{2} \\
        &= \frac{(m+1)(m+2)}{2} + nm + \frac{n (n+1)}{2} ,
    \end{align*}

    i.e. our bijection is the function
        $$ (m, n) \mapsto \frac{(m+1)(m+2)}{2} + nm + \frac{n (n+1)}{2} $$
    
\end{problem}

\begin{problem}{Problem sheet 1, Q3}{}

    \marginnote{Transitivity property of cardinality.}

    Prove carefully from the definition that if $\lvert A \rvert \leq \lvert B \rvert$ and $\lvert B \rvert \leq \lvert C \rvert$, then $\lvert A \rvert \leq \lvert C \rvert$.

    \tcblower

    From the definition, $\lvert A \rvert \leq \lvert B \rvert$ implies that there is an injection from $A$ to $B$. Similarly, there is an injection from $B$ to $C$.

    Now the entirety of $A$ is mapped to a subset of $B$, and the entirety of $B$ (and hence $\operatorname{range}(A)$) is mapped to a subset of $C$. Therefore there exists an injection from $A$ to $C$; that is, $\lvert A \rvert \leq \lvert C \rvert$.
    
\end{problem}

\begin{problem}{Problem sheet 1, Q4}{}

    \marginnote{Surjectivity as definition of cardinality inequality.}

    Prove that if $A$ is a nonempty set, then $\lvert A \rvert \leq \lvert B \rvert$ if and only if there exists an onto map $g : B \to A$ (assuming Axiom of Choice).

    \tcblower

    \underline{$f \ \text{injective} \ \implies g \ \text{surjective}$}:
    
    Define $g$ as follows: for each element in $\operatorname{range}(f)$, let $g = f^{-1}$. For those leftover elements $C = B \setminus \operatorname{range}(f)$, (assuming AC) choose any element $x_0 \in A$ and define $g(y) = x_0$ for all $y \in C$. We have therefore constructed $g$ to be a surjective function.

    \underline{$g \ \text{surjective} \ \implies f \ \text{injective}$}:

    For each $x \in A$, there is some set $S_x \subseteq B$ such that $g(y) = x$ for each $y \in S_x$. Define $f$ so that for each $x \in A$, (assuming AC) choose one element $y_0$ from $S_x$ and set $f(x) = y_0$. Then $f$ is injective.
    
\end{problem}

\begin{problem}{Problem set 1, Q5}{ps1_q5}

    \marginnote{$\lvert A \rvert = \lvert A \cup B \rvert$ when $A$ is infinite and $B$ is countable.}
    
    Prove the margin.

    \tcblower

    Interleaving method: take $\{a_n\}_{n \in \mathbb{N}}$ in $A$ and $\{b_n\}_{n \in \mathbb{N}}$ in $B$ and `interleave' $\{b_n\}_{n \in \mathbb{N}}$ between $\{a_n\}_{n \in \mathbb{N}}$.

\end{problem}

\begin{problem}{Problem set 1, Q6}{}

    \marginnote{Countable union of \textit{uncountable} sets.}

    Suppose that $\lvert A_n \rvert = \lvert \mathbb{R} \rvert$, for $n = 1, 2, 3, \ldots$. Prove that
        $$ \biggl\vert \bigcup_{n=1}^\infty A_n \biggr\vert
            = \lvert \mathbb{R} \rvert . $$

    \tcblower

    ?

\end{problem}

\begin{problem}{Problem set 1, Q7}{}

    \marginnote{$\lvert \mathbb{R} \setminus \mathbb{Q} \rvert = \lvert \mathbb{R} \rvert$}

    Prove the margin.

    \tcblower

    Using problem \ref{prob:ps1_q5}, let $A = \mathbb{R} \setminus \mathbb{Q}$ and $B = \mathbb{Q}$. Then $\lvert A \rvert = \lvert A \cup B \rvert$, i.e.
        $$ \lvert \mathbb{R} \setminus \mathbb{Q} \rvert = \lvert \mathbb{R} \rvert , $$
    as required.

\end{problem}

\begin{problem}{Problem sheet 1, Q8}{}

    how's it goin' fellas

\end{problem}

\begin{problem}{Problem sheet 1, Q9 (assignment 1)}{}

    \marginnote{Number of chess games with finite moves is countable.}

    In the game of chess, two players take turns moving pieces on an $8 \times 8$ grid.  The game ends when a state called \textit{checkmate} is reached.  Is the number of different possible chess games countable or uncountable?  (By a ``chess game'', we mean a game which ends in checkmate; we do not consider games that go on forever). Prove your answer.
    
    \underline{Hint}: You do not need to know the specific rules for how chess pieces move or what is considered checkmate to answer the question.

    \tcblower

    [\underline{NOTE}: This answer is unnecessarily complicated and does not consider positions where pieces have been captured and removed from the board.]

    \textit{I claim that the number chess games with finite moves is countable. To prove this, I represent positions and games as sets and sequences, and then use the theorem on slide 38 of Chapter 1.}
        
    There are 64 squares on a chess board and 32 pieces. Assuming all pieces are distinguishable from each other (e.g. Mike and Pradeep are the two white bishops), there are a total of $\frac{64!}{32!}$ ways to put the pieces on the board. The number of \textit{legal} positions $P$ --- that is, positions reachable only by a sequence of legal moves --- is less than $\frac{64!}{32!}$ and is hence finite.

    Let $P_n$ be the set of legal board positions that require $n$ moves to reach, starting from the initial position $p_0 \in P_0$. (Note that there are positions that are in multiple distinct $P_n$; this is irrelevant.) Now $P_n \subset P$ for any number of moves $n$, so $P_n$ is always finite.

    What is a chess game, then? It is simply a sequence of legal transitions between legal positions. We can describe the set $C_m$ of all chess games of $m$ moves as a subset of the Cartesian product
        $$ P_0 \times P_1 \times P_2 \times \cdots \times P_m . $$
    (Note that this product contains games with illegal moves between legal positions.) Now the Cartesian product of finitely many finite sets is finite, so any $C_m$ is finite.

    For the last step, note that the set of all possible games $C$ (with potentially illegal moves) is 
        $$ C = \bigcup_{m \in \mathbb{N}} C_m , $$
    which is a countable union of finite sets. Hence $C$ is countable. Now the number of \textit{legal} chess games of finite moves is a subset of $C$, and is also therefore countable.
\end{problem}

\begin{problem}{Problem sheet 1, Q10 (assignment 1)}{cardinality_set_functions}
    
    \marginnote{Cardinality of a set of functions.}

    Which of the following two sets has a greater cardinality, or are the cardinalities equal?

    \begin{enumerate}
        \item The set $S$ of functions from the interval $[0, 1]$ to $\mathbb{N}$
        \item The set $\mathbb{R}$ of real numbers
    \end{enumerate}

    Prove your answer.

    \tcblower

    I claim that
        $$ \lvert S \rvert
            \geq \lvert \mathcal{P}([0, 1]) \rvert
            > \lvert [0, 1] \rvert
            = \lvert \mathbb{R} \rvert , $$
    and hence that $\lvert S \rvert > \lvert \mathbb{R} \rvert$.

    To prove the first inequality, we can show that there is an injection
        $$ g : \mathcal{P}([0, 1]) \to S 
            \qquad \text{given by} \qquad
            T \mapsto f$$
    such that
        $$ f(y) = 
            \begin{cases}
                1 \qquad \text{if} \ y \in T \\
                0 \qquad \text{otherwise}.
            \end{cases} $$

    The second inequality comes from (the inequality version of) Cantor's theorem.

    To prove the equality, note that the sigmoid function
        $$ f(x) = \frac{e^x}{1 + e^x} $$
    is an injection from $\mathbb{R}$ to $(0, 1) \subset [0, 1]$. On the other hand, the map $x \mapsto x$ injects $[0, 1]$ into $\mathbb{R}$, and so by the Schröder-Bernstein theorem, the two sets are in bijection and hence $\lvert [0, 1] \rvert = \lvert \mathbb{R} \rvert$.

\end{problem}

\section{Metric spaces}

\begin{itemize}
    \item \textbf{Metric} conditions:
        \begin{enumerate}
            \item $d(x, y) = 0 \iff x = y$
            \item $d(x, y) = d(y, x)$
            \item $d(x, y) + d(y, z) \geq d(x, z)$ \quad (triangle inequality)
        \end{enumerate}
\end{itemize}

\begin{problem}{Problem set 2, Q1}{}
    
    \marginnote{Metric spaces: proofs.}

    Which of the following are metric spaces? (If they are not, can you `fix' the problem?)

    \begin{enumerate}[i)]
        \item $X = \mathbb{R}$ with $d(x, y) = \sqrt{\lvert x - y \rvert}$.
        \item $X = c_{00}$, the set of real sequences $\mathbf{x} = (x_k)_{k = 1}^\infty$ which have only finitely many non-zero terms, with $d(x,y) = \sum_{k = 1}^\infty \lvert x_k - y_k \rvert$.
        \item $X = $ all airports served by Qantas, with $d(A, B) = $ minimum travel time on Qantas flights to get from $A$ to $B$.
        \item $X = $ all $100$ digit bit-strings, with $d(v, w) = $ the number of digits where $v$ and $w$ differ.
        \item $X = $ the vector space of all real polynomials, with $d(p, q) = \displaystyle\int_0^3 \lvert p(t) - q(t) \rvert \ dt$.
        \item $X = $ the vector space of all real polynomials, with $d(p, q) = \displaystyle\sum_{k = 0}^\infty \frac{\lvert p(k) - q(k) \rvert}{2^k}$.
    \end{enumerate}

    \tcblower

    \begin{enumerate}[i)]
        \item $(X, d)$ is a metric space. Criteria 1 and 2 are easy. Then, if $x' = x - y$ and $y' = y - z$, we have
            \begin{align*}
                d(x, y) + d(y, z) &= \sqrt{\lvert x - y \rvert}
                    + \sqrt{\lvert y - z \rvert} \\
                &\geq \sqrt{\lvert x - z \lvert} \\
                &= d(x, z) ,
            \end{align*}
        since $\lvert x' \rvert + \lvert y' \rvert \geq \lvert x' + y' \rvert$.
        \item $(X, d)$ is a metric space for similar reasons to part i).
        \item $(X, d)$ is not a metric space because the minimum flying time from $A$ to $B$ may not be the same as from $B$ to $A$. Also, it is not guaranteed that Qantas has flights from every airport to \textit{every other} airport.
        \item $(X, d)$ is a metric space. Criteria 1 and 2 are easy. By contradiction, let us suppose that this space fails to meet criterion 3. Then, there must be at least one $k$ such that $d(x_k, y_k) + d(y_k, z_k) < d(x_k, z_k)$, in which case $d(x_k, z_k) \neq 0$ and hence $d(x_k, z_k) = 1$. Then there are two cases: $x_k = y_k$, and $x_k \neq y_k$. In both cases, we have $d(x_k, y_k) + d(y_k, z_k) = 1$ and so by contradiction transitivity must hold.
        \item $(X, d)$ is a metric space. Proof is similar to part i) with some additional subtlety when dealing with criterion 1: first, note that polynomials are continuous so we cannot have a constant zero polynomial with discontinuous jumps. Second, $d(p, q) = 0$ requires $p$ and $q$ to be zero on the interval $[0, 3]$. However, the only way for $p$ and $q$ to have uncountably many zeros is if they are identically $0$.
        \item $(X, d)$ is a metric space.
    \end{enumerate}
    
\end{problem}

\begin{problem}{Problem set 2, Q4}{}

    \marginnote{$\operatorname{Int}(Y) = Y$ in finite metric spaces.}

    Let $(X, d)$ be a metric space with only finitely many elements. Prove that every subset of $X$ must be open (and closed!).

    \tcblower

    Since $X$ is finite, for any $y \in Y$, there is some $z_y \in X$ that is \textit{closest} to $y$. Setting $\epsilon = \frac{1}{2} \cdot d(y, z_y)$ then yields a ball that sits entirely within $Y$. Hence, $\operatorname{Int(Y) = Y}$ and so every subset of $X$ is open. This also means that every subset is closed.
    
\end{problem}

\begin{problem}{Problem set 2, Q5}{}

    \marginnote{Symmetric difference of sets as metric.}

    Let $X = \mathcal{P}(S)$ where $S$ is finite, and let $d(A, B) = \lvert A \Delta B$. Show that $(X, d)$ is a metric space.
    
    \tcblower

    For criterion 3, use a Venn diagram and label the sub sections $D$ through $F$ (we can ignore) $A \cap B \cap C$. Then the desired inequality is easy to prove.

\end{problem}

\subsection{Interior, closure, boundary}

\begin{itemize}
    \item \textbf{Interior} (greatest open subset):
        $$ \operatorname{Int}(Y) = \{ y \in Y : \exists \epsilon > 0 
            \ \text{such that} \ B(y, \epsilon) \subseteq Y \} . $$
    \item \textbf{Closure} (smallest closed superset): $\operatorname{Cl}(Y) = \operatorname{Int}(Y) \sqcup \operatorname{Bd}(Y)$
    \item A set $Y \subset (X, d)$ is \textbf{open} if $\operatorname{Int}(Y) = Y$, and \textbf{closed} if $Y^c$ is open.
    \item \textbf{Boundary}:
        $$ \operatorname{Bd}(Y) = X \setminus (\operatorname{Int}(Y)
            \cup \operatorname{Int}(Y^c)) . $$
\end{itemize}

\begin{problem}{Problem sheet 2, Q10}{}

    \marginnote{\textit{Limit points}, interior, closure, boundary of union of varied sets (interesting).}

    Let $X = \mathbb{R}$ with the usual metric and let
        $$ Y = \{\sin k\}_{k = 1}^\infty \cup (0, 2)
            \cup \{ 3, 3.1, 3.14, 3.141, 3.1415, \ldots \} . $$
    
    \begin{enumerate}[i)]
        \item What are the interior points of $Y$?
        \item What are the boundary points of $Y$?
        \item What are the limit points of $Y$?
        \item What is the closure of $Y$?
    \end{enumerate}

    \underline{Note}: \textit{limit points} are points in the closure of a set that aren't `isolated'.

    \tcblower

    \begin{enumerate}[i)]
        \item $(0, 2)$
        \item $[-1, 0] \cup \{2\} \cup \{ 3, 3.1, 3.14, \ldots \}
            \cup \{ \pi \}$
        \item $[-1, 2] \cup \{ \pi \}$
        \item $[-1, 2] \cup \{ 3, 3.1, 3.14, \ldots \}
            \cup \{ \pi \}$
    \end{enumerate}

\end{problem}

\begin{problem}{Problem sheet 2, Q11}{}

    \marginnote{Closure theorems.}

    Suppose that $Y$ is a subset of a metric space $(X, d)$.

    \begin{enumerate}[i)]
        \item Prove that $\operatorname{cl(Y)}$ is closed in $(X, d)$.
        \item Prove that $Y$ is closed if and only if $Y = \operatorname{cl}(Y)$.
        \item Let $\mathcal{S}$ denote the set of all closed subsets of $X$ which contain $Y$, that is
            $$ \mathcal{S} = \{ C \subseteq X : C \
            \text{is closed and} \ Y \subseteq C \} . $$
        \begin{enumerate}[(a)]
            \item Explain why $\mathcal{S}$ is always nonemtpy.
            \item Prove that if $C \in \mathcal{S}$ then $\operatorname{cl}(Y) \subseteq C$ (and hence that $\operatorname{cl}(Y)$ is the `smallest' closed set cotaining $Y$).
            \item Prove that $\operatorname{cl}(Y) = \cap_{C \in \mathcal{S}} C$.
        \end{enumerate}
    \end{enumerate}

    \tcblower

    \begin{enumerate}[i)]
        \item $\operatorname{cl}(Y)^c = \operatorname{Int}(Y^c)$ is open, so $\operatorname{cl}(Y)$ is closed.
        \item \begin{align*}
            Y \ \text{closed} &\iff Y^c \ \text{open} \\
                &\iff \operatorname{Int}(Y^c) = Y^c \\
                &\iff \operatorname{cl}(Y) = Y .
                    &&\text{(complement of both sides)}
        \end{align*}
        \item \begin{enumerate}[(a)]
            \item The entire metric space is closed and contains $Y$, so at the very least $X \subseteq \mathcal{S}$.
            \item Since $\operatorname{cl}(Y)$ is closed, Q\ref{prob:closed_alternate} says that the limit of all of its convergent sequences are contained within $\operatorname{cl}(Y)$. But since $Y \subseteq C$, and because $C$ is also closed, the limit of every convergent sequence in $\operatorname{cl}(C)$ is also contained in $C$. Hence $\operatorname{cl}(C) \subseteq C$.
            \item Consider $C_1 \subseteq C_2 \subseteq \cdots \subseteq X$. Then
                \begin{align*}
                    c \in C_1 &\iff c \in C_1
                        \ \text{and} \ c \in C_2
                        \ \text{and} \ \cdots 
                        \ \text{and} \ c \in X \\
                        &\iff c \in \bigcap_{C \in \mathcal{S}} C .
                \end{align*}
            But from part (b), $\operatorname{cl}(Y) = C_1$, so
                $$ \operatorname{cl}(Y)
                    = \bigcap_{C \in \mathcal{S}} C . $$
        \end{enumerate}
    \end{enumerate}
    
\end{problem}

\begin{problem}{Problem sheet 3, Q1 (assignment 2)}{}

    \marginnote{Interior, closure, boundary of subset of $(\ell^1, \lVert \cdot \rVert_1)$.}

    Let $S \subset (\ell^1, \lVert \cdot \rVert_1)$ be the set of sequences $\{ x_n \}_{n=1}^\infty \in \ell^1$ such that
        $$ x_{2n} = x_{2n - 1}, \quad \forall n . $$
    What are the interior, closure, and boundary of $S$? (Prove your answer.)

    \tcblower

    An $\epsilon$-ball around any sequence in $(\ell^1, \lVert \cdot \rVert_1)$ can be written as
        \begin{align*}
            B \bigl( \{ x_n \}_{n=1}^\infty, \epsilon \bigr)
                &= \biggl\{ \{ y_n \}_{n=1}^\infty \in S :
                \Bigl\Vert
                    \{ y_n \}_{n=1}^\infty
                    - \{ x_n \}_{n=1}^\infty 
                \biggr\Vert_1 < \epsilon \Bigr\} \\
            &= \biggl\{
                \{ y_n \}_{n=1}^\infty \in S :
                \sum_{n = 1}^\infty
                \bigl\vert x_n - y_n \bigr\vert < \epsilon
            \biggr\} .
        \end{align*}
    Given \textit{any} sequence $\{ x_n \}_{n=1}^\infty \in S$ and \textit{any} positive $\epsilon$, we can choose some positive $\delta$ less than $\frac{\epsilon}{2}$ and construct a new sequence
        $$ \{ x'_n \}_{n=1}^\infty = (x_1 - \delta, x_1 + \delta, x_3, x_3, x_5, x_5, \ldots) , $$
    which lies outside $S$ but within $\ell^1$, since
        $$ \sum_{n = 1}^\infty
            \bigl\vert x_n - x'_n \bigr\vert
            = 2 \delta < \epsilon < \infty . $$
    However, $\{ x'_n \}_{n=1}^\infty$ also lies in the $\epsilon$-ball of $\{ x_n \}_{n=1}^\infty$, so the interior of $S$ is empty.

    Next, let us consider sequences in $S^c$. For \textit{any} such sequence, choose
        $$ \epsilon < \frac{1}{2} \cdot \inf_{i, j \in \mathbb{Z}^+} \bigl\{ \lvert x_i - x_j \rvert \bigr\}, \quad x_i \neq x_j . $$
    Then the $\epsilon$-ball around $\{ x_n \}_{n=1}^\infty$ is wholly contained within $S^c$; this means that the interior of $S^c$ is itself.

    To conclude, we have
        \begin{align*}
            \operatorname{Bd}(S)
                &= \ell^1 \setminus (\operatorname{Int}(S) \cup \operatorname{Int}(S^c)) \\
            &= \ell^1 \setminus S^c \\
            &= S ,
        \end{align*}
    which also implies that the closure of $S$ is itself.
        
\end{problem}

\subsection{Topology}

\begin{itemize}
    \item $\mathcal{O}(X)$ is the \textbf{topology} of $X$, with properties:
        \begin{enumerate}
            \item $\emptyset, X \in \mathcal{O}(X)$
            \item Union of open sets is open
            \item \textit{Finite} intersection of open sets is open
        \end{enumerate}
\end{itemize}

\begin{problem}{Problem sheet 2, Q8}{}

    \marginnote{Unions of open sets.}

    
    
\end{problem}

\begin{itemize}
    \item \textbf{Minkowski's inequality}
\end{itemize}

\begin{problem}{5}{}
    .
\end{problem}

\begin{problem}{6}{}
    .    
\end{problem}

\begin{problem}{12}{}
    prove property 1 (iff 0)
\end{problem}

\subsection{Convergence}

\marginnote{\underline{IMPORTANT} \underline{NOTE}} `Open' and `closed' are NOT logical opposites, so one must use the $\epsilon$-ball definition \textit{directly} when proving via contradiction or via contrapositive.

\begin{itemize}
    \item \textbf{Cauchy sequences}: in general,
        $$ \text{convergent} \implies \text{Cauchy} , $$
    but not vice versa. The utility of Cauchy sequences is that they don't \textit{presuppose} the existence of a limit. In a complete metric space, though, all Cauchy sequences converge to a limit.
\end{itemize}

\begin{problem}{Problem sheet 2, Q7}{closed_alternate}

    \marginnote{Sequences in $Y$ converge iff $Y$ closed (important).}

    Let $(X, d)$ be a metric space, and let $Y \subseteq X$. Show that $Y$ is closed iff the limit of every convergent sequence $\{y_n\}_{n=1}^\infty \subseteq Y$ is in $Y$.

    \tcblower

    Note that
        \begin{align*}
            Y \ \text{closed} &\iff Y^c \ \text{open} \\
                &\iff \forall x \in Y^c, \ \exists \epsilon > 0
                \ \text{such that} \ B(x, \epsilon)
                \subseteq Y^c \tag{$A$} .
        \end{align*}
    We can also write the contrapositive of the second statement as
        \begin{equation}
            y \in Y \qquad
            \forall \{y_n\} \subseteq Y
                \ \text{where} \ y_n \to y .  \tag{$B$}
        \end{equation}
    Let us first prove $A \implies B$. If $A$ is true, then there is an $\epsilon$-ball around any $x \in Y^c$, so if there is a sequence in $Y$ that converges, its limit must be within $Y$.

    For $A \impliedby B$, let us prove the contrapositive $\neg A \implies \neg B$. We can write $\neg A$ as:
        \begin{equation}
            \exists x \in Y^c
                \ \text{such that} \ B(x, \epsilon) \not \subseteq Y^c
                \ \forall \epsilon > 0 . \tag{$\neg A$}
        \end{equation}
    Then $\neg B$ is:
        \begin{equation}
            \exists \{y_n\} \subseteq Y \ \text{where} \ y_n \to x
            \ \text{such that} \ x \in Y^c . \tag{$\neg B$}
        \end{equation}
    Now if $\neg A$ is true, then we can form an $\epsilon$-ball around $x$ that necessarily intersects with $Y$---therefore, we can construct a sequence wholly within $Y$ that converges to $x$, and so $\neg B$ is also true.
    
\end{problem}

\begin{problem}{Problem sheet 3, Q10 (assignment 2)}{}

    \marginnote{Cauchy sequence in $L^1$. Showing no limit in $L^1$.}

    Let $X = C[0,1]$ with the metric $d_1$. Consider the sequence of piecewise linear functions
    $$ f_n(x) =
        \begin{cases}
            0 & 0 \leq x \leq \frac{1}{2} - \frac{1}{n} \\
            \frac{1}{2} - \frac{n}{4} + \frac{n}{2} x
                & \frac{1}{2} - \frac{1}{n} < x
                < \frac{1}{2} + \frac{1}{n} \\
            1 & \frac{1}{2} + \frac{1}{n} \leq x \leq 1
        \end{cases} ,
        \qquad n \geq 2 .
    $$
    \begin{enumerate}
        \item Sketch the first few functions in the sequence.
        \item Show that this is a Cauchy sequence.
        \item Show that the sequence does not converge. (Do not refer to any larger spaces such as $L^p$ spaces---rather, show directly that the sequence does not converge in the $d_1$ metric to any function in $C[0,1])$.
    \end{enumerate}

    \tcblower

    \begin{enumerate}
        \item Use the following Mathematica code:
        \begin{lstlisting}
f[x_, n_] = Piecewise[{{0, 0 <= x <= 1/2 - 1/n}, {1/2 - n/4 + (n x)/2, 1/2 - 1/n <= x <= 1/2 + 1/n}, {1, 1/2 + 1/n <= x <= 1}}]
Plot[f[x, 4], {x, 0, 1}]
        \end{lstlisting}
        \item Let $m, n \in \mathbb{Z}^+$ and assume without loss of generality that $m < n$. Then
            $$ d_1 \bigl( f_m(x), f_n(x) \bigr)
                = \int_0^1 \bigl\vert f_m(x) - f_n(x) \bigr\vert \ dx . $$
        Instead of solving this integral directly, we can draw a picture of $f_m$ and $f_n$ like so:

        \begin{center}
            \begin{tikzpicture}[scale=0.5]
                \draw (0, 0) -- (10/6, 0)
                    -- (50/6, 10) -- (10, 10);
                \draw (0, 0) -- (3, 0)
                    -- (7, 10) -- (10, 10);
                \draw[densely dotted] (27/13, 40/64) -- (3, 0);
                \draw[dotted] (5, 10) -- (5, 0);
                \node at (10/6, 0) [below] {A};
                \node at (5, 5) [right] {B};
                \node at (3, 0) [below] {C};
                \node at (27/13, 40/64) [left] {M};
            \end{tikzpicture}
        \end{center}

        and find the areas of the two triangles. Let us begin by calculating the perpendicular distance between $\overline{AB}$ and $C$. The line $\overline{AB}$ has equation
            $$ 2mx - 4y + (2 - m) = 0 , $$
        so its perpendicular distance to $C$ is
            $$
                \lvert \overline{MC} \rvert = \frac{\Bigl\vert 
                    2m \cdot (\frac{1}{2} - \frac{1}{n})
                    + (2 -  m) \Bigr\vert}
                    {\sqrt{4m^2 + 16}}
                    = \frac{\Bigl\vert 
                    \frac{2}{n} (n - m) \Bigr\vert}
                    {\sqrt{4m^2 + 16}} 
                    = \frac{\frac{2}{n} (n - m)}
                    {\sqrt{4m^2 + 16}} .
            $$
        (The absolute value signs are redundant since $n > m$.) Next, we have
            $$
                \lvert \overline{AB} \rvert
                    = \sqrt{\biggl( \frac{1}{2} \biggr)^2
                    + \Biggl( \frac{1}{2} 
                    - \biggl( \frac{1}{2} - \frac{1}{m} \biggr)
                    \Biggr)^2 }
                    = \frac{1}{4m} \sqrt{4m^2 + 16} ,
            $$
        and so the two triangles have a total area of
            $$ 
                d_1 \bigl( f_m(x), f_n(x) \bigr)
                    = 2 \cdot \frac{1}{2}
                    \cdot \lvert \overline{AB} \rvert
                    \cdot \lvert \overline{MC} \rvert
                    = \frac{n - m}{2mn} .
            $$
        We can now show that the sequence is Cauchy. Set $K(\epsilon) = \frac{1}{2 \epsilon}$. Then whenever $m, n > K$,
            \begin{align*}
                \frac{n - m}{2mn}
                    &= \frac{1}{2m} - \frac{1}{2n} \\
                    &< \frac{1}{2m} \\
                    &< \frac{1}{2 K} \\
                    &= \frac{1}{2 \frac{1}{2 \epsilon}} \\
                    &= \epsilon .
            \end{align*}
        In other words, for any $\epsilon > 0$, there is a $K(\epsilon)$ such that
            $$ m, n > K \qquad \implies
                \qquad d_1 \bigl( f_m(x), f_n(x) \bigr)
                < \epsilon , $$
        and therefore $\{ f_n \}_{n = 2}^\infty$ is a Cauchy sequence.
        \item \textit{My idea for the following argument was inspired by the MathSoc Discord channel; I improved the idea by being more careful with how I split the integral.} Suppose there is some function $f(x)$ to which the sequence converges. Then we would have
            $$ \lim_{n \to \infty} d_1(f_n(x), f(x)) = 0 , $$
        i.e.
            $$ \lim_{n \to \infty} \int_0^1 \bigl\vert
                f_n(x) - f(x) \bigr\vert \ dx = 0 . $$
        We can split the integral like so:
            \begin{align*}
                \lim_{n \to \infty} \int_0^1 \bigl\vert
                f_n(x) - f(x) \bigr\vert \ dx
                &= \lim_{n \to \infty} \biggl(
                    \int_0^{\frac{1}{2} - \frac{1}{n}}
                    \bigl\vert f_n(x) - f(x) \bigr\vert \ dx \\
                &\qquad \qquad + \int_{\frac{1}{2} - \frac{1}{n}}^
                    {\frac{1}{2} + \frac{1}{n}}
                    \bigl\vert f_n(x) - f(x) \bigr\vert \ dx \\
                &\qquad \qquad + \int_{\frac{1}{2} + \frac{1}{n}}^1 
                    \bigl\vert f_n(x) - f(x) \bigr\vert \ dx \biggr) \\
                &= \int_0^\frac{1}{2}
                    \bigl\vert 0 - f(x) \bigr\vert \ dx
                    + \int_\frac{1}{2}^1
                    \bigl\vert 1 - f(x) \bigr\vert \ dx .
            \end{align*}
        We want this expression to equal zero, which, because of the absolute value signs, necessitates \textit{both} integrals equalling zero. Therefore, when $0 \leq x \leq \frac{1}{2}$, the function $f(x)$ must equal $0$ except for at finitely many points. Similarly, when $\frac{1}{2} \leq x \leq 1$, $f(x)$ must equal $1$---except at finitely many points. However, this means that $f(x)$ will necessarily be discontinuous at $x = \frac{1}{2}$, and so the sequence $\{ f_n \}_{n = 2}^\infty$ does not converge in $C[0, 1]$.
    \end{enumerate}

\end{problem}


\section{Sequences and series of functions}

\section{Topological spaces}

\section{Compactness}